\documentclass[a4paper, 12pt]{article}
\usepackage{index}
\usepackage{amsmath}
\usepackage[T2A]{fontenc}
\usepackage[utf8]{inputenc}
\usepackage[english, russian]{babel}
\usepackage{upquote}
\usepackage{textcomp}
\usepackage[pdftex]{graphicx}
\usepackage{pdfpages}
\usepackage{color}
\usepackage[pdfborder={0 0 0}]{hyperref}
\begin{document}
\begin{enumerate}
\item{Методом наименьших квадратов найти апроксимирующий полином первой и второй степени:
    
    \begin{tabular}{|c|c|}
      \hline
      x & F(x) \\ \hline
      3.0 & 4.0 \\ \hline
      3.2 & 2.0 \\ \hline
      3.4 & 6.0 \\ \hline
      3.6 & 6.0 \\ \hline
      3.8 & 3.0 \\ \hline
      4.0 & 5.0 \\
      \hline
    \end{tabular}

    \emph{Решение:} Для полинома первой степени ответ будем искать в виде $\varphi (x) = ax + b$. В этом случае $\dfrac {\partial \varphi} {\partial a} = x$ и $\dfrac {\partial \varphi} {\partial b} = 1$. Получаем такую систему:
    $$
    \begin{cases}
      \sum \limits_{i = 0}^5 (ax_i+b - y_i)x_i = 0 \\
      \sum \limits_{i = 0}^5 (ax_i + b - y_i) = 0
    \end{cases}
    \Rightarrow
    $$
    $$
    \begin{cases}
      a(9+10.24+11.56+12.96+14.44+16) + \\ + b(3.0 + 3.2 + 3.4 + 3.6 + 3.8 + 4.0) - \\ -(12 + 6.4 + 20.4 + 21.6 + 11.4 + 20) = 0 \\
      a(3.0 + 3.2 + 3.4 + 3.6 + 3.8 + 4.0) + 5b - \\ -(4.0 + 2.0 + 6.0 + 6.0 + 3.0 + 5.0) = 0
    \end{cases}
    \Rightarrow
    $$
    $$
    \begin{cases}
      74.2a + 21b - 91.8 = 0 \\
      21a + 5b - 26 = 0
    \end{cases}
    \Rightarrow
    $$

    $$
    \begin{cases}
      74.2a + 21b = 91.8 \\
      21a + 5b = 26
    \end{cases}
    .
    $$

    $$
    \Delta =
    \left |
      \begin{array}{cc}
        74.2 & 21 \\
        21 & 5
      \end{array}
    \right |
    = 74.2 \cdot 5 - 21 \cdot 21 = 371 - 441 = -70.
    $$


    $$
    \Delta_a =
    \left |
      \begin{array}{cc}
        91.8 & 21 \\
        26 & 5
      \end{array}
    \right |
    = 91.8 \cdot 5 - 21 \cdot 26 = 459 - 546 = -87.
    $$

    $$
    \Delta_b =
    \left |
      \begin{array}{cc}
        74.2 & 91.8 \\
        21 & 26
      \end{array}
    \right |
    = 74.2 \cdot 26 - 91.8 \cdot 21 = 1929.2 - 1927.8 = 1.4;
    $$

    $$
    a = \dfrac {\Delta_a} {\Delta} = \dfrac {-87} {-70} \approx 1.24;
    $$

    $$
    b = \dfrac {\Delta_b} {\Delta} = \dfrac {1.4} {-70} \approx -0.02;
    $$

    \emph{Ответ:} $\varphi (x) = 1.24x - 0.02$.

    Теперь найдём апроксимирующий полином второй степени. Ответ будем искать в виде $\varphi (x) = ax^2 + bx +c$.

    $$
    \begin{cases}
      \sum \limits_{i = 0}^5 (ax_i^2+bx_i + c - y_i)x_i^2 = 0 \\
      \sum \limits_{i = 0}^5 (ax_i^2+bx_i + c - y_i)x_i = 0 \\
      \sum \limits_{i = 0}^5 (ax_i^2+bx_i + c - y_i) = 0 
    \end{cases}
    \Rightarrow
    $$

    $$
    \begin{cases}
      951.96a + 264.6b + 74.2c=326.92 \\
      264.6a + 74.2b + 21c =91.8 \\
      74.2a + 21 b + 5c =26
    \end{cases}
    ;
    $$

    $$
    \Delta =
    \left |
      \begin{array}{ccc}
        951.96 & 264.6 & 74.2 \\
        264.6 & 74.2 & 21 \\
        74.2 & 21 & 5
      \end{array}
    \right |
    = -622.05;
    $$

    $$
    \Delta_a =
    \left |
      \begin{array}{ccc}
        326.92 & 264.6 & 74.2 \\
        91.8 & 74.2 & 21 \\
        26 & 21 & 5
      \end{array}
    \right |
    = 31.92;
    $$

    $$
    \Delta_b =
    \left |
      \begin{array}{ccc}
        951.96 & 326.92 & 74.2 \\
        264.6 & 91.8 & 21 \\
        74.2 & 26 & 5
      \end{array}
    \right |
    = -880.37;
    $$

    $$
    \Delta_c =
    \left |
      \begin{array}{ccc}
        951.96 & 264.6 & 326.92 \\
        264.6 & 74.2 & 91.8 \\
        74.2 & 21 & 26
      \end{array}
    \right |
    = -10.8;
    $$

    $$
    a = \dfrac {\Delta_a} {\Delta} = \dfrac {31.92} {-622.05} \approx -0.05;
    $$

    $$
    b = \dfrac {\Delta_b} {\Delta} = \dfrac {-880.37} {-622.05} \approx 1.42;
    $$

    $$
    c = \dfrac {\Delta_c} {\Delta} = \dfrac {-10.8} {-622.05} \approx 0.02;
    $$

    \emph{Ответ:} $\varphi (x) = -0.05x^2+1.42x+0.02$.
  }
\item{Вычислить определённый интеграл аналитически и численно:
    \begin{equation}\label{eq:1}
    \int_1^2x^3 \ln{x} dx.
    \end{equation}
    Для численного интегрирования использовать формулу прямоугольников, трапеций и формулу Симпсона. Сравнить результаты.

    \emph{Решение:} В начале решим задачу аналитически. Найдём первообразную методом интегрирования по частям:

    $$
    \int x^3 \ln{x} dx = \int \ln{x} d \left ( \dfrac {x^4} {4} \right ) = \dfrac {1} {4} \int \ln{x} d x^4 = 
    $$

    $$
    \dfrac {1} {4} \left ( x^4 \ln{x} - \int x^4 d (\ln x) \right ) = \dfrac {1} {4} \left ( x^4 \ln{x} - \int x^3 d x \right ) = 
    $$

    $$
    \dfrac {1} {4} \left ( x^4 \ln{x} - \dfrac {x^4} {4} \right ).
    $$

    Теперь вычислим точное значение интеграла:

    $$
    \int_1^2x^3 \ln{x} dx = \dfrac {1} {4} \left ( x^4 \ln{x} - \dfrac {x^4} {4} \right ) \Big|_1^2 = \dfrac {1} {4} \left ( 16 \ln 2 - 4 -1 \cdot 0 + \dfrac {1} {4} \right ) = 
    $$
    
    $$
    \dfrac {1} {4} \left ( 16 \ln 2 - \dfrac {15} {4} \right ) = 4 \ln 2 - \dfrac {15} {16} \approx 4 \cdot 0.69 - 0.94 = 1.83.
    $$

    \emph{Ответ:} $\int_1^2x^3 \ln{x} dx \approx 1.83$.

    Теперь подсчитаем значение интеграла \eqref{eq:1} с помощью формулы прямоугольников.

    Формула прямоугольников имеет вид:
    $$
    I \approx \sum_{i=1}^n f(\xi_i)(x_i - x_{i-1})
    $$
    где $I$ --- вычисляемый интеграл, $\xi_i$ --- некоторое значение между $x_i$ и $x_{i-1}$, $i=0,1,...,n$.

    Разобьём отрезок $[1,2]$ на 4 части, получаем такую таблицу:

    \begin{center}
    \begin{tabular}{|c|c|c|}
      \hline
      $n$ & $x_i$ & $\xi_i$ \\ \hline
      0 & 1 &  \\ \hline
      1 & 1.25 & 1.125 \\ \hline
      2 & 1.5 & 1.375 \\ \hline
      3 & 1.75 & 1.625 \\ \hline
      4 & 2 & 1.875 \\ \hline
    \end{tabular}
    \end{center}

    Тогда
    $$I \approx \sum_{i=1}^4 f(\xi_i)(x_i - x_{i-1}) = 0.25 \sum_{i=1}^4 f(\xi_i) =$$
    $$ = 0.25(1.125^3 \ln 1.125 + 1.375^3 \ln 1.375 + 1.625^3 \ln 1.625 +$$
    $$+1.875^3 \ln 1.875) \approx 1.81.$$

    \emph{Ответ:} вычисление интеграла \eqref{eq:1} с помощью формулы прямоугольников даёт значение 1.81.

    Теперь применяем формулу трапеций:
    $$I \approx h \left ( \dfrac {y_0 + y_n} {2} + y_1 + y_2 + ... + y_{n-1} \right );$$
    где $h = \dfrac {b -a} {n}$, здесь $a$, $b$ --- начальные и конечные точки отрезка, $n$ --- количество разбиений, $y_0 = f(x_0)$, $y_1 = f(x_1)$, ..., $y_{n} = f(x_n)$.
    Получаем:
    $$\int_1^2x^3 \ln{x} dx \approx 0.25 \left ( \dfrac {0 + 5.52} {2} + 1.31 + 1.39 + 3 \right ) \approx 2.12.$$
    \emph{Ответ:} вычисление определённого интеграла \eqref{eq:1} с помощью формулы трапеций даёт значение 2.12.

    Теперь нахождение значения определённого интеграла с помощью формулы Симпсона.

    Формула Симпсона:
    $$I \approx \dfrac {h} {3} \left ( y_0 + y_{2m} + 4 \sigma_1 + 2 \sigma_2 \right );$$
    где $\sigma_1 = y_1 + y_3 + ... + y_{2m-1}$, $\sigma_2 = y_2 + y_4 + ... + y_{2m-2}$.
    Получаем:
    $$\int_1^2x^3 \ln{x} dx \approx \dfrac {0.25} {3} \left ( 0 + 5.52 + 4 \cdot (1.31 + 3) + 2 \cdot (1.39) \right ) \approx 2.04.$$
    \emph{Ответ:} вычисление определённого интеграла \eqref{eq:1} с помощью формулы Симпсона даёт значение 2.04.

  }
\end{enumerate}
\end{document}