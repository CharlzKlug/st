\documentclass[10pt]{article}
\usepackage{hyphsubst}
\usepackage[T2A]{fontenc}
\usepackage[utf8]{inputenc}
\usepackage[english,ukrainian,russian]{babel}
\usepackage{mathtools}
\usepackage{amsmath}
\newcommand\aug{\fboxsep=-\fboxrule\!\!\!\fbox{\strut}\!\!\!}
\author{Аметов Имиль}
\title{Введение в системный анализ \\ Домашнее задание №3}
\usepackage[thinlines]{easytable}
\usepackage[left=2cm,right=2cm,top=2cm,bottom=2cm,bindingoffset=0cm]{geometry}
\usepackage{tikz}
\usetikzlibrary{graphs}
\usetikzlibrary{positioning}
%\usepackage{setspace}
%\onehalfspacing
\setlength{\parindent}{5ex}
\setlength{\parskip}{1em}
\begin{document}
\maketitle

\begin{enumerate}
\item{Дайте определение понятию <<Функция системы>>.}

  \emph{Ответ:} Функция системы характеризует проявление её свойств в данной совокупности отношений и представляет собой способ действия системы при взаимодействии с внешней средой.
  
\item{Опишите сходства и различия понятий <<Функция системы>> и <<Цель системы>>. Бывают ли случаи, когда можно сказать что функция системы полностью совпадает с целью системы?}

  \emph{Ответ:} У системы может быть больше чем одной цели, поскольку цель --- это абстрактная модель желаемого состояния системы. Пример подобной системы: задача о рюкзаке. Пусть у нас есть ведро вместимостью 5 литров. И есть 3 разнородных вещества. Причём 1 литр первого вещества стоит 3 условные единицы, 1 литр второго вещества стоит 1 условную единицу, 1 литр третьего вещества стоит 3 условные единицы. Зададим ещё и такое требование: ни одно вещество не должно контактировать с другим. Требуется определить: какое вещество нужно набрать в ведро так, что бы максимизировать стоимость ведра вещества? Очевидно, что нужно набрать либо первое вещество, либо третье вещество. Таким образом, наша система может находиться в двух состояниях: ведро наполненное первым веществом или ведро наполненное третьим веществом.
  Функцией этой системы является ведро, наполненное самым дорогим веществом.

  Бывают ли случаи, когда можно сказать что функция системы полностью совпадает с целью системы? Да, бывают. Тот же самый пример с задачей о рюкзаке, но с вот такой модификацией: есть два вещества, 1-е вещество стоит 3 условные единицы, а второе вещество --- 1 условная единица за литр. И опять же требуется наполнить пятилитровое ведро не смешивая вещества. Ясно, что нужно набрать полное ведро первого вещества. Таким образом функция системы совпадает с её целью.

\item{Что такое функциональная, структурная и функционально-структурная организация? Какую организацию для описания вы бы предпочли для описания системы <<Стул>> и почему?}

  \emph{Ответ:} Функциональная организация --- это совокупность функций системы, связей и отношений между ними.

  Структурная организация --- это совокупность элементов, связей и отношений между ними, т.е. структура системы на основе элементов (и подсистем).

  Функционально-структурная организация системы --- это сплав функциональной и структурной организации. Сюда относятся как особенности строения и взаимодействие системы с внешней средой, так и внутреннее взаимодействие элементов в процессе функционирования системы.

  Обычно стулом считают мебель, предназначенную для сидения одного человека, со спинкой и сиденьем с подлокотниками или без.

  Возьмём стул на четырёх ножках, без частей механизма не сдвигающихся относительно друг друга. Тогда этот стул логично рассматривать с точки зрения структурной организации, причина этому: наличие жёстких связей между всеми состаляющими частями системы <<Стул>>. Здесь налицо: крепления ножек к сиденью (клей, саморезы, болты, гайки и прочее), крепление спинки к ножкам или сиденью (опять же механическое в подавляющем большинстве случаев). То есть имеется некоторая жёсткая, неизменная, инвариантная структура. И в то же время стул как система крайне ограничен (если отбросить эстетическую составляющую) в утилитарно-функциональном плане: его единственная функция в ответ на воздействие внешней среды --- это не развалиться под воздействием некоторой массы.

  А вот если в качестве стула рассматривать катапультное кресло (кресло считается разновидностью стула) К-36ДМ, то в этом случае мы имеем дело с функционально-структурной организаций системы. Причина: катапультное кресло состоит из сиденья с установленной на нём профилированной крышкой с блоком жизнеобеспечения, комбинированного стреляющего механизма (пиромеханизм с электромеханическим затвором), коробки механизма, заголовника, спасательной системы с куполом, уложенным в заголовник, эксплуатационных систем, обеспечивающих удобство размещения и работы члена экипажа в кресле, аварийных систем, обеспечивающих безопасное катапультирование, система фиксации (механизм притягивания плеч, механизм притяга пояса, ограничители разброса рук, механизмы подъёма ног).

  В данном случае каждая подсистема системы <<Катапультное кресло>> несёт свою функцию. Катапультное кресло --- это система в которой задействована не только механика, но и химия и электроника. Благодаря этому катапультное кресло позволяет пилоту выживать как на разных высотах, так и на разных скоростях.

  Здесь система <<Катапультное кресло>> взаимодействует и со внешней средой, в частности, определяет скорость и высоту, и только тогда, когда высота и скорость оказываются приемлемыми --- отсоединяется от пилота.

  То есть кресло проявляет различные свойства в зависимости от различных внешних показателей среды.

\item{На какие 3 группы могут быть разделены все функции системы? Каково предназначение каждой из них.}

  \emph{Ответ:} Все функции, реализуемые сложной системой, могут быть условно разделены на три группы:

  \begin{itemize}
  \item{\emph{целевая функция} --- соответствует основному функциональному назначению системы;}
  \item{\emph{основные функции} --- отражают ориентацию системы и представляют набор макрофункций, реализуемых системой. Они обусловливают существование систем определённого класса.}
  \item{\emph{дополнительные функции} --- расширяют функциональные возможности системы, сферу применения. Обычно они рассматриваются как сервисные, повышающие эффективность и уровень эксплуатации.}
  \end{itemize}

\item{Опишите подробно процесс формирования дерева функций системы. Каково обыденное среднестатистическое количество уровней декомпозиции в дереве функций для любой системы?}

  \emph{Ответ:} Формирование дерева функций системы, оно же декомпозиция функций системы, происходит итеративно в несколько этапов.

  На первом этапе выявляют \emph{целевую функцию} системы. Выявленная целевая функция называется нулевым уровнем в дереве функций. Эта же функция называется функцией первой группы.

  После выявления целевой функции приступают к декомпозиции целевой функции. Результатом являются основные функции и дополнительные функции. Обнаруженные основные и дополнительные функции заносятся на первый уровень дерева функций. Основные и дополнительные функции формируют функции второй группы, что соответствует функциям отдельных подсистем.

  Основные и дополнительные функции в свою очередь декомпозируются на нижележащие функции. Полученные функции заносятся на второй уровень дерева функций и формируют функции третьей группы, соответствующие функциям элементов системы.

  При необходимости осуществляют дальнейшие этапы декомпозиции.

  Число уровней декомпозиции обычно не превосходит 5--7 уровней.

\item{Может ли дерево функций иметь 100 уровней? Хорошо это или плохо? Проиллюстрируйте общее построение подобного дерева на системе <<Ракета>> (в общих словах).}

  \emph{Ответ:} Теоретически дерево функций может иметь неограниченно много уровней. Если система велика и требуется освещение многих функций, то, вполне возможно что будет трудно уложиться в семь уровней и тогда понадобится увеличить количество уровней. С другой стороны, большое количество уровней может вводить путаницу и усложнять восприятие системы. Поэтому, если возникает потребность в чрезмерно большом количестве уровней, то стоит сделать передышку и пересмотреть дерево --- возможно, окажется что некоторые функции не стоит включать в систему или пересмотреть саму декомпозицию функций.

  Общее построение дерева на системе <<Ракета>>:
  
  \begin{tikzpicture}[>=latex]
    \node[draw, rectangle] (moving) {Перемещение в пространстве};
    \node[draw, rectangle, below left=of moving] (engines) {Ракетные двигатели};
    \node[draw, rectangle, below=of engines] (stages) {Ступени};
    \node[draw, rectangle, below=of stages] (tug) {Разгонный блок};
    \node[draw, rectangle, below=of moving] (safety) {Безопасность};
    \node[draw, rectangle, below=of safety] (emergency) {Система аварийного спасения};
    \node[draw, rectangle, below right=of moving] (connection) {Связь с Землёй};
    \draw[->] (engines) to (stages);
    \draw[->] (engines) to [out=180, in=180] (tug);
    \draw[->] (moving) to (engines);
    \draw[->] (moving) to [out=-90, in=90] (safety);
    \draw[->] (safety) to (emergency);
    \draw[->] (moving) to (connection);
  \end{tikzpicture}

\item{Постройте, оптимальное, на ваш взгляд, по уровням, дерево функций для системы <<Библиотека>>.}

  \emph{Ответ:}
  
  \begin{tikzpicture}
    \node[draw, rectangle] (readersServe) {Обслуживание читателей};
    \node[draw, rectangle, below left=of readersServe, text width=2.5cm] (readerRegister) {Регистрация новых читателей};
    \node[draw, rectangle, below=of readersServe, text width=2.5cm] (bookSearch) {Поиск литературы по запросу читателя};
    \node[draw, rectangle, below=of bookSearch, text width=2.5cm] (selfSearch) {Поиск книги читателем};
    \node[draw, rectangle, below=of selfSearch, text width=2.5cm] (helpSearch) {Помощь читателю в поиске книги};
    \node[draw, rectangle, below right=of readersServe, text width=2.5cm] (bookHome) {Отпуск книг на дом};
    \node[draw, rectangle, below=of bookHome, text width=2.5cm] (bookDate) {Регистрация отпущенной книги};
    \node[draw, rectangle, below=of bookDate, text width=2.5cm] (reminder) {Уведомление об сроке возвращения книги (SMS, e-mail)};
    \node[draw, rectangle, right=of bookHome, text width=2.5cm] (statistics) {Анализ пользовательских интересов};
    \node[draw, rectangle, left=of readersServe] (readingRoom) {Читательский зал};
    \draw[->] (readersServe) to (readerRegister);
    \draw[->] (readersServe) to (bookSearch);
    \draw[->] (bookSearch) to (selfSearch);
    \draw[->] (bookSearch) to [out=180, in=180] (helpSearch);
    \draw[->] (readersServe) to (bookHome);
    \draw[->] (bookHome) to (bookDate);
    \draw[->] (bookHome) to [out=180, in=180] (reminder);
    \draw[->] (readersServe) -| (statistics);
    \draw[->] (readersServe) to (readingRoom);
  \end{tikzpicture}

\item{В чём принципиальная разница между структурно-функциональным подходом (СФП) и функционально-структурным подходом (ФСП)? Опишите особенности каждого подхода.}

  \emph{Ответ:} Структурно-функциональный подход в первую очердь акцентирован на структуру системы и подсистем её образующих. После декомпозиции структуры изучается функциональная сторона системы.

  Функционально-структурный подход напротив уделяет внимание на функции системы, после чего изучается структурная составляющая.

  При структурно-функциональном подходе, из-за изучения структуры, возможна потеря определённой части знаний о системе и о её функциях.

  Функционально-структурный подход бережно относится к функциональной составляющей, позволяет её более полно проанализировать и выстроить структуру.

\item{Выберите произвольную систему и примените к ней ФСП, а затем СФП. Оцените разницу.}

  \emph{Ответ:} Возьмём систему <<Карандаш>>. Система <<Карандаш>> состоит из двух элементов:

  \begin{itemize}
  \item{стержень;}
  \item{оправа.}
  \end{itemize}

  Изучение с точки зрения функционально-структурного подхода даёт нам следующие знания:

  На нулевом уровне стоит функция ``Писать на поверхности'', на нижележащем уровне функции ``удобство удержания карандаша в руке'' и ``контроль длины стержня, выступающего из оправы''.

  Из полученных данных следует структура: стержень окружён оправой.

  Изучение с точки зрения структурно-функционального подхода:

  Структура стержня --- продолговатый маркий предмет в виде цилиндра с малым диаметром. Структура оправы --- продолговатый удобный для хвата предмет в виде цилиндра с отверстием посередине (диаметр отверстия равен диаметру стержня.

  Раз стержень маркий --- значит им можно оставлять следы на поверхностях, но стержень неудобно держать в руке. Но оправа даёт возможность удобно манипулировать стержнем. Вывод: этот предмет может находиться в руке и им можно писать.

  Разница в подходах: в первом случае к выявлению структуры приходили после выявления функций, во втором случае изучалась структура и через структуру возможные функции.

\item{По каким признакам общности классифицируются системы?}

  \emph{Ответ:} Системы можно классифицировать на следующие виды: генетические, идеальные или абстрактные, по взаимодействию со средой, по сложности структуры и поведения, по переменным системы, по степени определённости функционирования и другие.

\item{Какие бывают материальные системы? Приведите пример материальных экологических систем.}

  \emph{Ответ:} Материальные системы бывают следующего вида:

  \begin{itemize}
  \item{естественные системы: физико-химические, биологические, экологические, социальные;}
  \item{искусственные системы: приборы, механизмы, машины и прочее;}
  \item{смешанные системы: биотехнические, эргономические, организационные.}
  \end{itemize}

  Примеры материально-экологических систем: аквариум с рыбками, флора и фауна Австралии.

\item{Опишите открытые и закрытые системы. Существуют ли полностью окрытые/закрытые системы? Если да, приведите примеры.}

  \emph{Ответ:} Под открытыми системами понимают системы, взаимодействующие с окружающей средой веществом, энергией или информацией.

  Под закрытыми системами понимают системы, не взаимодействующие с окружающей средой веществом, энергией или информацией.

  Полностью открытых или закрытых систем не существует.

  Доказательством невозможности существования полностью закрытой системы служит логический эксперимент: пусть существует полностью закрытая система, тогда эта система будет гораздо более чёрной, чем чёрная дыра. Такую систему будет невозможно засечь никакими приборами.

  Аналогично, нет полностью открытой среды --- такая система растворялась бы в окружающей среде без остатка.

\item{Объясните что значат понятия энтропии и негэнтропии для открытых систем.}

  \emph{Ответ:} Энтропия --- это мера беспорядка в системе. Негэнтропия --- это мера упорядоченности системы.

\item{В чем суть теоремы мощности межэлементных связей? Как с её помощью решается задача отделения системы от окружающей среды?}

  \emph{Ответ:} Суть теоремы о мощности межэлементных связей заключается в оценке мощности межэлементных связей с помощью построения эквивалентной поверхности на множестве элементов и связей $\omega = \omega (\alpha, \gamma)$, где $\alpha$ --- элементы системы; $\omega$ --- мощность межэлементных связей $\gamma$.

  При таком построении следующая эквипотенциальная поверхность будет границей системы с окружающей средой.

\item{Что понимают под малыми, сложными, ультрасложными и суперсистемами?}

  \emph{Ответ:} Под сложностью систем принято считать количество элементов, при этом:

  \begin{itemize}
  \item{малые системы --- с числом элементов между 10 и $10^3$;}
  \item{сложные системы --- с числом элементов между $10^4$ и $10^7$;}
  \item{ультрасложные системы --- с числом элементов между $10^7$ и $10^{30}$;}
  \item{суперсистемы --- с числом элементов больше $10^{30}$.}
  \end{itemize}

\item{Приведите примеры слабоформализованных систем.}

  \emph{Ответ:} Слабоформализованными системами считаются системы плохо поддающиеся или не поддающиеся описанию.

  Примером слабоформализованных систем может служить человеческий мозг --- в высшей степени плохо формализуемая и не поддающееся созданию полной модели система.
\end{enumerate}
\end{document}