\documentclass[10pt]{article}
\usepackage{hyphsubst}
\usepackage[T2A]{fontenc}
\usepackage[utf8]{inputenc}
\usepackage[english,ukrainian,russian]{babel}
\usepackage{mathtools}
\usepackage{amsmath}
\newcommand\aug{\fboxsep=-\fboxrule\!\!\!\fbox{\strut}\!\!\!}
\author{Аметов Имиль}
\title{Введение в системный анализ \\ Домашнее задание №2}
\usepackage[thinlines]{easytable}
\usepackage[left=2cm,right=2cm,
top=2cm,bottom=2cm,bindingoffset=0cm]{geometry}

\begin{document}
\maketitle

\begin{enumerate}
\item{Что понимают под целью системы? Как связаны понятия цели и задачи?}

  \emph{Ответ:} Из учебного пособия Н.Н. Горлушкина <<Системный анализ и моделирование информационных процессов и систем>>, страница 16, цитата: ``Причиной и движущей силой любой деятельности является наличие \emph{противоречия} между имеющимся и желаемым состоянием объекта''. Я это вижу следующим образом: существует система, которая находится в некотором состоянии $A_1$, и есть некоторое противоречие (оно может быть экономическим, политическим, физическим, биологическим и прочим) в системе. Существует некоторое положение системы $A_2$ при котором это противоречие было бы устранено. В этом же учебном пособии приводится определение: ``Цель --- идеальное, мысленное предвосхищение результата деятельности и путей его достижения с помощью определённых средств.'' То есть существуют некоторые средства и некоторые способы воздействия этими средствами на систему, таким образом, что система из состояния $A_1$ перейдёт в состояние $A_2$ и противоречие будет устранено. Если же пути и способы устранения противоречия не являются очевидными, то противоречие называется \emph{проблемой}. А такая проблемная ситуация называется \emph{задачей}.

  На примере автореферата Аверченковой Е.Э.: Цитата: ``Цель диссертационного исследования --- разработать методологию управления региональной социально-экономической системой на основе анализа влияния внешней среды.'' То есть, Елена Эдуардовна представляет себе какими средствами и какими способами можно разработать требуемую методологию управления и для достижения этой цели требуется разрешить некоторые промежуточные задачи, с помощью решения которых она достигнет своей цели.
  
\item{Локальные цели системы и подсистемы --- Как они могут быть связаны?}

  \emph{Ответ:} Обычно, для достижения глобальной цели выполняютс следующие действия: а) формулируют локальные цели, стоящие перед элементами системы и подсистемами; б) целенаправленное вмешиваются в работу системы. Таким образом, последовательно приближаются к цели.
  
  Достижение локальных целей подсистемами приводит к достижению локальных целей общей системы.

\item{Постройте декомпозицию целей, задачи подсистем для систем:}

  \begin{enumerate}
  \item{Лифт}
    
    Рассмотрим лифт с противовесом.
    
    В данной постановке вопроса не стоит никакого вопроса о задаче лифта, поскольку не указаны исходные данные и состояние системы и нет данных о конечном состоянии.

    Что касается целей возлагаемых человеком на систему <<Лифт>>, то я вижу следующие цели:

    \begin{enumerate}
    \item{Обеспечить комфортное, безопасное и удобное перемещение грузов и людей;}
      \begin{enumerate}
      \item{Безопасность;}
        \begin{itemize}
        \item{Пожаробезопасность;}
        \item{Контроль скорости перемещения лифта;}
        \item{Контроль перегруза;}
        \item{Управление дверями;}
        \item{Доведение кабины лифта до ближайшего этажа и открытие дверей в случае отключения электроэнергии;}
        \item{Амортизация лифта в случае падения;}
        \end{itemize}
      \item{Взаимодействие с человеком;}
        \begin{itemize}
        \item{Обработка нажатий кнопок на этажах;}
        \item{Обработка нажатий кнопок в кабине лифта;}
        \item{Возможность вызова диспетчера;}
        \end{itemize}
      \item{Подъём и опускание грузов и людей;}
        \begin{itemize}
        \item{Определение текущего состояния лифта и приведение его в требуемое состояние;}
        \end{itemize}
      \end{enumerate}
    \end{enumerate}

  \item{Учебная группа}

    Основная цель --- получение знаний и умений в рамках магистерской (бакалаврской и пр.) программы.

    \begin{enumerate}
    \item{Взаимодействие с преподавателями:}
      \begin{itemize}
      \item{староста --- осуществление централизованного взаимодействия между группой и преподавателями;}
      \item{личное взаимодействие;}
      \end{itemize}
    \item{Обучение:}
      \begin{itemize}
      \item{посещение занятий (лекции, семинарские занятия, лабораторные работы);}
      \item{выполнение домашних и самостоятельных занятий;}
      \item{факультативная деятельность (изучение дисциплин не входящих в стандартные занятия);}
      \item{взаимопомощь в обучении;}
      \end{itemize}
    \item{участие в мероприятиях:}
      \begin{itemize}
      \item{научные конференции, симпозиумы;}
      \item{спортивные;}
      \item{культурные;}
      \end{itemize}
    \end{enumerate}
    
  \item{Дерево}

    Основная задача дерева как представителя флоры --- вырасти из зёрнышка и дать максимально возможное количество потомства.

    Для решения этой задачи участвуют:

    \begin{itemize}
    \item{Фиксация на почве:}
      \begin{itemize}
      \item{корень;}
      \end{itemize}
    \item{Питание:}
      \begin{itemize}
      \item{корень --- всасывает из почвы полезные вещества;}
      \item{крона --- синтез новых полезных веществ с помощью фотосинтеза;}
      \end{itemize}
    \item{Опора для кроны:}
      \begin{itemize}
      \item{ствол дерева;}
      \end{itemize}
    \item{Транспортировка и хранение полезных веществ;}
      \begin{itemize}
        \item{ствол дерева;}
      \end{itemize}
    \end{itemize}

  \item{Диссертация}

    Основная цель диссертации --- рассмотреть какую-либо проблему и найти пути её решения или доказать, что решения не судествует.

    Основную цель можно разбить на несколько подцелей:

    \begin{itemize}
    \item{предоставление информации об учебном заведении, ФИО автора, название работы;}
      \begin{itemize}
      \item{титульный лист;}
      \end{itemize}
    \item{ориентация по материалу диссертации;}
      \begin{itemize}
        \item{оглавление;}
      \end{itemize}
    \item{сведения о научной работе;}
      \begin{itemize}
      \item{введение;}
      \item{основная часть;}
      \item{заключение;}
      \item{приложения;}
      \end{itemize}
    \item{сведения о научной новизне и значимости работы;}
      \begin{itemize}
      \item{введение;}
      \end{itemize}
    \item{дать сведения о применённых методах и полученных результатах;}
      \begin{itemize}
      \item{основная часть;}
      \item{заключение;}
      \item{приложения;}
      \end{itemize}
    \item{сделать заключение о полученных результатах и перспективах дальнейшего развития темы;}
      \begin{itemize}
      \item{заключение;}
      \item{приложения;}
      \end{itemize}
    \item{сведения об источниках данных;}
      \begin{itemize}
        \item{библиографический список;}
      \end{itemize}
    \end{itemize}
  \end{enumerate}

\item{Что понимают под организацией системы? Как соотносятся понятия упорядоченности и организованности системы? Как оценивается степень организованности системы?}

  Под организацией системы понимают такую структуру системы, которая позволяет системе с наименьшими затратами достигать целей.

  Организованность --- это характеристика системы, представляющая более высокую ступень упорядоченности системы. В организованной системе заложены способность снижения внутренней энтропии (беспорядка) и устойчивости к внешним возмущениям, выводящим из стабильного состояния.

  Уровень организованности системы определяется по формуле

  $$R = 1 - \dfrac {\text {Э}_{\text{реал}}} {\text {Э}_{\text{макс}}},$$

  где $\text{Э}_{\text{реал}}$ --- реальное или текущее значение энтропии (неопределённости) системы, $\text{Э}_{\text{макс}}$ --- максимально возможная энтропия или неопределённость по структуре и функции системы.

\item{Как повысить организованность системы ``Учебная группа''?}

  \emph{Ответ:} Для повышения организованности системы ``Учебная группа'' нужна, первым делом, идеология. По моим наблюдениям достаточно большое количество парней идут в вузы, в целом, и в магистратуру, в частности, только для того чтобы избежать призыва в армию. Но это слабая мотивация для вовлечения молодёжи в учёбу и вузовскую жизнь.

  С развалом Советского Союза в Российской Федерации, как и во всех остальных пост-Советских странах, нет никакой работы с молодёжью. Нет никакой идеологии, но природа не терпит пустоты и в эту пустоту лезут различные представители как религиозных сект, так и экстремистских организаций предлагающие свою идеологию. Примером подобных экстремистских организаций является ``АУЕ''\footnote{``Арестантский Уклад Един'', экстремистская организация, запрещённая в России}, романтизирующая и пропагандирующая воровство и криминал. 17 августа 2020 года эта организация была признана экстремистской организацией. Причастность к этой организации теперь трактуется как экстремизм и карается по Уголовному кодексу.

  Вторая мотивация --- это, конечно, экономическая. Для подавляющего большинства молодёжи наука и образование больше не являются привлекательной сферой деятельности в связи с малыми заработками в этой сфере. Ну а если на заработки с научной деятельности не прожить, то зачем стараться?

\item{Какие этапы характеризуют существование системы? Что понимают под развитием системы?}

  \emph{Ответ:} В существовании любых систем присутствуют три основных этапа: развитие, зрелость, деградация.

  Под развитием понимают увеличение порядка, рост организованности, увеличение информации, снижение энтропии системы.

\item{Охарактеризуйте стадии развития системы ``Программа для ЭВМ''.}

  \emph{Ответ:} Steve Oualline в книге ``Practical C Programming, 3rd Edition'' приводит следующие этапы в развитии программ:

  \begin{itemize}
  \item{\emph{Требования}. На этом этапе в очень общих словах описывается что требуется от программы.}
  \item{\emph{Техническое задание}. В документе с техническим заданием описываются некоторые детали программы: язык программирования, фреймворки, системные требования, условия функционирования программы и так далее.}
  \item{\emph{Архитектура программы}. На этом этапе осуществляется выбор технологий программирования, создаются основные функции, процедуры и алгоритмы программы.}
  \item{\emph{Процесс написания программы}. Воплощаются в виде текста программы все выбранные алгоритмы и объединяются воедино все функции и процедуры, создаётся интерфейс пользователя.}
  \item{\emph{Тестирование}. Написанная программа подвергается процедурам тестирования. Здесь проверяются работа как отдельных узлов, функций и процедур программы, так и работа всей программы в целом. При обнаружении каких-либо недоработок и неправильной работы программы переходят к этапу отладки, если никаких проблем в тестировании не было обнаружено, то выполняется инспекция текста программы.}
  \item{\emph{Отладка}. Выявляются причины не правильной работы программы. После обнаружения причин некорректной работы переходят к этапу ``Процесс написания программы.''}
  \item{\emph{Инспекция текста программы}. Выполняется анализ текстов программы, осуществляется документирование исходного кода, пишутся вспомогательные материалы к программе. При обнаружении каких-либо проблем переходят на этап ``Техническое задание'', где обнаруженные проблемы заносятся в техническое задание. Если же нет никакой критики и ничего исправлять не надо, то переходят к следующему этапу.}
  \item{\emph{Выпуск программы}. Формируется готовая исполняемая программа, осуществляется упаковка программы для развёртывания на машинах пользователей и полученные программные пакеты передаются конечным пользователям.}
  \item{\emph{Сопровождение}. На этапе сопровождения осуществляется обучение конечных пользователей работе с программой, при обнаружении некорректного поведения программы или проблем с программой переходят на этап отладки.}
  \item{\emph{Ревизия}. Собирается информация от конечных пользователей, их предложения и замечания. На основе собранной информации формируют дополнительные требования к программе и переходят к этапу ``Техническое задание''.}
  \item{\emph{Смерть программы}. В силу тех или иных обстоятельств (смена операционной системы, исчезновение программной ниши, возникновения новых технологий и прочее) наступает момент когда существующая программа перестаёт удовлетворять предъявляемым требованиям или попросту становится не нужна. В этом случае либо программой перестают пользоваться, либо создают новые технические задания.}
  \end{itemize}

\item{Каковы критерии развития любой системы?}

  \emph{Ответ:} Критериями развития системы являются: увеличение порядка, рост организованности, увеличение информации, снижение энтропии системы.

\item{Сформируйте группы критериев развития системы ``Человек --- Компьютер''.}

  \emph{Ответ:} Под ``Компьютером'' я понимаю не столько компьютер как вещь, сколько компьютер плюс программное обеспечение. В этом случае, происходит развитие как аппаратной части, так и программной составляющей. Опять же система ``Человек --- Компьютер'' создаётся не просто как вещь в себе, не самоцель, а предназначена для решения каких-то проблем, или хотя бы для упрощения труда человека.

  \begin{itemize}
  \item{\emph{Увеличение порядка.} Проектировщики системы должны рассмотреть предметную область для которой они разрабатывают систему. Должны изучить как взаимодействуют между собой те или иные элементы этой области и если эта область уже обладает хорошим порядком, то проектировщики должны внедрением ЭВМ придать большую строгость, если же порядка в этой предметной области мало, следует изучить эту предметную область и вывести основные связи, которые в дальнейшем помогут привнести хорошую структуру в эту область. Возможно, даже ценой реструктуризации этой области (правда не всегда возможна лёгкая реструктуризация).}

  \item{\emph{Рост организованности:} Если после внедрения информационной системы в некоторую предметную область упрощается работа в этой предметной области, уменьшаются простои, затраты, становится проще контролировать эту предметную область, то можно заключить, что внедрение информационной системы пошло на пользу и организованность растёт.}

  \item{\emph{Увеличение информации:} В данном случае критерием может служить накопление точной информациии: как происходят явления в предметной области, как они протекают, каковы качественные и количественные изменения.}

    \item{\emph{Снижение энтропии системы:} Если после внедрения информационной системы у пользователей упрощается получение информации о состоянии предметной области, устраняются белые пятна об этой области, то можно заключить что энтропия системы снижается, если же пользователь по прежнему не ведает что происходит в системе, не знает о её текущих характеристиках, то делаем вывод, что энтропия не снизилась, а, возможно, даже повысилась.}
    \end{itemize}

  \item{Что понимают под адаптацией системы?}

    Под адаптацией понимают способность системы к обработке и использованию полезных для своего развития сигналов, которые поступают извне.

  \item{Опишите в общем виде (уровень наименований и взаимодействий потоков, подсистем и$\backslash$или функций) взаимодействие адаптивной системы ``Газета'' с внешней средой.}

    \emph{Ответ:} Каждое средство массовой информации можно разбить на следующие условные подсистемы:

    \begin{itemize}
    \item{редакция (редакторы, журналисты, творческие и технические работники),}
    \item{рекламная служба,}
    \item{служба распространения (печатных номеров или эфирного сигнала)}
    \item{служба маркетинга,}
    \item{финансовая служба,}
    \item{юридическая служба,}
    \item{администрация}
    \end{itemize}

    Полезными сигналами для печатного издания являются различные новости, которые происходят в стране и мире.

    Обработкой этих сигналов занимаются элементы ``журналисты''. Сигналы о новостях приходят к журналистам по разным каналам: из новостных агенств, от знакомых журналиста, из писем в редакцию, из других СМИ. В некоторых случаях журналисты наблюдают за событиями, находясь непосредственно в месте происшествий.

    Полученный сигнал журналисты перерабатывают, уточняют, получают дополнительную информацию и на основе полученных сведений пишут статью.

    Дальше вступают в дело корректоры и редакторы. Они выполняют исправление статьи, редактируют её подгоняя под требуемые параметры: объём слов, место расположения статьи в номере, важность и актуальность статьи.

    Фотографы и дизайнеры подбирают оформление статьи, находят фотоиллюстрации, картинки, подбирают шрифты и цвета.

    Главный редактор решает пустить эту статью в номер или отложить, перередактировать её или оставить как есть. На каком месте номера разместить статью и прочее.

    Ещё одним немаловажным сигналом являются рекламодатели.

    Рекламодатели связываются с отделом рекламы посредством писем, звонков или непосредственно с помощью живого общения.

    Менеджер по рекламе после получения сигнала о желании размещения рекламы обговаривает условия размещения: место размещения, количество слов, дизайн рекламного текста, иллюстрации и прочее.

    Также менеджер по рекламе связывается с финансовыми и юридическими службами на предмет допустимости рекламы и стоимости публикации. В случае соблюдения всех условий проект рекламы передаётся в редакцию для вёрстки и проходит одобрение главного редактора.

    Служба маркетинга изучает сигналы от потребителей и анализирует их. В процессе анализа служба маркетинга выбирает целевую аудиторию, места продажи изданий, продвижение изданий через другие СМИ, объявления и прочее.

  \item{Что понимают под гомеостазом системы? В чем суть гомеостатического подхода к изучению систем?}

    \emph{Ответ:} Гомеостаз --- это функциональное состояние системы, при котором обеспечивается поддержание динамического постоянства в допустимых пределах жизненно важных функций и параметров системы при различных изменениях внутренней и внешней среды.

    Суть гомеостатического подхода заключается в поиске критически важных для выживания системы параметров и создания механизма поддержки этих параметров в нормальных пределах.

  \item{Определите сбалансированные противовесы, обеспечивающие устойчивость системы ``Хороший учебник''.}

    \emph{Ответ:} Понятие ``хороший учебник'' --- крайне размытое. Например, учебник Зорича Владимира Антоновича по математическому анализу рекомендуется слушаетлям лекций по математическому анализу в Независимом Московском Университете, но эта книга не подходит для роли первой книги по математическому анализу по той причине, что в этой книге крайне высокий уровень абстракций для среднестатистического первокурсника. Более подходящим курсом для первоначального знакомства с математическим анализом я считаю книги Кудрявцева Льва Дмитриевича. В его книгах уровень абстракций не так высок, вместе с тем даётся хорошая теория и задачи. Если же человек хочет читать ``Математический анализ'' Зорича, то ему нужно будет дополнительно обращаться ещё и к другой литературе.

    Я считаю, что учебники пишутся исходя из той целевой аудитории для которой они предназначены. Ясно, что если перед нами стоит задача бороться с безграмотностью, как это было в СССР в 1920-1930 годах, то, скорее всего, не стоит рассказывать человеку не умеющему ни писать, ни читать о теории групп и понятии производной, а лучше сосредоточиться на обучении его письму и простому счёту.

    Что касается сбалансированных противовесов для системы ``Хороший учебник'': Суворову приписывают слова ``Теория без практики --- мертва, практика без теории --- слепа''. Я считаю, что в системе ``Хороший учебник'' должна быть с одной стороны подача теории, приведение примеров по этой теории, с другой стороны --- должно быть большое число задач различного уровня сложности для отработки теории и изучения её на практике.

  \item{Что понимают под управлением системой? Что происходит с системой под действием управляющего воздействия?}

    \emph{Ответ:} Управление объектом (системой) --- это воздействие на него с целью достичь желаемых свойств его поведения, в частности, гомеостаза.

    При управлении системой (объектом) происходит перевод (переход) системы из одного состояния в другое, то есть управляемый объект под воздействием управляющего изменяет своё поведение так, чтобы достичь заданной цели и при помощи обратной связи выдаёт ответную реакцию о своём состоянии или поведении.

  \item{Перечислите основные компоненты системы управления рабочим коллективом.}

    \emph{Ответ:} Целевой функцией системы управления рабочим коллективом является решение задач делегированных этому рабочему коллективу.

    Управляющей системой (субъектом управления) в данном случае является начальник рабочего коллектива.

    Для плодотворного управления системой ``Рабочий коллектив'' начальник должен обладать следующими знаниями: производственные возможности коллектива, задачи, с которыми может справиться коллектив, возможности и особенности каждого элемента рабочего коллектива --- работника коллектива. В обязанности начальника рабочего коллектива входит распределение объёма работы между участниками рабочего коллектива, с учётом особенностей коллектива. Кроме того, начальник также является участником рабочего коллектива, а значит на него в значительной мере падает нагрузка по поддержанию гомеостаза рабочего коллектива: здесь, создание условий для работы коллектива, устранение причин приводящих к разладу системы ``Рабочий коллектив'' и улучшение условий функционирования ``Рабочего коллектива''. При этом система ``Рабочий коллектив'' имеет возможность влиять на начальника, отчитываясь о проделанной работе, о возникших различных ситуациях в процессе работы.

  \item{Приведите примеры управления как воздействия, как взаимодействия через прямую и обратную связи.}

    \emph{Ответ:} Примером воздействия может служить маяк --- маяк светит в ночи давая ориентир для кораблей и судов, при этом сам маяк не получает никакой информации от судов.

    Пример взаимодействия через прямую и обратную связь: пилот самолёта, здесь пилот через органы управления влияет на самолёт и переводит его в различные состояния, в то же время самолёт посредством различных датчиков (высоты, скорости, температуры окружающего воздуха, температуры обшивки, уровня топлива, положения в пространстве, углов атаки и так далее) влияет на пилота заставляя его предпринимать те или иные действия: снизить или повысить высоту, скорость, спуститься на дозаправку и прочее.    
\end{enumerate}
\end{document}