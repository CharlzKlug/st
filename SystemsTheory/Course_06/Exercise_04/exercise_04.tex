\documentclass[10pt]{article}
\usepackage{hyphsubst}
\usepackage[T2A]{fontenc}
\usepackage[utf8]{inputenc}
\usepackage[english,ukrainian,russian]{babel}
\usepackage{mathtools}
\usepackage{amsmath}
\newcommand\aug{\fboxsep=-\fboxrule\!\!\!\fbox{\strut}\!\!\!}
\author{Аметов Имиль}
\title{Введение в системный анализ \\ Домашнее задание №4}
\usepackage[thinlines]{easytable}
\usepackage[left=2cm,right=2cm,top=2cm,bottom=2cm,bindingoffset=0cm]{geometry}
\usepackage{tikz}
\usetikzlibrary{graphs}
\usetikzlibrary{positioning}
\setlength{\parindent}{5ex}
\setlength{\parskip}{1em}
\begin{document}
\maketitle

\begin{enumerate}
\item{Какие два свойства описывают системно-интегративный аспект?}

  \emph{Ответ:} Системно-интегративный аспект описывают следующие два свойства:

  \begin{enumerate}
  \item{Свойства системы не являются суммой свойств элементов или частей.}
  \item{Свойства системы зависят от свойств элементов, частей (изменение в одной части вызывает изменение во всех остальных частях и во всей системе).}
  \end{enumerate}

\item{Даны две системы: <<Футбольная команда>> и <<Хоккейная команда>>. В какой из этих систем сильнее выражен системно-интегративный аспект? Ответ обоснуйте.}

  \emph{Ответ: } На футбольном поле со стороны одной команды принимают участие 11 игроков, в хоккейной команде со стороны одной команды на поле должно быть не более 6 игроков.

  Обе системы удовлетворяют требованию первого свойства: свойство системы не является суммой свойств элементов или частей системы.

  Но второе свойство (свойства системы зависят от свойств элементов) сильнее выражено в системе <<Хоккейная команда>>. Если удалить из этой системы одного игрока, то характер игры меняется гораздо выраженнее, чем в случае удаления одного игрока из системы <<Футбольная команда>>, а значит, система сильнее зависит от свойств своих элементов.

\item{Что понимается под принципом связности?}

  \emph{Ответ: } Под принципом связности или под системно-компонентным аспектом понимают рассмотрение любой части системы совместно с её связями с окружением.

\item{В чём заключается принцип коммуникативности?}

  \emph{Ответ:} Принцип коммуникативности, он же системно-коммуникационный аспект выражается в выработке и/или утере системой некоторых её свойств в ответ на требования, предъявляемые внешней средой.

\item{Приведите пример, подчёркивающий принцип коммуникативности.}

  \emph{Ответ:} Примером принципа коммуникативности могут служить киты. На данный момент считается, что предками китов являются пакицеты --- хищные млекопитающие из семейства Pakicetidae, относящиеся к примитивным китообразным. Пакицеты вели земноводный образ жизни и внешне представляли собой тонконогого зверя с крохотными копытцами на пальцах. После выхода на сушу некоторые пакицеты вернулись в воду, полностью перейдя на водный образ жизни и дав начало возникновению китообразных. Постепенно киты избавились от лап и копыт. Здесь соблюдается принцип коммуникативности~---~в водной среде лапы и копыта не представляли никаких преимуществ, поэтому атрофировались.

\item{Что понимают под системно-историческим аспектом?}

  \emph{Ответ:} Под системно-историческим аспектом понимают изучение ретроспективы и перспективы развития систем.

\item{На что обращают внимание при изучении систем в системно-историческом аспекте?}

  \emph{Ответ:} При изучении систем в системно-историческим аспекте нужно обращать внимание на следующие элементы: учёт изменяемости системы, её способности к развитию, расширению, замене частей, накапливанию информации.

\item{Пользуясь принципом историчности укажите возможное будущее медицины.}

  \emph{Ответ:} Я считаю, что в будущем медицина будет развиваться в сторону генетической инженерии. Уже в прошлом году 25 ноября 2018 года профессор Хэ Цзянькуй объявил о рождении генетически модифицированных детей, устойчивых к вирусу ВИЧ\footnote{https://www.bbc.com/russian/features-46347487

    Китайский профессор объявил о рождении генетически модифицированных детей. Ученые в гневе. Дата публикации: 26 ноября 2018. Дата посещения: 20 октября 2020.}.

  В наши дни возможно проведение скрининга плода человека на предмет наличия у плода предрасположенностей к различным генетическим и наследственным заболеваниям. Я думаю, что рано или поздно наступет день, когда будет возможным уже на этапе планирования изменять генетический код зародыша с целью уменьшить подверженность к различным болезням.

  На данный момент самой главной преградой является морально-этическая сторона проведений экспериментов на зародышах человека. Но мораль человека гибка и подвержена изменениям. Как знать, быть может лет через 15-20 человечество созреет до генетической модификации не только овощей, насекомых и животных, но и самого человека?

\item{Постройте иерархическую структуру несущих конструкций одноэтажного здания.}
  
  \emph{Ответ:} Иерархическую структуру несущих конструкций одноэтажного здания можно построить следующим образом:
  \begin{enumerate}
  \item{фундамент --- основа здания;}
  \item{стены --- опорные конструкции, передающие на фундамент тяжесть пролётных конструкций, оборудования и архитектурных деталей;}
  \item{пролётные конструкции --- служат перекрытиями для формирования потолка и основой для крыши.}
  \end{enumerate}


\item{Какие три описательных уровня обычно применяют для исследования системы?}

  \emph{Ответ:} Обычно выделяют следующие три обобщённых описательных уровня для исследования систем:

  \begin{enumerate}
  \item{описание с точки зрения присущих системе внешних, целостных свойств;}
  \item{описание с точки зрения внутреннего строения и участия компонентов в формировании целостных свойств системы;}
  \item{описание системы как подсистемы более широкой системы.}
  \end{enumerate}

\item{Перечислите дополнительные принципы системного подхода.}

  \emph{Ответ:} Возможные дополнительные принципы изучения систем:

  \begin{itemize}
  \item{Принцип конечной цели --- подчинённость системы некоторой глобальной цели;}
  \item{Принцип модульного построения --- выделение модулей в системе и изучение системы как совокупности модулей;}
  \item{Принцип функциональности --- рассмотрение системы как совокупности структуры и функции, большее внимание отдаются функциональной составляющей;}
  \item{Принцип децентрализации --- сведение к минимуму централизованного управления системой и повышение доли самостоятельных управленческих решений, принимаемыми компонентами системы;}
  \item{Принцип неопределённости --- изучение системы с точки зрения отказоустойчивости с применением элементов теории вероятностей и математической статистики;}
  \item{Принцип чувствительности --- управление системой с учётом реакции системы на управляющие сигналы;}
  \item{Принцип свёртки --- обобщение и укрупнение информации и управляющих воздействий при движении снизу-вверх по иерархическим уровням.}
  \end{itemize}
\end{enumerate}
\end{document}