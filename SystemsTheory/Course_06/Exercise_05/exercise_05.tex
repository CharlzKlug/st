\documentclass[10pt]{article}
\usepackage{hyphsubst}
\usepackage[T2A]{fontenc}
\usepackage[utf8]{inputenc}
\usepackage[english,ukrainian,russian]{babel}
\usepackage{mathtools}
\usepackage{amsmath}
\newcommand\aug{\fboxsep=-\fboxrule\!\!\!\fbox{\strut}\!\!\!}
\author{Аметов Имиль}
\title{Введение в системный анализ \\ Домашнее задание №5}
\usepackage[thinlines]{easytable}
\usepackage[left=2cm,right=2cm,top=2cm,bottom=2cm,bindingoffset=0cm]{geometry}
\usepackage{tikz}
\usetikzlibrary{graphs}
\usetikzlibrary{positioning}
\setlength{\parindent}{5ex}
\setlength{\parskip}{1em}
\begin{document}
\maketitle

\begin{enumerate}
\item{Назовите несколько качественных методов описания систем.}

  \emph{Ответ:} Я думаю, что в роли качественных методов описания системы можно использовать следующие способы описания:

  \begin{itemize}
  \item{описывать систему как <<чёрный ящик>>, то есть, описывать поведение системы в ответ на реакцию изменения качеств внешней среды;}
  \item{описывать структуру системы такими языками как UML (Unified Modeling Language --- Унифицированный Язык Моделирования);}
  \item{описывать систему с помощью графа её возможных состояний;}
  \end{itemize}

\item{Назовите несколько количественных методов описания систем.}

  \emph{Ответ:} В качестве количественных методов описания системы можно выбрать следующие:

  \begin{itemize}
  \item{как соотношение меры потребления системой или её компонентами некоторых ресурсов (информационных, энергетических, материальных) и результатом работы системы (если этот результат возможно как-то оценить);}
  \item{методами имитационного моделирования (например, написав программу, некоторым образом моделирующую систему);}
  \item{оценивать устойчивость системы с помощью вероятностно-статистических методов математики;}
  \end{itemize}

\item{Чем отличаются качественные и количественные методы описания систем?}

  \emph{Ответ:} В качественных методах описание носит большей частью идейный, понятийный аппарат характеризующий, возможно несколько субъективно, протекающие процессы в системе.

  В количественных методах описание больше формализовано на языке математики и описывает происходящие процессы с точки зрения математического аппарата. В количественных методах значительная часть уделяется способам измерений характеристик системы и их соотношениям.

\item{Приведите пример задачи, в которой возможно использование только качественных методов для описания системы.}

  \emph{Ответ:} Пусть стоит задача снять фильм, который хорошо встретили бы кинозрители. Причём под кинозрителями я понимаю не прослойку в возрасте 6 --- 19 лет, а постарше. Для того, чтобы угодить зрителям условно школьного возраста рецепт прост: достаточно снять ещё один фильм-киножевачку из категории супергеройского и густо приравить его спецэффектами.

  Задача угодить возрастным и думающим зрителям является гораздо более сложной. Вдумчивый зритель наблюдает за героями, наблюдает за историей, разворачиваемой на экране, следит за логичностью повествования, здесь недопустимы <<рояли в кустах>>. Каждый год Голливуд выпускает около 120 фильмов. Из этих 120 фильмов 1-2 фильма --- хорошие. Не отличные, а просто хорошие.

  А каков рецепт хорошего фильма? Его не существует. Один и тот же режиссёр на начале карьеры может снять за сущие копейки шедевр, но как только на него обратят внимание и дадут миллионы условных долларов, как он выпускает дорогостоящий шлак. В фильме могут быть задействованы именитые звёзды, дорогостоящие сценаристы, гримёры, костюмеры и операторы. А на выходе мы получим ещё одно проходное кино.

  Подобным образом обстоят дела и в индустрии видеоигр. Крупнейшие издатели видеоигр могут тратить миллионы на производство игр, применять дорогостоящие технологии, использовать труд сотен специалистов, а игра проваливается. Пример: игра <<Anthem>>. Несмотря на то, что эту игру разрабатывала опытная игровая студия BioWare, а издателем выступала крупная компания Electronic Arts игра оказалась провалом. И в то же время маленькая и никому не известная группа ZA/UM состоящая из примерно 35 разработчиков создала игру <<Disco Elysium>>, которая моментально стала известной, получила кучу восторженных откликов как от рядовых игроков, так и от профессиональных критиков. Другой пример: игра <<Papers, Please>> от одного единственного разработчика Lucas Pope. Сразу же после выхода стала известной, также получила множество положительных откликов от игроков и наград от игровой прессы.

\item{Приведите пример задачи, в которой возможно использование только количественных методов для описания системы.}

  \emph{Ответ:} Примером может служить тренировка искусственной нейронной сети на основе персептронов для решения задачи классификации. Персептрон --- это математическая модель биологического нейрона и служит строительным блоком для построения многослойных искусственных нейронных сетей. У персептрона есть несколько входов, на которые поступют некоторые значения. Полученные значения умножаются на соответствующие весовые коэффициенты и складываются и если сумма достигает некоторого порогового значения, то на выходе выдаётся единица, при недостижении порогового значения на выход подаётся ноль.

  Такую нейронную сеть составленную из персептронов можно назвать системой. Результат совместной работы всех нейронов сети позволяет добиться таких результатов, которые недостижимы для отдельно взятых нейронов. При этом свойства нейросети не является суммой свойств нейронов.

\item{Какие предположения (постулаты) о характере функционирования систем отражают взаимодействие системы с внешней средой?}

  \emph{Ответ:} К предположениям (постулатам) о характере функционирования систем отражающим взаимодействие системы с внешней средой относят:

  \begin{itemize}
  \item{на вход системы могут поступать входные сигналы;}
  \item{система способна выдавать выходные сигналы;}
  \end{itemize}
\end{enumerate}
\end{document}