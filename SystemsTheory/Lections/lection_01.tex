\chapter{Лекция №1}

\section{Лекция №1. Задача о ранце.}

\subsection{Не делимые предметы}

Предметы нельзя разделять. Тогда решение задачи о ранце приобретает вид:

$$
Q(i + 1, P) = \left\{
\begin{aligned}
  Q(i, P),\text{если $\upsilon_{i+1}$ (вес предмета) $>P$} \\
  \max \{Q(i,P); C_{i+1}+Q(i, P- \upsilon_{i+1})\}
\end{aligned}
\right.
$$

где $Q(i + 1, P)$ --- решение для $i+1$-го предметов с максимальной грузоподъёмностью рюкзака в $P$ килограмм, $C_{i+1}$ --- стоимость $i+1$-го предмета.

Решение для ценовой стоимости $5 x_1 + 3 x_2 + 6 x_3 + 2 x_4 + 4 x_5 \to max$, массы $3 x_1 + 2 x_2 + 4 x_3 + x_4 + 2 x_5 \le 9$, $x_i \in \{0,1\}$, $i = 1, 2, ..., 5$.

\begin{tabular}{|c|c|c|c|c|c|c|c|c|c|}\hline
  \diagbox{$i$}{P} & 1 & 2 & 3 & 4 & 5 & 6 & 7 & 8 & 9 \\ \hline
  1 & 0 & 0 & $5_1$ & $5_1$ & $5_1$ & $5_1$ & $5_1$ & $5_1$ & $5_1$ \\ \hline
  2 & 0 & $3_2$ & $5_1$ & $5_1$ & $8_{1,2}$ & $8_{1,2}$ & $8_{1,2}$ & $8_{1,2}$ & $8_{1,2}$ \\ \hline
  3 & 0 & $3_2$ & $5_1$ & $6_{3}1$ & $8_{2,5}$ & $9_{3,2}$ & $11_{3,1}$ & $11_{3,1}$ & $14_{3,1,2}$ \\ \hline
  4 & $2_4$ & $3_2$ & $5_{1,2}$ & $7_{4,1}1$ & $8_{4,3}$ & $9_{3,2}$ & $11_{4,3,2}$ & $13_{4,3,1}$ & $14_{3,1,2}$ \\ \hline
  5 & $2_4$ & $4_5$ & $6_{5,4}$ & $7_{4,1}1$ & $9_{5,1}$ & $11_{5,4,1}$ & $12_{5,4,3}$ & $12$ & $15_{5,4,3,2}$ \\ \hline
\end{tabular}

Ответ: максимальную по цене нагрузку ранца можно получить из продуктов за номерами 5, 4, 3, 2.

\subsection{Задача о ранце с дробимыми предметами}

В этом случае алгоритм решения задачи принимает такой вид:

\begin{enumerate}
\item{Для каждого предмета находим цену за единицу его массы (удельная стоимость) $\mu_i = \frac {C_i} {\upsilon_i}$}

\item{Упорядочиваем предметы по убыванию цены (по удельной стоимости).}

\item{В планируемом порядке предметы последовательно помещаются в ранец до тех пор, пока вес ранце это позволяет. Последний предмет помещается лишь частично.}
\end{enumerate}

\subsection{Алгоритм Данцига}

Задача заключается в том, чтобы разместить в ранец дробимые и не дробимые предметы с упором на максимальную стоимость.

В начале по старинке заполняются не дробимые предметы, а оставшееся место заполняется не дробимыми предметами.

Задача:

$$
\left \{
\begin{aligned}
5 x_1 + x_2 + 3 x_3 + 2 x_6 + 3 x_7 \to \max \\
x_1 + 2 x_2 + x_3 + 4 x_6 + 3 x_7 \le 9 \\
x_1, x_2, x_3 \in {0, 1} \\
x_6, x_7 \in [0; 1]
\end{aligned}
\right.
$$

Найдём удельную стоимость для $x_6, x_7$. Она будет такой:

$$\mu_6=\frac {2} {4} = \frac {1} {2}, \mu_7 = 1.$$

Видим, что 7-й продукт более дорог чем 6-й и начинать надо именно с 7-го.

Строим таблицу:

{\tiny
\begin{tabular}{|c|c|c|c|c|c|c|c|c|c|c|} \hline
  \diagbox {$i$} {$P$} & 0 & 1 & 2 & 3 & 4 & 5 & 6 & 7 & 8 & 9 \\ \hline
  1 & 0 & $5_1$ &$5_1$ &$5_1$ &$5_1$ &$5_1$ &$5_1$ &$5_1$ &$5_1$ &$5_1$ \\ \hline
  2 & 0 & $5_1$ &$5_1$ &$6_{1,2}$ &$6_{1,2}$ &$6_{1,2}$ &$6_{1,2}$ &$6_{1,2}$ &$6_{1,2}$ &$6_{1,2}$ \\ \hline
  3 & 0 & $5_1$ &$8_{1,3}$ &$8_{1,3}$ &$9_{1,2,3}$ &$9_{1,2,3}$ &$9_{1,2,3}$ &$9_{1,2,3}$ &$9_{1,2,3}$ &$9_{1,2,3}$ \\ \hline
  * & 9 & 8 &7 &7 &5 &5 &5 &5 &5 &5 \\ \hline
  ** & $5_{6(1),7(1)}$ & $5_{6(1),7(1)}$ &$5_{6(1),7(1)}$ &$5_{6(1),7(1)}$ &$4_{7(1), 6 (\frac {1} {2})}$ &$4_{7(1), 6 (\frac {1} {2})}$ &$4_{7(1), 6 (\frac {1} {2})}$ &$4_{7(1), 6 (\frac {1} {2})}$ &$4_{7(1), 6 (\frac {1} {2})}$ &$4_{7(1), 6 (\frac {1} {2})}$ \\ \hline
  
\end{tabular}
}
