\documentclass[10pt]{article}
\usepackage[T2A]{fontenc}
\usepackage[utf8]{inputenc}
\usepackage[english,russian]{babel}
\usepackage{amsmath}

\author{Аметов Имиль, гр. М07-903}
\title{Расшифрование RSA по китайской теореме об остатках}

\begin{document}
\maketitle

\emph{Задача:}

Расшифровать сообщение $c = 251$ для $p = 67$, $q = 31$ и $d = 463$ зашифрованное с помощью RSA. При расшифровании использовать китайскую теорему об остатках.

\emph{Решение:}

Находим $n = p \cdot q = 67 \cdot 31 = 2077$.

Находим $m_p = c^{d \mod (p-1)} \mod p = 251^{463 \mod 66} \mod 67 = 251^1 \mod 67 = 50$.

Находим $m_q = c^{d \mod (q-1)} \mod q = 251^{463 \mod 30} \mod 31 = 251^{13} \mod 31 = 24$.

Вычисляем открытое сообщение $m$ по формуле:

$$m = m_p \cdot q \cdot (q^{-1} \mod p) + m_q \cdot p \cdot (p^{-1} \mod q) \mod n.$$

Получаем:

$$m = 50 \cdot 31 \cdot (31^{-1} \mod 67) + 24 \cdot 67 \cdot (67^{-1} \mod 31) \mod 2077 = $$
$$= 1550 \cdot (13 \mod 67) + 1608 \cdot (25 \mod 31) \mod 2077 = $$
$$= 1550 \cdot 13 + 1608 \cdot 25 \mod 2077 =$$
$$= 20150 + 40200 \mod 2077 = 60350 \mod 2077 = 117.$$

\emph{Ответ:} $m = 117$.
\end{document}