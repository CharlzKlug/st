\documentclass[10pt]{article}
\usepackage[T2A]{fontenc}
\usepackage[utf8]{inputenc}
\usepackage[english,russian]{babel}
\usepackage{amsmath}
\author{Аметов Имиль, гр. М07-903}
\title{RC4}

\begin{document}
\maketitle
Задача: Приведите пример работы потокового шифра RC4 для массива размером 16 ячеек и чисел $\{ 0 ... 15 \}$.

Решение: Заполняем значения для ячеек состояния $S_0$, $S_1$, ..., $S_{15}$:

\begin{tabular}{l|l|l|l|l|l|l|l}
  $S_0$ & $S_1$ & $S_2$ & $S_3$ & $S_4$ & $S_5$ & $S_6$ & $S_7$ \\ \hline
  0 & 1 & 2 & 3 & 4 & 5 & 6 & 7 \\ \hline
  $S_8$ & $S_9$ & $S_{10}$ & $S_{11}$ & $S_{12}$ & $S_{13}$ & $S_{14}$ & $S_{15}$ \\ \hline
  8 & 9 & 10 & 11 & 12 & 13 & 14 & 15 \\ \hline
\end{tabular}

Предоставленный по условию ключ выглядит так:

\begin{tabular}{l|l|l|l|l|l|l|l}
  $K_0$ & $K_1$ & $K_2$ & $K_3$ & $K_4$ & $K_5$ & $K_6$ & $K_7$ \\ \hline
  0 & 1 & 2 & 3 & 4 & 5 & 6 & 7 \\ \hline
  $K_8$ & $K_9$ & $K_{10}$ & $K_{11}$ & $K_{12}$ & $K_{13}$ & $K_{14}$ & $K_{15}$ \\ \hline
  8 & 9 & 10 & 11 & 12 & 13 & 14 & 15 \\ \hline
\end{tabular}

Задаю $j = 0$.

Теперь вычисляется цикл от $i = 0$ до $i = 15$.

Для $i = 0$: $j = (0 + S_0 + K_0) \mod 16 = (0 + 0 + 0) \mod 16 = 0 \mod 16 = 0$. Меняем местами значения для $S_0$ и $S_0$.

Для $i = 1$: $j = (0 + 1 + 1) \mod 16 = 2$. Меняем местами $S_1$ и $S_2$. В результате $S_1 = 2$ и $S_2 = 1$.

Для $i = 2$: $j = (2 + 1 + 2) \mod 16 = 5$. Меняем местами $S_2$ и $S_5$. В результате $S_2 = 5$ и $S_5 = 1$.

Для $i = 3$: $j = (5 + 3 + 3) \mod 16 = 11$. Меняем местами $S_3$ и $S_{11}$. В результате $S_3 = 5$ и $S_{11} = 1$.

Для $i = 4$: $j = (11 + 4 + 4) \mod 16 = 3$. Меняем местами $S_4$ и $S_{3}$. В результате $S_4 = 11$ и $S_{3} = 4$.

Для $i = 5$: $j = (3 + 1 + 5) \mod 16 = 9$. Меняем местами $S_5$ и $S_{9}$. В результате $S_5 = 9$ и $S_{9} = 1$.

Для $i = 6$: $j = (9 + 6 + 6) \mod 16 = 5$. Меняем местами $S_6$ и $S_{5}$. В результате $S_6 = 9$ и $S_{5} = 6$.

Для $i = 7$: $j = (5 + 7 + 7) \mod 16 = 3$. Меняем местами $S_7$ и $S_{3}$. В результате $S_7 = 4$ и $S_{3} = 7$.

Для $i = 8$: $j = (3 + 8 + 8) \mod 16 = 3$. Меняем местами $S_8$ и $S_{3}$. В результате $S_8 = 7$ и $S_{3} = 8$.

Для $i = 9$: $j = (3 + 1 + 9) \mod 16 = 13$. Меняем местами $S_9$ и $S_{13}$. В результате $S_9 = 13$ и $S_{13} = 1$.

Для $i = 10$: $j = (13 + 10 + 10) \mod 16 = 1$. Меняем местами $S_{10}$ и $S_{1}$. В результате $S_{10} = 2$ и $S_{1} = 10$.

Для $i = 11$: $j = (1 + 3 + 11) \mod 16 = 15$. Меняем местами $S_{11}$ и $S_{15}$. В результате $S_{11} = 15$ и $S_{15} = 3$.

Для $i = 12$: $j = (15 + 12 + 12) \mod 16 = 7$. Меняем местами $S_{12}$ и $S_{7}$. В результате $S_{12} = 4$ и $S_{7} = 12$.

Для $i = 13$: $j = (7 + 1 + 13) \mod 16 = 5$. Меняем местами $S_{13}$ и $S_{5}$. В результате $S_{13} = 6$ и $S_{5} = 1$.

Для $i = 14$: $j = (5 + 14 + 14) \mod 16 = 1$. Меняем местами $S_{14}$ и $S_{1}$. В результате $S_{14} = 10$ и $S_{1} = 14$.

Для $i = 15$: $j = (1 + 3 + 15) \mod 16 = 3$. Меняем местами $S_{15}$ и $S_{3}$. В результате $S_{15} = 8$ и $S_{3} = 3$.

В итоге таблица состояния имеет вид

\begin{tabular}{l|l|l|l|l|l|l|l}
  $S_0$ & $S_1$ & $S_2$ & $S_3$ & $S_4$ & $S_5$ & $S_6$ & $S_7$ \\ \hline
  0 & 14 & 5 & 3 & 11 & 1 & 9 & 12 \\ \hline
  $S_8$ & $S_9$ & $S_{10}$ & $S_{11}$ & $S_{12}$ & $S_{13}$ & $S_{14}$ & $S_{15}$ \\ \hline
  7 & 13 & 2 & 15 & 4 & 6 & 10 & 8 \\ \hline
\end{tabular}

Поскольку у меня урезанная версия таблицы состояний, то будет видоизменён алгоритм вычисления следующего байта result (гамма):

\begin{enumerate}
\item {$i := (i+1) \mod 16$,}
\item {$j := (j + S_i) \mod 16$,}
\item {замена местами $S_i$ и $S_j$,}
\item {$t := (S_i + S_j ) \mod 16$,}
\item {$\text{result} := S_t$}
\end{enumerate}

Найдём очередную гамму. Вычисляем $i$: $i = (15 + 1) \mod 16 = 16 \mod 16 = 0$. Вычисляем $j$: $j = (3 + S_0) \mod 16 = (3 + 0) \mod 16 = 3 \mod 16 = 3$. Меняем местами $S_0$ и $S_3$. Получаем, что $S_0 = 3$ и $S_3 = 0$. Вычисляем $t$: $t = (S_0 + S_3) \mod 16 = (0 + 3) \mod 16 = 3$. И $\text{result} = S_3 = 0$.

Шифровка выполняется с помощью операции исключающее-или. В данном случае применение операции исключающего-или не приведёт к шифровке и байт будет передан без преобразования.

Найдём очередную гамму. Вычисляем $i$: $i = (0 + 1) \mod 16 = 1$. Вычисляем $j$: $j = (3 + S_1) \mod 16 = (3 + 14) \mod 16 = 1$. Меняем местами $S_1$ и $S_1$. Получаем, что $S_1 = 14$ и $S_1 = 14$. Вычисляем $t$: $t = (S_1 + S_1) \mod 16 = (14 + 14) \mod 16 = 12$. И $\text{result} = S_{12} = 4$.

Пусть нужно зашифровать байт со значением 153, в двоичном виде 153 выглядит как 10011001 и гамма 00000100. После применения ис\-клю\-чаю\-щего-или получим 10011101 в двоичном виде или 157 в десятичном.

Принимающая сторона, в свою очередь применит эту же операцию и получит из 157 исходный байт со значением 153.
\end{document}