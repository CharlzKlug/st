\documentclass[10pt]{article}
\usepackage[T2A]{fontenc}
\usepackage[utf8]{inputenc}
\usepackage[english,russian]{babel}
\usepackage{amsmath}
\author{Аметов Имиль, гр. М07-903}
\title{Blum-Blum-Shub}

\begin{document}
\maketitle
Дано: $p = 67$, $q = 31$. Требуется вычислить начальную последовательность генератора псевдослучайных чисел Blum-Blum-Shub и указать период.

Решение: генератор псевдослучайных чисел Blum-Blum-Shub вычисляется по формуле $$x_{N+1} = x_N^2 \mod {n},$$ где $n = pq = 67 \cdot 31 = 2077$, а $N$ --- номер элемента последовательности.
В качестве начального элемента было выбрано $x_0 = 3$. Была вычислена следующая последовательность: $$x_0 = 3,$$ $$x_1 = 3^2 \mod 2077 = 9 \mod 2077 = 9,$$
$$x_2 = 9^2 \mod 2077 = 81 \mod 2077 = 81,$$
$$x_3 = 81^2 \mod 2077 = 6561 \mod 2077 = 330,$$
$$x_4 = 330^2 \mod 2077 = 108900 \mod 2077 = 896,$$
$$x_5 = 896^2 \mod 2077 = 802816 \mod 2077 = 1094,$$
$$x_6 = 1094^2 \mod 2077 = 1196836 \mod 2077 = 484,$$
$$x_7 = 484^2 \mod 2077 = 234256 \mod 2077 = 1632,$$
$$x_8 = 1632^2 \mod 2077 = 2663424 \mod 2077 = 710,$$
$$x_9 = 710^2 \mod 2077 = 504100 \mod 2077 = 1466,$$
$$x_{10} = 1466^2 \mod 2077 = 2149156 \mod 2077 = 1538,$$
$$x_{11} = 1538^2 \mod 2077 = 2365444 \mod 2077 = 1818,$$
$$x_{12} = 1818^2 \mod 2077 =3305124 \mod 2077 = 617,$$
$$x_{13} = 617^2 \mod 2077 = 380689 \mod 2077 = 598,$$
$$x_{14} = 598^2 \mod 2077 = 357604 \mod 2077 = 360,$$
$$x_{15} = 360^2 \mod 2077 = 129600 \mod 2077 = 826,$$
$$x_{16} = 826^2 \mod 2077 = 682276 \mod 2077 = 1020,$$
$$x_{17} = 1020^2 \mod 2077 = 1040400 \mod 2077 = 1900,$$
$$x_{18} = 1900^2 \mod 2077 = 3610000 \mod 2077 = 174,$$
$$x_{19} = 174^2 \mod 2077 = 30276 \mod 2077 = 1198,$$
$$x_{20} = 1198^2 \mod 2077 = 1435204 \mod 2077 = 2074,$$
$$x_{21} = 2074^2 \mod 2077 = 4301476 \mod 2077 = 9.$$
Элементы последовательности $x_1$ и $x_{21}$ совпадают и равны $9$. Следовательно, период генератора псевдослучайных чисел Blum-Blum-Shub для $p = 67$, $q = 31$ и $x_0 = 3$ равен $20$.
\end{document}