\documentclass[10pt]{article}
\usepackage[T2A]{fontenc}
\usepackage[utf8]{inputenc}
\usepackage[english,russian]{babel}
\usepackage{amsmath}
\usepackage {multirow}
\author{Аметов Имиль, гр. М07-903}
\title{Эллиптические кривые}

\begin{document}
\maketitle

\emph{Задача:}

Приведите пример эллиптической кривой над $GF(p)$, где $p = 71$.

\emph{Решение:}

Я выбрал эллиптическую кривую $y^2 = x^3 + 3x + 7$. Здесь

$$4 \cdot 3^3 + 27 \cdot 7^2 = 11 \mod 71.$$

Это удовлетворяет условию $4a^3 + 27 b^2 \not \equiv 0 (\mod p)$.

У меня получились следующие точки :

$$P_1 = O, P_2 = (3; 16), P_3 = (3;55), P_4 = (4;15),$$
$$P_5 = (4;56), P_6 = (5;17), P_7 = (5;54), P_8 = (7;4),$$
$$P_9 = (7;67), P_{10} = (10,16), P_{11} = (10;55), P_{12} = (14;33),$$
$$P_{13} = (14;38), P_{14} = (15;27), P_{15} = (15;44), P_{16} = (17;1),$$
$$P_{17} = (17;70), P_{18} = (18;0), P_{19} = (19;6), P_{20} = (19;65),$$
$$P_{21} = (21;32), P_{22} = (21;39), P_{23} = (22;0), P_{24} = (24;22),$$
$$P_{25} = (24;49), P_{26} = (25;4), P_{27} = (25;67), P_{28} = (31;0),$$
$$P_{29} = (34;24), P_{30} =(34;47), P_{31} = (35;23), P_{32} = (35;48),$$
$$P_{33} = (37;19), P_{34} = (37;52), P_{35} = (39;4), P_{36} = (39;67),$$
$$P_{37} = (45;23), P_{38} = (45;48), P_{39} = (47;13), P_{40} = (47;58),$$
$$P_{41} = (48;14), P_{42} = (48;57), P_{43} = (52;7), P_{44} = (52;64),$$
$$P_{45} = (58;16), P_{46} = (58;55), P_{47} = (59;35), P_{48} = (59;36),$$
$$P_{49} = (62;23), P_{50} = (62;48), P_{51} = (65;25), P_{52} = (65;46),$$
$$P_{53} = (66;3), P_{54} = (66;68), P_{55} = (67;12), P_{56} = (67;59),$$
$$P_{57} = (69;8), P_{58} = (69;63), P_{59} = (70;28), P_{60} = (70;43).$$

Порядок эллиптической кривой оказался равен 60. Этот порядок также подтверждается теоремой Хассе:

$$(\sqrt{71} - 1)^2 \le |E(GF(71))| \le (\sqrt{71} + 1)^2,$$
или

$$55 \le |E(GF(71))| \le 88.$$
Делители 60 следующие: 1, 2, 3, 4, 5, 6, 10, 15, 30, 60. Среди всех точек я нашёл точку $P_8 = (7; 4)$ с рангом 30.

Доказательство:

$$2 P_8 = (58;16), 3 P_8 = (66;3), 4 P_8 = (34; 47), 5 P_8 = (4; 56),$$
$$6 P_8 = (37; 19), 7 P_8 = (45; 48), 8 P_8 = (5;17), 9 P_8 = (48; 14),$$
$$10 P_8 = (24; 49), 11 P_8 = (19; 6), 12 P_8 = (47; 13), 13 P_8 = (21; 39),$$
$$14 P_8 = (67;59), 15 P_8 = (22; 0), 16 P_8 = (67; 12), 17 P_8 = (21; 32), $$
$$18 P_8 = (47; 58), 19 P_8 = (19;65), 20 P_8 = (24; 22), 21 P_8 = (48; 57),$$
$$22 P_8 = (5; 54), 23  P_8 = (45; 23), 24 P_8 = (37; 52), 25 P_8 = (4; 15),$$
$$26 P_8 = (34; 24), 27 P_8 = (66,68), 28 P_8 = (58; 55), 29 P_8 = (7; 67),$$
$$30 P_8 = O.$$

Теперь выбираю закрытый ключ $d$, который больше нуля и меньше ранга выбранной точки (т.е., меньше 30 в моём случае). Я выбрал $d = 23$.

Находим точку $Q = dP = 23 \cdot (7; 4) = (45; 23)$.

В результате, получен открытый ключ
$$[(a, b), P, p, Q] = [(3,7), (7;4), 71, (45; 23)].$$

\emph{Зашифровывание сообщения}. Пусть надо зашифровать сообщение
$$m = 50.$$

Выбирается случайное число $k (0 < k < p)$. Я выбрал $k = 11$. Вычисляется точка $P_k = k \cdot P = 11 \cdot (7;4) = (19; 6)$. После этого вычисляется точка $Q_k(x_{q_k}, y_{q_k}) = k \cdot Q = 11 \cdot (45; 23) = (21; 39)$. Вычисляется
$$c = m \cdot x_{qk} \mod p = 50 \cdot 21 \mod 71 = 56.$$

Получен зашифрованный текст $[P_k, c]$. В моём случае, это $[(19; 6), 56]$.

\emph{Расшифровывание сообщения}. Вычисляется точка
$$D(x_d; y_d) = d \cdot P_k = 23 \cdot (19; 6) = (21; 39).$$

Теперь вычисляем сообщение
$$m = c \cdot x_d^{-1} \mod p = 56 \cdot 21^{-1} \mod 71 = 56 \cdot 44 \mod 71 =$$
$$= 50 \mod 71.$$

Окончательно получил $m = 50$.
\end{document}