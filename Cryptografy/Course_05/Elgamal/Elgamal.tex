\documentclass[10pt]{article}
\usepackage[T2A]{fontenc}
\usepackage[utf8]{inputenc}
\usepackage[english,russian]{babel}
\usepackage{amsmath}
\usepackage {multirow}
\author{Аметов Имиль, гр. М07-903}
\title{Схема Эль-Гамаля}

\begin{document}
\maketitle

\emph{Задача:}

Продемонстрировать зашифрование/расшифрование по схеме Эль-Гамаля. Использовать $p = 71$ и генератор $g = 7$.

\emph{Решение:}

Порядок группы $\varphi (p) = p - 1 = 71 - 1 = 70$.

Делители 70: 1, 2, 5, 7, 10, 14, 35. Проверяем, является ли $g = 7$ генератором группы:

\begin{tabular}{|c|c|c|c|c|c|c|c|c|}
  \hline
  \multirow {2}{*}{Элемент} & \multicolumn {7}{c|}{Степени} & \multirow {2}{*}{Порядок элемента} \\ \cline{2 - 8}
  \multirow {2} {*}{} & 2 & 5 & 7 & 10 & 14 & 35 & 70 & \\ \hline
  7 & -22 & -20 & 14 & -26 & -17 & -1 & 1 & 70 \\ \hline
\end{tabular}

Выбираю случайное $x = 53 \in [0, 70]$.

Вычисляю

$$y = g^x \mod p = 7^{53} \mod 71 = -8 \mod 71.$$

Получаю следующие открытые и закрытые ключи: $\text{PK} = (p : 71, g : 7, y : -8)$, $\text{SK} = (p : 71, g : 7, x : 53)$.

Пусть нужно зашифровать $m = 42$. Выбираю случайное число $r = 29 \in [1, 70]$.

Вычисляю

$$a = g^r \mod p = 7^{29} \mod 71 = 35 \mod 71.$$
$$b = 42 \cdot (-8)^{29} \mod 71 = 32 \mod 71.$$

Получаю шифртекст

$$c = (a : 35, b : 32).$$

Теперь осуществляю расшифровку

$$m = \dfrac {b} {a^x} \mod p = 32 \cdot (35^{-1})^{53} \mod 71 = $$
$$= 32 \cdot (-2)^{53} \mod 71 = 42.$$

Получил открытый текст $m = 42$.
\end{document}