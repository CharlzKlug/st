\section{Заключение}

Ключевым фактором, определяющим сегодня развитие ИИ-технологий, считается темп роста вычислительной мощности компьютеров, так как принципы работы человеческой психики по-прежнему остаются неясными (на доступном для моделирования уровне детализации). Поэтому тематика ИИ-конференций выглядит достаточно стандартно и по составу почти не меняется уже довольно давно.

Но рост производительности современных компьютеров в сочетании с повышением качества алгоритмов периодически делает возможным применение различных научных методов на практике. Так случилось с интеллектуальными игрушками, так происходит с домашними роботами. Снова будут интенсивно развиваться временно забытые методы простого перебора вариантов (как в шахматных программах), обходящиеся крайне упрощенным описанием объектов. Но с помощью такого подхода (главный ресурс для его успешного применения --- производительность) удастся решить, как ожидается, множество самых разных задач (например, из области криптографии). Уверенно действовать автономным устройствам в сложном мире помогут достаточно простые, но ресурсоемкие алгоритмы адаптивного поведения. При этом ставится цель разрабатывать системы, не внешне похожие на человека, а действующие, как человек.

Ученые пытаются заглянуть и в более отдаленное будущее. Можно ли создать автономные устройства, способные при необходимости самостоятельно собирать себе подобные копии (размножаться)? Способна ли наука создать соответствующие алгоритмы? Сможем ли мы контролировать такие машины? Ответов на эти вопросы пока нет. Продолжится активное внедрение формальной логики в прикладные системы представления и обработки знаний. В то же время такая логика не способна полноценно отразить реальную жизнь, и произойдет интеграция различных систем логического вывода в единых оболочках. При этом, возможно, удастся перейти от концепции детального представления информации об объектах и приемов манипулирования этой информацией к более абстрактным формальным описаниям и применению универсальных механизмов вывода, а сами объекты будут характеризоваться небольшим массивом данных, основанных на вероятностных распределениях характеристик.

Сфера ИИ, ставшая зрелой наукой, развивается постепенно --- медленно, но неуклонно продвигаясь вперед. Поэтому результаты достаточно хорошо прогнозируемы, хотя на этом пути не исключены и внезапные прорывы, связанные со стратегическими инициативами. 
