\section{Искусственный интеллект, как направление исследований}

В понятие ``искусственный интеллект'' вкладывается различный смысл --- от признания интеллекта у ЭВМ, решающих логические или даже любые вычислительные задачи, до отнесения к интеллектуальным лишь тех систем, которые решают весь комплекс задач, осуществляемых человеком, или еще более широкую их совокупность. Постараемся же определить тот смысл понятия ``искусственный интеллект'', который в наибольшей степени соответствует реальным исследованиям в этой области. В исследованиях по искусственному интеллекту ученые отвлекаются от сходства процессов, происходящих в технической системе или в реализуемых ею программах, с мышлением человека. Если система решает задачи, которые человек обычно решает посредством своего интеллекта, то мы имеем дело с системой искусственного интеллекта. Однако это ограничение недостаточно. Создание традиционных программ для ЭВМ --- работа программиста --- не есть конструирование искусственного интеллекта. Какие же задачи, решаемые техническими системами, можно рассматривать как конструирующие искусственный интеллект? Чтобы ответить на этот вопрос, надо уяснить, прежде всего, что такое задача. Как отмечают психологи, этот термин тоже не является достаточно определенным. По-видимому, в качестве исходного можно принять понимание задачи как мыслительной задачи, существующее в психологии. Они подчеркивают, что задача есть только тогда, когда есть работа для мышления, т. е. когда имеется некоторая цель, а средства к ее достижению не ясны; их надо найти посредством мышления. По этому поводу знаменитый математик Д. Пойа сказал: ``...трудность решения в какой-то мере входит в само понятие задачи: там, где нет трудности, нет и задачи''. Если человек имеет очевидное средство, с помощью которого, наверное, можно осуществить желание, поясняет он, то задачи не возникает. Если человек обладает алгоритмом решения некоторой задачи и имеет физическую возможность его реализации, то задачи в собственном смысле уже не существует. Так понимаемая задача, в сущности, тождественна проблемной ситуации, и решается она посредством преобразования последней. В ее решении участвуют не только условия, которые непосредственно заданы. Человек использует любую находящуюся в его памяти информацию, ``модель мира'', имеющуюся в его психике и включающую фиксацию разнообразных законов, связей, отношений этого мира. Если задача не является мыслительной, то она решается на ЭВМ традиционными методами и, значит, не входит в круг задач искусственного интеллекта. Ее интеллектуальная часть выполнена человеком. На долю машины осталась часть работы, которая не требует участия мышления, т. е. ``бессмысленная'', неинтеллектуальная. Под словом ``машина'' здесь понимается машина вместе с ее совокупным математическим обеспечением, включающим не только программы, но и необходимые для решения задач ``модели мира''. Задачи, решаемые искусственным интеллектом, целесообразно определить таким образом, чтобы человек, по крайней мере, в определении отсутствовал. Основная функция мышления заключается в выработке схем целесообразных внешних действий в бесконечно варьирующих условиях. Специфика человеческого мышления (в отличие от рассудочной деятельности животных) состоит в том, что человек вырабатывает и накапливает знания, храня их в своей памяти. Выработка схем внешних действий происходит не по принципу ``стимул --- реакция'', а на основе знаний, получаемых дополнительно из среды, для поведения в которой вырабатывается схема действия. Этот способ выработки схем внешних действий (а не просто действия по командам, пусть даже меняющимся как функции от времени или как однозначно определенные функции от результатов предшествующих шагов) является существенной характеристикой любого интеллекта. Отсюда следует, что к системам искусственного интеллекта относятся те, которые, используя заложенные в них правила переработки информации, вырабатывают новые схемы целесообразных действий на основе анализа моделей среды, хранящихся в их памяти. Способность к перестройке самих этих моделей в соответствии с вновь поступающей информацией является свидетельством более высокого уровня искусственного интеллекта. Большинство исследователей считают наличие собственной внутренней модели мира у технических систем предпосылкой их ``интеллектуальности''. Формирование такой модели связано с преодолением синтаксической односторонности системы, т.е. с тем, что символы или та их часть, которой оперирует система, интерпретированы, имеют семантику. Характеризуя особенности систем искусственного интеллекта, специалисты указывают на: 1) наличие в них собственной внутренней модели внешнего мира; эта модель обеспечивает индивидуальность, относительную самостоятельность системы в оценке ситуации, возможность семантической и прагматической интерпретации запросов к системе; 2) способность пополнения имеющихся знаний; 3) способность к дедуктивному выводу, т.е. к генерации информации, которая в явном виде не содержится в системе; это качество позволяет системе конструировать информационную структуру с новой семантикой и практической направленностью; 4) умение оперировать в ситуациях, связанных с различными аспектами нечеткости, включая ``понимание'' естественного языка; 5) способность к диалоговому взаимодействию с человеком; 6) способность к адаптации. На вопрос, все ли перечисленные условия обязательны, необходимы для признания системы интеллектуальной, ученые отвечают по-разному. В реальных исследованиях, как правило, признается абсолютно необходимым наличие внутренней модели внешнего мира, и при этом считается достаточным выполнение хотя бы одного из перечисленных выше условий. П. Армер выдвинул мысль о ``континууме интеллекта'': различные системы могут сопоставляться не только как имеющие и не имеющие интеллекта, но и по степени его развития. При этом, считает он, желательно разработать шкалу уровня интеллекта, учитывающую степень развития каждого из его необходимых признаков. Известно, что в свое время А.Тьюринг предложил в качестве критерия, определяющего, может ли машина мыслить, ``игру в имитацию''. Согласно этому критерию, машина может быть признана мыслящей, если человек, ведя с ней диалог по достаточно широкому кругу вопросов, не сможет отличить ее ответов от ответов человека. Критерий Тьюринга в литературе был подвергнут критике с различных точек зрения. Действительно серьезный аргумент против этого критерия заключается в том, что в подходе Тьюринга ставится знак тождества между способностью мыслить и способностью к решению задач переработки информации определенною типа. Успешная ``игра в имитацию'' не может без тщательного предварительного анализа мышления как целостности быть признана критерием способности машины к мышлению. Однако этот аргумент бьет мимо цели, если мы говорим не о мыслящей машине, а об искусственном интеллекте, который должен лишь продуцировать физические тела знаков, интерпретируемые человеком в качестве решений определенных задач. Поэтому прав В.М. Глушков, утверждая, что наиболее естественно, следуя Тьюрингу, считать, что некоторое устройство, созданное человеком, представляет собой искусственный интеллект, если, ведя с ним достаточно долгий диалог по более или менее широкому кругу вопросов, человек не сможет различить, разговаривает он с разумным живым существом или с автоматическим устройством. Если учесть возможность разработки программ, специально рассчитанных на введение в заблуждение человека, то, возможно, следует говорить не просто о человеке, а о специально подготовленном эксперте. Этот критерий, на взгляд многих ученых, не противоречит перечисленным выше особенностям системы искусственного интеллекта. Но что значит по ``достаточно широкому кругу вопросов'', о котором идет речь в критерии Тьюринга и в высказывании В.М.Глушкова? На начальных этапах разработки проблемы искусственного интеллекта ряд исследователей, особенно занимающихся эвристическим программированием, ставили задачу создания интеллекта, успешно функционирующего в любой сфере деятельности. Это можно назвать разработкой ``общего интеллекта''. Сейчас большинство работ направлено на создание ``профессионального искусственного интеллекта'', т. е. систем, решающих интеллектуальные задачи из относительно ограниченной области (например, управление портом, интегрирование функций, доказательство теорем геометрии и т.п.). В этих случаях ``достаточно широкий круг вопросов'' должен пониматься как соответствующая область предметов. Исходным пунктом рассуждений об искусственном интеллекте было определение такой системы как решающей мыслительные задачи. Но перед нею ставятся и задачи, которые люди обычно не считают интеллектуальными, поскольку при их решении человек сознательно не прибегает к перестройке проблемных ситуаций. К их числу относится, например, задача распознания зрительных образов. Человек узнает человека, которого видел один-два раза, непосредственно в процессе чувственного восприятия. Исходя из этого, кажется, что эта задача не является интеллектуальной. Но в процессе узнавания человек не решает мыслительных задач лишь постольку, поскольку программа распознания не находится в сфере осознанного. Но так как в решении таких задач на неосознанном уровне участвует модель среды, хранящаяся в памяти, то эти задачи, в сущности, являются интеллектуальными. Соответственно и система, которая ее решает, может считаться интеллектуальной. Тем более это относится к ``пониманию'' машиной фраз на естественном языке, хотя человек в этом не усматривает обычно проблемной ситуации. Теория искусственного интеллекта при решении многих задач сталкивается с проблемами познания. Одна из таких проблем состоит в выяснении вопроса, доказуема ли теоретически (математически) возможность или невозможность искусственного интеллекта. На этот счет существуют две точки зрения. Одни считают математически доказанным, что ЭВМ в принципе может выполнить любую функцию, осуществляемую естественным интеллектом. Другие полагают в такой же мере доказанным математически, что есть проблемы, решаемые человеческим интеллектом, которые принципиально недоступны ЭВМ. Эти взгляды высказываются как кибернетиками, так и философами. Знание --- основа интеллектуальной системы. Многие виды умственной деятельности человека, такие, как написание программ для вычислительной машины, занятие математикой, ведение рассуждений на уровне здравого смысла и даже вождение автомобиля --- требуют ``интеллекта''. На протяжении последних десятилетий было построено несколько типов компьютерных систем, способных выполнять подобные задачи. Имеются системы, способные диагностировать заболевания, планировать синтез сложных синтетических соединений, решать дифференциальные уравнения в символьном виде, анализировать электронные схемы, понимать ограниченный объем человеческой речи и естественного языкового текста. Можно сказать, что такие системы обладают в, некоторой степени, искусственным интеллектом. Работа по построению таких систем проводится в области, получившей название искусственный интеллект (ИИ). При реализации интеллектуальных функций непременно присутствует информация, называемая знаниями. Другими словами, интеллектуальные системы являются в то же время системами обработки знаний. В настоящее время в исследованиях по искусственному интеллекту выделились несколько основных направлений.

\begin{enumerate}

\item{Представление знаний. В рамках этого направления решаются задачи, связанные с формализацией и представлением знаний в памяти системы ИИ. Для этого разрабатываются специальные модели представления знаний и языки описания знаний, внедряются различные типы знаний. Проблема представления знаний является одной из основных проблем для системы ИИ, так как функционирование такой системы опирается на знания о проблемной области, которые хранятся в ее памяти.}

\item{Манипулирование знаниями. Чтобы знаниями можно было пользоваться при решении задачи, следует научить систему ИИ оперировать ими. В рамках данного направления разрабатываются способы пополнения знаний на основе их неполных описаний, создаются методы достоверного и правдоподобного вывода на основе имеющихся знаний, предлагаются модели рассуждений, опирающихся на знания и имитирующих особенности человеческих рассуждений. Манипулирование знаниями очень тесно связано с представлением знаний, и разделить эти два направления можно лишь условно.}

\item{Общение. В круг задач этого направления входят: проблема понимания и синтеза связных текстов на естественном языке, понимание и синтез речи, теория моделей коммуникаций между человеком и системой ИИ. На основе исследований в этом направлении формируются методы построения лингвистических процессов, вопросно-ответных систем, диалоговых систем и других систем ИИ, целью которых является обеспечение комфортных условий для общения человека с системой ИИ.}

\item{Восприятие. Это направление включает разработку методов представления информации о зрительных образах в базе знаний, создание методов перехода от зрительных сцен к их текстовому описанию и методов обратного перехода, создание средств, порождающих зрительные сцены на основе внутренних представлений в системах ИИ.}

\item{Обучение. Для развития способности систем ИИ к обучению, т.е. к решению задач, с которыми они раньше не встречались, разрабатываются методы формирования условий задач по описанию проблемной ситуации или по наблюдению за ней, методы перехода от известного решения частных задач (примеров) к решению общей задачи, создание приемов разбиения исходной задачи на более мелкие и уже известные для систем ИИ. В этом направлении ИИ сделано еще весьма мало.}

\item{Поведение. Поскольку системы ИИ должны действовать в некоторой окружающей среде, то необходимо разрабатывать некоторые поведенческие процедуры, которые позволили бы им адекватно взаимодействовать с окружающей средой, другими системами ИИ и людьми. Это направление в ИИ также разработано ещё очень слабо. В последние годы термин "знание" все чаще употребляется в информатике. Специалисты подчеркивают, что совершенствование так называемых интеллектуальных систем (информационно-поисковых систем высокого уровня, диалоговых систем, базирующихся на естественных языках, интерактивных человеко-машинных систем, используемых в управлении, проектировании, научных исследованиях) во многом определяется тем, насколько успешно будут решаться задачи (проблемы) представления знаний.}
  
\end{enumerate}

\subsection{Некоторые подходы к решению проблемы ИИ}

\paragraph{Механистический подход}

Идея создания мыслящих машин ``человеческого типа'', которые думают, двигаются, слышат, говорят, и вообще ведут себя как живые люди уходит корнями в глубокое прошлое. Еще в античности люди стремились создать машину, подобную себе. В 1736 г. французский изобретатель Жак де Вокансон изготовил механического флейтиста в человеческий рост, который исполнял двенадцать мелодий, перебирая пальцами отверстия и дуя в мундштук, как настоящий музыкант. В середине 1750-х годов Фридрих фон Кнаус, австрийский автор, служивший при дворе Франциска I, сконструировал серию машин, которые умели держать перо и могли писать довольно длинные тексты. Другой мастер, Пьер Жак-Дроз из Швейцарии, построил пару изумительных по сложности механических кукол размером с ребенка: мальчика, пишущего письма и девушку, играющую на клавесине. Успехи механики XIX в. стимулировали еще более честолюбивые замыслы. Так, в 1830-х годах английский математик Чарльз Бэббидж задумал, правда, так и не завершив, сложный цифровой калькулятор, который он назвал Аналитической машиной; как утверждал Бэббидж, его машина в принципе могла бы рассчитывать шахматные ходы. Позднее, в 1914 г., директор одного из испанских технических институтов Леонардо Торрес-и-Кеведо действительно изготовил электромеханическое устройство, способное разыгрывать простейшие шахматные эндшпили почти также хорошо, как и человек. Но все эти механические устройства имеют лишь отдаленное сходство с тем, что может быть названо ИИ, хотя интересны с исторической точки зрения.

\paragraph{Электронный подход}

После второй мировой войны появились устройства, казалось бы, подходящие для достижения заветной цели --- моделирования разумного поведения; это были электронные цифровые вычислительные машины. К концу 50-х годов все эти увлечения выделились в новую более или менее самостоятельную ветвь информатики, получившую название ``искусственный интеллект''. Исследования в области ИИ, первоначально сосредоточенные в нескольких университетских центрах США --- Массачусетском технологическом институте, Технологическом институте Карнеги в Питтсбурге, Станфордском университете, --- ныне ведутся во многих других университетах и корпорациях США и других стран. Исследователей ИИ, работающих над созданием мыслящих машин, можно разделить на две группы. Одних интересует чистая наука и для них компьютер --- лишь инструмент, обеспечивающий возможность экспериментальной проверки теорий процессов мышления. Интересы другой группы лежат в области техники: они стремятся расширить сферу применения компьютеров и облегчить пользование ими. Многие представители второй группы мало заботятся о выяснении механизма мышления --- они полагают, что для их работы это едва ли более полезно, чем изучение полета птиц и самолетостроения. На протяжении всей своей короткой истории исследователи в области ИИ всегда находились на переднем крае информатики. Многие ныне обычные разработки, в том числе усовершенствованные системы программирования, текстовые редакторы и программы распознавания образов, в значительной мере рассматриваются на работах по ИИ. Короче говоря, теории, новые идеи, и разработки ИИ неизменно привлекают внимание тех, кто стремится расширить области применения и возможности компьютеров, сделать их более ``дружелюбными'' то есть более похожими на разумных помощников и активных советчиков, чем те педантичные и туповатые электронные рабы, какими они всегда были. Несмотря на многообещающие перспективы, ни одну из разработанных до сих пор программ ИИ нельзя назвать ``разумной'' в обычном понимании этого слова. Это объясняется тем, что все они узко специализированы; самые сложные экспертные системы по своим возможностям скорее напоминают дрессированных животных или механических кукол, нежели человека с его гибким умом и широким кругозором. Даже среди исследователей ИИ теперь многие сомневаются, что большинство подобных изделий принесет существенную пользу. Немало критиков ИИ считают, что такого рода ограничения вообще непреодолимы, и решение проблемы ИИ надо искать не в сфере непосредственно электроники, а где-то за ее пределами.

\paragraph{Кибернетический подход}

Попытки построить машины, способные к разумному поведению, в значительной мере вдохновлены идеями профессора Массачусетского технологического института, Норберта Винера, одной из выдающихся личностей в интеллектуальной истории Америки и всего мира. Помимо математики он обладал широкими познаниями в других областях, включая нейропсихологию, медицину, физику и электронику. Винер был убежден, что наиболее перспективны научные исследования в так называемых пограничных областях, которые нельзя конкретно отнести к той или иной конкретной дисциплине. Они лежат где-то на стыке наук, поэтому к ним обычно не подходят столь строго. ``Если затруднения в решении какой-либо проблемы психологии имеют математический характер, пояснял он, --- то десять несведущих в математике психологов продвинутся не дальше одного столь же несведущего''. Таким образом, междисциплинарность --- краеугольный камень современной науки. Винеру и его сотруднику Джулиану Бигелоу принадлежит разработка принципа ``обратной связи'', который был успешно применен при разработке нового оружия с радиолокационным наведением. Принцип обратной связи заключается в использовании информации, поступающей из окружающего мира, для изменения поведения машины. В основу разработанных Винером и Бигелоу систем наведения были положены тонкие математические методы; при малейшем изменении отраженных от самолета радиолокационных сигналов они соответственно изменяли наводку орудий, то есть --- заметив попытку отклонения самолета от курса, они тотчас рассчитывали его дальнейший путь и направляли орудия так, чтобы траектории снарядов и самолетов пересеклись. В дальнейшем Винер разработал на принципе обратной связи теории как машинного, так и человеческого разума. Он доказывал, что именно благодаря обратной связи все живое приспосабливается к окружающей среде и добивается своих целей. ``Все машины, претендующие на разумность'', --- писал он, --- ``должны обладать способностью преследовать определенные цели и приспосабливаться, т.е. обучаться''. В 1948 году выходит книга Винера, в которой он заложил фундамент новой науки, названной им кибернетикой, что в переводе с греческого означает рулевой. Следует отметить, что принцип ``обратной связи'', введенный Винером, был в какой-то степени предугадан Сеченовым в описанном им в книге ``Рефлексы головного мозга'' (1863 г.) феномене ``центрального торможения'', т. е. почти за 100 лет до Винера, и рассматривался как механизм регуляции деятельности нервной системы, и который лег в основу многих моделей произвольного поведения в отечественной психологии.

