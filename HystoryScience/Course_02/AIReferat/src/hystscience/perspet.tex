\section{Перспективы и тенденции развития искусственного интеллекта}

Сообщения об уникальных достижениях специалистов в области искусственного интеллекта (ИИ), суливших невиданные возможности, пропали со страниц научно-популярных изданий много лет назад. Эйфория, связанная с первыми практическими успехами в сфере ИИ, прошла довольно быстро, потому что перейти от исследования экспериментальных компьютерных моделей к решению прикладных задач реального мира оказалось гораздо сложнее, чем предполагалось. На трудности такого перехода обратили внимание специалисты всего мира, и после детального анализа выяснилось, что практически все проблемы связаны с нехваткой ресурсов двух типов: компьютерных (вычислительной мощности, емкости оперативной и внешней памяти) и людских (наукоемкая разработка интеллектуального ПО требует привлечения ведущих специалистов из разных областей знания и организации долгосрочных и дорогостоящих исследовательских проектов). К сегодняшнему дню ресурсы первого типа вышли (или выйдут в ближайшие пять-десять лет) на уровень, позволяющий системам ИИ решать весьма сложные для человека практические задачи. А вот с ресурсами второго типа ситуация в мире даже ухудшается --- именно поэтому достижения в сфере ИИ связываются в основном с небольшим числом ведущих ИИ-центров при крупнейших университетах.

\subsection{Нейронные сети}

Это направление стабильно держится на первом месте. Продолжается совершенствование алгоритмов обучения и классификации в масштабе реального времени, обработки естественных языков, распознавания изображений, речи, сигналов, а также создание моделей интеллектуального интерфейса, подстраивающегося под пользователя. Среди основных прикладных задач, решаемых с помощью нейронных сетей, --- финансовое прогнозирование, data mining (буквально ``раскопка данных'', нахождение взаимосвязи в большом количестве наблюдаемых данных), диагностика систем, контроль за деятельностью сетей, шифрование данных. В последние годы идет усиленный поиск эффективных методов синхронизации работы нейронных сетей на параллельных устройствах.

\subsection{Эволюционные вычисления}

На развитие сферы эволюционных вычислений (ЭВ; автономное и адаптивное поведение компьютерных приложений и робототехнических устройств) значительное влияние оказали прежде всего инвестиции в нанотехнологии. ЭВ затрагивают практические проблемы самосборки, самоконфигурирования и самовосстановления систем, состоящих из множества одновременно функционирующих узлов. При этом удается применять научные достижения из области цифровых автоматов. Другой аспект ЭВ --- использование для решения повседневных задач автономных агентов в качестве персональных секретарей, управляющих личными счетами, ассистентов, отбирающих нужные сведения в сетях с помощью поисковых алгоритмов третьего поколения, планировщиков работ, личных учителей, виртуальных продавцов и т. д. Сюда же относится робототехника и все связанные с ней области. Основные направления развития --- выработка стандартов, открытых архитектур, интеллектуальных оболочек, языков сценариев/запросов, методологий эффективного взаимодействия программ и людей. Модели автономного поведения предполагается активно внедрять во всевозможные бытовые устройства, способные убирать помещения, заказывать и готовить пищу, водить автомобили и т. п. В дальнейшем для решения сложных задач (быстрого исследования содержимого Сети, больших массивов данных наподобие геномных) будут использоваться коллективы автономных агентов. Для этого придется заняться изучением возможных направлений эволюции подобных коллективов, планирования совместной работы, способов связи, группового самообучения, кооперативного поведения в нечетких средах с неполной информацией, научиться разрешать конфликты взаимодействия и т. п. Особняком стоят социальные аспекты --- как общество будет на практике относиться к таким сообществам интеллектуальных программ.

\subsection{Нечёткая логика}

Системы нечеткой логики активнее всего будут применяться преимущественно в гибридных управляющих системах.

\subsection{Обработка изображений}

Продолжится разработка способов представления и анализа изображений (сжатие, кодирование при передаче с использованием различных протоколов, обработка биометрических образов, снимков со спутников), независимых от устройств воспроизведения, оптимизации цветового представления на экране и при выводе на печать, распределенных методов получения изображений. Дальнейшие развитие получат средства поиска, индексирования и анализа смысла изображений, согласования содержимого справочных каталогов при автоматической каталогизации, организации защиты от копирования, а также машинное зрение, алгоритмы распознавания и классификации образов.

\subsection{Экспертные системы}

Спрос на экспертные системы остается на достаточно высоком уровне. Наибольшее внимание сегодня привлечено к системам принятия решений в масштабе времени, близком к реальному, средствам хранения, извлечения, анализа и моделирования знаний, системам динамического планирования.

\subsection{Интеллектуальные приложения}

Рост числа интеллектуальных приложений, способных быстро находить оптимальные решения комбинаторных проблем (возникающих, например, в транспортных задачах), связан с производственным и промышленным ростом в развитых странах.

\subsection{Распределённые вычисления}

Распространение компьютерных сетей и создание высокопроизводительных кластеров вызвали интерес к вопросам распределенных вычислений --- балансировке ресурсов, оптимальной загрузке процессоров, самоконфигурированию устройств на максимальную эффективность, отслеживанию элементов, требующих обновления, выявлению несоответствий между объектами сети, диагностированию корректной работы программ, моделированию подобных систем.

\subsection{Операционные системы реального времени}

Появление автономных робототехнических устройств повышает требования к ОС реального времени --- организации процессов самонастройки, планирования обслуживающих операций, использования средств ИИ для принятия решений в условиях дефицита времени. Один из примеров использования операционной системы реального времени --- это Троицкий мост в Санкт-Петербурге. В автоматике, управляющей механизмами моста, применена операционная система QNX.

\subsection{Интеллектуальная инженерия}

Особую заинтересованность в ИИ проявляют в последние годы компании, занимающиеся организацией процессов разработки крупных программных систем (программной инженерией). Методы ИИ все чаще используются для анализа исходных текстов и понимания их смысла, управления требованиями, выработкой спецификаций, проектирования, кодогенерации, верификации, тестирования, оценки качества, выявления возможности повторного использования, решения задач на параллельных системах.

Программная инженерия постепенно превращается в так называемую интеллектуальную инженерию, рассматривающую более общие проблемы представления и обработки знаний (пока основные усилия в интеллектуальной инженерии сосредоточены на способах превращения информации в знания).

\subsection{Самоорганизующиеся системы управления базами данных}

Самоорганизующиеся СУБД будут способны гибко подстраиваться под профиль конкретной задачи и не потребуют администрирования. На данный момент никаких более-менее рабочих прототипов самоорганизующихся СУБД нет.

\subsection{Искусственный интеллект для анализаторских функций}

Автоматический анализ естественных языков (лексический, морфологический, терминологический, выявление незнакомых слов, распознавание национальных языков, перевод, коррекция ошибок, эффективное использование словарей). Возможность высокопроизводительного OLAP-анализ (online analytical processing --- интерактивная аналитическая аналитическая обработка) и раскопка данных, способы визуального задания запросов. Медицинские системы, консультирующие врачей в экстренных ситуациях, ро\-бо\-ты-ма\-ни\-пу\-ля\-то\-ры для выполнения точных действий в ходе хирургических операций. Создание полностью автоматизированных киберзаводов, гибкие экономные производства, быстрое прототипирование, планирование работ, синхронизация цепочек снабжения, авторизации финансовых транзакций путем анализа профилей пользователей. Небольшое число конференций посвящено выработке прикладных методов, направленных на решение конкретных задач промышленности в области финансов, медицины и математики.

Традиционно высок интерес к ИИ в среде разработчиков игр и развлекательных программ (это отдельная тема). Среди новых направлений их исследований --- моделирование социального поведения, общения, человеческих эмоций, творчества.

\subsection{Военные технологии}

Исследования в области нейронных сетей, позволяющих получить хорошие (хотя и приближенные) результаты при решении сложных задач управления, часто финансирует военное научное агентство DARPA (Defense Advanced Research Projects Agency --- Управление перспективных исследовательских проектов Министерства обороны США). Пример --- проект Smart Sensor Web, который предусматривает организацию распределенной сети разнообразных датчиков, синхронно работающих на поле боя. Каждый объект (стоимостью не более 300 долларов США) в такой сети представляет собой источник данных --- визуальных, электромагнитных, цифровых, инфракрасных, химических и т. п. Проект требует новых математических методов решения многомерных задач оптимизации. Ведутся работы по автоматическому распознаванию целей, анализу и предсказанию сбоев техники по отклонениям от типовых параметров ее работы (например, по звуку). Операция ``Буря в пустыне'' стала стимулом к развитию экспертных систем с продвинутым ИИ, применяемым в области снабжения. На разработках, связанных с технологиями машинного зрения, основано все высокоточное оружие. В СМИ нередко можно прочитать о грядущих схватках самостоятельно действующих армий самоходных машин-роботов и беспилотных самолетов. Однако существует ряд нерешенных научных проблем, не позволяющих в ближайшие десятилетия превратить подобные прогнозы в реальность. Прежде всего это недостатки систем автоматического распознавания, не способных правильно анализировать видеоинформацию в масштабе реального времени. Не менее актуальны задачи разрешения коллизий в больших сообществах автономных устройств, абсолютно точного распознавания своих и чужих, выбора подлежащих уничтожению целей, алгоритмов поведения в незнакомой среде и т. п. Поэтому на практике военные пытаются достичь менее масштабных целей. Значительные усилия вкладываются в исследования по распознаванию речи, создаются экспертные и консультационные системы, призванные автоматизировать рутинные работы и снизить нагрузку на пилотов. Нейронные сети достаточно эффективно применяются для обработки сигналов сонаров и отличения подводных камней от мин. Генетические алгоритмы используются для эвристического поиска решения уравнений, определяющих работу военных устройств (систем ориентации, навигации), а также в задачах распознавания --- для разделения искусственных и естественных объектов, распознавания типов военных машин, анализа изображения, получаемого от камеры с низким разрешением или инфракрасных датчиков.
