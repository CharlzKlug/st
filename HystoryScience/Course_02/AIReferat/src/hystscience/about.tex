\Csection{Введение}

Эпоха думающих машин и проблемы человечества.
Человек уже давно начал задумываться об интеллекте и разуме. Что-же такое интеллект? Как его можно описать? Где он расположен? Из чего состоит собственное ``Я'' каждого человека? Над этими вопросами работали и работают большое количество философов, учёных и инженеров. И судя по всему до окончательного ответа ещё очень далеко.

Мы живём в век продолжающегося бурного развития компьютеров и компьютерной техники. За последние пятьдесят лет произошёл стремительный взлёт информационно-вычислительной техники. В 1905 году Джон Флеминг запатентовал ``прибор для преобразования переменного тока в постоянный'' --- первую электронную лампу. Вакуумные электронные лампы стали элементной базой для компьютеров первого поколения. В 1960-м году после работ американцев Канга и Аталлы, на основе кристалла кремния был впервые изготовлен полевой транзистор. Вычислительная техника стала стремительно дешеветь, уменьшаться в размерах, стало сокращаться энергопотребление и вместе с тем начала расти производительность вычислений. Позже возникли интегральные схемы, что в свою очередь ещё более расширило возможности вычислительной техники. Появились языки программирования, стало возможным писать большие и сложные программы которые уже стали представлять не только академический, но уже и практический интересы. Люди стали задумываться: нельзя ли написать такую программу, собрать такую схему чтобы получить уже электронный интеллект?

Эта работа представляет попытку произвести краткий обзор того, что человечеству удалось добиться в создании думающих машин и тех результатов, которых удалось достичь.

\pagebreak
