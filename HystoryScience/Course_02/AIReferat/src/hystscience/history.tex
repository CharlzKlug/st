\section{История развития искусственного интеллекта}

Как наука ``Искусственный интеллект'' имеет достаточно богатую историю. Можно выделить как теоретическую, так и экспериментальную части. Суть науки ``Искусственный интеллект'' лучше всего отражают слова ``Дух в машине'', при этом не столь важно развитие отдельно понятий о машине и духе, как важно их сочетание. Но в то же время понятно, что чем более развиты представления о машине, чем они более совершенны с одной стороны, и чем мы более знаем о духе с другой стороны - тем о более скажем так мощном ИИ мы можем говорить. Но отличает науку ``Искусственный интеллект'' от Вычислительной техники (Информатики) с одной стороны и от Медицины (Биологии) с другой - это именно связь одного с другим. И только при наличии этой связи мы можем говорить о достижениях в области ИИ, а не отдельно в областях Информатики или Биологии. Этому вопросу уделяется особенно большое значение в теоретической части, а для подтверждения теорий как и в других науках используется эксперимент. Но исторически появление теорий и первых экспериментов всегда разнесено во времени. Поэтому начала теории обычно относят к философии искусственного интеллекта, и только с появлением первых экспериментов теория приобретает самостоятельное значение. Причем саму теорию ``Искусственного интеллекта'', которая сейчас находится на рубеже с философией, не нужно совмещать с теорией математических, алгоритмических, робототехнических, физиологических и прочих методов, которые имеют собственное значение в соответствующих науках. Сейчас четкого различия между рядом связанных наук и собственно ``Искусственным интеллектом'' найти очень сложно, а тем более различить теоретические и экспериментальный разделы науки. И здесь главную помощь может оказать история становления и развития науки ``Искусственный интеллект''.

Исторически сложились три основных направления в моделировании ИИ.

В рамках первого подхода объектом исследований являются структура и механизмы работы мозга человека, а конечная цель заключается в раскрытии тайн мышления. Необходимыми этапами исследований в этом направлении являются построение моделей на основе психофизиологических данных, проведение экспериментов с ними, выдвижение новых гипотез относительно механизмов интеллектуальной деятельности, совершенствование моделей и т. д.

Второй подход в качестве объекта исследования рассматривает ИИ. Здесь речь идет о моделировании интеллектуальной деятельности с помощью вычислительных машин. Целью работ в этом направлении является создание алгоритмического и программного обеспечения вычислительных машин, позволяющего решать интеллектуальные задачи не хуже человека.

Наконец, третий подход ориентирован на создание смешанных человеко-машинных, или, как еще говорят, интерактивных интеллектуальных систем, на симбиоз возможностей естественного и искусственного интеллекта. Важнейшими проблемами в этих исследованиях является оптимальное распределение функций между естественным и искусственным интеллектом и организация диалога между человеком и машиной.

Самыми первыми интеллектуальными задачами, которые стали решаться при помощи ЭВМ были логические игры (шашки, шахматы), доказательство теорем. Хотя, правда здесь надо отметить еще кибернетические игрушки типа ``электронной мыши'' Клода Шеннона, которая управлялась сложной релейной схемой. Эта мышка могла ``исследовать'' лабиринт, и находить выход из него. А кроме того, помещенная в уже известный ей лабиринт, она не искала выход, а сразу же, не заглядывая в тупиковые ходы, выходила из лабиринта.

Американский кибернетик А. Самуэль составил для вычислительной машины программу, которая позволяет ей играть в шашки, причем в ходе игры машина обучается или, по крайней мере, создает впечатление, что обучается, улучшая свою игру на основе накопленного опыта. В 1962 г. эта программа сразилась с Р. Нили, сильнейшим шашистом в США и победила.

Каким образом машине удалось достичь столь высокого класса игры?

Естественно, что в машину были программно заложены правила игры так, что выбор очередного хода был подчинен этим правилам. На каждой стадии игры машина выбирала очередной ход из множества возможных ходов согласно некоторому критерию качества игры. В шашках (как и в шахматах) обычно невыгодно терять свои фигуры, и, напротив, выгодно брать фигуры противника. Игрок (будь он человек или машина), который сохраняет подвижность своих фигур и право выбора ходов и в то же время держит под боем большое число полей на доске, обычно играет лучше своего противника, не придающего значения этим элементам игры. Описанные критерии хорошей игры сохраняют свою силу на протяжении всей игры, но есть и другие критерии, которые относятся к отдельным ее стадиям — дебюту, миттэндшпилю, эндшпилю.

Разумно сочетая такие критерии (например в виде линейной комбинации с экспериментально подбираемыми коэффициентами или более сложным образом), можно для оценки очередного хода машины получить некоторый числовой показатель эффективности — оценочную функцию. Тогда машина, сравнив между собой показатели эффективности очередных ходов, выберет ход, соответствующий наибольшему показателю. Подобная автоматизация выбора очередного хода не обязательно обеспечивает оптимальный выбор, но все же это какой-то выбор, и на его основе машина может продолжать игру, совершенствуя свою стратегию (образ действия) в процессе обучения на прошлом опыте. Формально обучение состоит в подстройке параметров (коэффициентов) оценочной функции на основе анализа проведенных ходов и игр с учетом их исхода.

По мнению А. Самуэля, машина, использующая этот вид обучения, может научиться играть лучше, чем средний игрок, за относительно короткий период времени.

Можно сказать, что все эти элементы интеллекта, продемонстрированные машиной в процессе игры в шашки, сообщены ей автором программы. Отчасти это так. Но не следует забывать, что программа эта не является ``жесткой'', заранее продуманной во всех деталях. Она совершенствует свою стратегию игры в процессе самообучения. И хотя процесс ``мышления'' у машины существенно отличен оттого, что происходит в мозгу играющего в шашки человека, она способна у него выиграть.

Ярким примером сложной интеллектуальной игры до недавнего времени являлись шахматы. В 1974 г. состоялся международный шахматный турнир машин, снабженных соответствующими программами. Как известно, победу на этом турнире одержала советская машина с шахматной программой ``Каисса''.

Почему здесь употреблено ``до недавнего времени''? Дело в том, что недавние события показали, что несмотря на довольно большую сложность шахмат, и невозможность, в связи с этим произвести полный перебор ходов, возможность перебора их на большую глубину, чем обычно, очень увеличивает шансы на победу. К примеру, по сообщениям в печати, компьютер фирмы IBM, победивший Каспарова, имел 256 процессоров, каждый из которых имел 4 Гб дисковой памяти и 128 Мб оперативной. Весь этот комплекс мог просчитывать более 100 000 000 ходов в секунду. До недавнего времени редкостью был компьютер, могущий делать такое количество целочисленных операций в секунду, а здесь мы говорим о ходах, которые должны быть сгенерированы и для которых просчитаны оценочные функции. Хотя с другой стороны, этот пример говорит о могуществе и универсальности переборных алгоритмов.

В настоящее время существуют и успешно применяются программы, позволяющие машинам играть в деловые или военные игры, имеющие большое прикладное значение. Здесь также чрезвычайно важно придать программам присущие человеку способность к обучению и адаптации. Одной из наиболее интересных интеллектуальных задач, также имеющей огромное прикладное значение, является задача обучения распознавания образов и ситуаций. Решением ее занимались и продолжают заниматься представители различных наук — физиологи, психологи, математики, инженеры. Такой интерес к задаче стимулировался фантастическими перспективами широкого практического использования результатов теоретических исследований: читающие автоматы, системы ИИ, ставящие медицинские диагнозы, проводящие криминалистическую экспертизу и т. п., а также роботы, способные распознавать и анализировать сложные сенсорные ситуации.

В 1957 г. американский физиолог Ф. Розенблатт предложил модель зрительного восприятия и распознавания — перцептрон. Появление машины, способной обучаться понятиям и распознавать предъявляемые объекты, оказалось чрезвычайно интересным не только физиологам, но и представителям других областей знания и породило большой поток теоретических и экспериментальных исследований.

Перцептрон или любая программа, имитирующая процесс распознавания, работают в двух режимах: в режиме обучения и в режиме распознавания. В режиме обучения некто (человек, машина, робот или природа), играющий роль учителя, предъявляет машине объекты и о каждом их них сообщает, к какому понятию (классу) он принадлежит. По этим данным строится решающее правило, являющееся, по существу, формальным описанием понятий. В режиме распознавания машине предъявляются новые объекты (вообще говоря, отличные от ранее предъявленных), и она должна их классифицировать, по возможности, правильно.

Проблема обучения распознаванию тесно связана с другой интеллектуальной задачей — проблемой перевода с одного языка на другой, а также обучения машины языку. При достаточно формальной обработке и классификации основных грамматических правил и приемов пользования словарем можно создать вполне удовлетворительный алгоритм для перевода, скажем научного или делового текста. Для некоторых языков такие системы были созданы еще в конце 60-г. Однако для того, чтобы связно перевести достаточно большой разговорный текст, необходимо понимать его смысл. Работы над такими программами ведутся уже давно, но до полного успеха еще далеко. Имеются также программы, обеспечивающие диалог между человеком и машиной на урезанном естественном языке.

Что же касается моделирования логического мышления, то хорошей модельной задачей здесь может служить задача автоматизации доказательства теорем. Начиная с 1960 г., был разработан ряд программ, способных находить доказательства теорем в исчислении предикатов первого порядка. Эти программы обладают, по словам американского специалиста в области ИИ Дж. Маккатти, ``здравым смыслом'', т. е. способностью делать дедуктивные заключения.

В программе К. Грина и др., реализующей вопросно-ответную систему, знания записываются на языке логики предикатов в виде набора аксиом, а вопросы, задаваемые машине, формулируются как подлежащие доказательству теоремы. Большой интерес представляет ``интеллектуальная'' программа американского математика Хао Ванга. Эта программа за 3 минуты работы IBM-704 вывела 220 относительно простых лемм и теорем из фундаментальной математической монографии, а затем за 8.5 минуты выдала доказательства еще 130 более сложных теорем, часть их которых еще не была выведена математиками. Правда, до сих пор ни одна программа не вывела и не доказала ни одной теоремы, которая бы, что называется ``позарез'' была бы нужна математикам и была бы принципиально новой.

Очень большим направлением систем ИИ является роботехника. В чем основное отличие интеллекта робота от интеллекта универсальных вычислительных машин?

Для ответа на этот вопрос уместно вспомнить принадлежащее великому русскому физиологу И. М. Сеченову высказывание: ``… все бесконечное разнообразие внешних проявлений мозговой деятельности сводится окончательно лишь к одному явлению — мышечному движению''. Другими словами, вся интеллектуальная деятельность человека направлена в конечном счете на активное взаимодействие с внешним миром посредством движений. Точно так же элементы интеллекта робота служат прежде всего для организации его целенаправленных движений. В то же время основное назначение чисто компьютерных систем ИИ состоит в решении интеллектуальных задач, носящих абстрактный или вспомогательный характер, которые обычно не связаны ни с восприятием окружающей среды с помощью искусственных органов чувств, ни с организацией движений исполнительных механизмов.

Первых роботов трудно назвать интеллектуальными. Только в 60-х годах появились очуствленные роботы, которые управлялись универсальными компьютерами. К примеру в 1969 г. в Электротехнической лаборатории (Япония) началась разработка проекта ``промышленный интеллектуальный робот''. Цель этой разработки — создание очуствленного манипуляционного робота с элементами искусственного интеллекта для выполнения сборочно-монтажных работ с визуальным контролем.

Манипулятор робота имеет шесть степеней свободы и управляется мини-ЭВМ NEAC-3100 (объем оперативной памяти 32000 слов, объем внешней памяти на магнитных дисках 273000 слов), формирующей требуемое программное движение, которое отрабатывается следящей электрогидравлической системой. Схват манипулятора оснащен тактильными датчиками.

В качестве системы зрительного восприятия используются две телевизионные камеры, снабженные красно-зелено-синими фильтрами для распознавания цвета предметов. Поле зрения телевизионной камеры разбито на $64 \times 64$ ячеек. В результате обработки полученной информации грубо определяется область, занимаемая интересующим робота предметом. Далее, с целью детального изучения этого предмета выявленная область вновь делится на 4096 ячеек. В том случае, когда предмет не помещается в выбранное ``окошко'', оно автоматически перемещается, подобно тому, как человек скользит взглядом по предмету. Робот Электротехнической лаборатории был способен распознавать простые предметы, ограниченные плоскостями и цилиндрическими поверхностями при специальном освещении. Стоимость данного экспериментального образца составляла примерно 400000 долларов.

Постепенно характеристики роботов монотонно улучшались, Но до сих пор они еще далеки по понятливости от человека, хотя некоторые операции уже выполняют на уровне лучших жонглеров. К примеру удерживают на лезвии ножа шарик от настольного тенниса.

Также можно рассмотреть созданный еще в 70-х годах макет транспортного автономного интегрального робота (ТАИР). Конструктивно ТАИР представляет собой трехколесное шасси, на котором смонтирована сенсорная система и блок управления. Сенсорная система включает в себя следующие средства очуствления: оптический дальномер, навигационная система с двумя радиомаяками и компасом, контактные датчики, датчики углов наклона тележки, таймер и др. И особенность, которая отличает ТАИР от многих других систем, созданных у нас и за рубежом, это то, что в его составе нет компьютера в том виде, к которому мы привыкли. Основу системы управления составляет бортовая нейроподобная сеть, на которой реализуются различные алгоритмы обработки сенсорной информации, планирования поведения и управления движением робота.

В конце данного очень краткого обзора рассмотрим примеры крупномасштабных экспертных систем.

\begin{enumerate}
\item{MYICIN --- экспертная система для медицинской диагностики. Разработана группой по инфекционным заболеваниям Стенфордского университета. Ставит соответствующий диагноз, исходя из представленных ей симптомов, и рекомендует курс медикаментозного лечения любой из диагностированных инфекций. База данных состоит из 450 правил. На практике никогда не использовалась. Причиной этому были как недостаточное развитие компьютерной техники, так и морально-этические проблемы: кто будет ответственен в случае сбоя системы и/или ошибки в лечении?}

\item{PUFF --- анализ нарушения дыхания. Данная система представляет собой MYCIN, из которой удалили данные по инфекциям и вставили данные о легочных заболеваниях.}

\item{DENDRAL --- распознавание химических структур. Данная система старейшая, из имеющих звание экспертных. Первые версии данной системы появились еще в 1965 году во все том же Стенфордском университете. Пользователь дает системе DENDRAL некоторую информацию о веществе, а также данные спектрометрии (инфракрасной, ядерного магнитного резонанса и масс-спектрометрии), и та в свою очередь выдает диагноз в виде соответствующей химической структуры.}

\item{PROSPECTOR --- экспертная система, созданная для содействия поиску коммерчески оправданных месторождений полезных ископаемых.}

\item{Wolfram|Alpha --- база знаний и набор вычислительных алгоритмов, весьма популярное в среде студентов-математиков.}

\end{enumerate}
