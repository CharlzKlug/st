\Csection{Алгоритм программы}

Главная функция, с которой начинается исполнение программы --- это функция \verb|(main)|.

В ней осуществляется ввод начальных данных с клавиатуры. Запрашивается количество принтеров доступных для покупки. После чего вводятся следующие данные по каждому отдельно взятому принтеру: цена принтера, стоимость картриджа, количество распечатываемых листов на одном картридже и критерий энергопотребления (чем значение критерия, тем меньше потребляет принтер). После заполнения сведений о принтерах вводится бюджет на закупку оргтехники.

После этого данные передаются в функцию \verb|knapsack-bicrit|. В этой функции реализована рекурсивная схема работы: программа вызывает саму себя для того, чтобы получить сведения о:

\begin{enumerate}
\item{Возможных решениях при том же уровне бюджета, но на единицу уменьшенным количеством принтеров.}
\item{Если цена последнего введённого принтера меньше, либо равна бюджету, то ищется решение для бюджета уменьшенного на стоимость последнего принтера и уменьшенного на единицу числа принтеров.}
\end{enumerate}

Из полученных двух решений выбирается максимальное решение.

\paragraph*{Пример работы программы:}

\begin{verbatim}
> (main)
Введите количество принтеров: 5
Введите цену принтера N 0: 3
Введите цену картриджа принтера N 0: 2
Введите количество печатаемых листов принтера N 0: 10
Введите критерий энергопотребления принтера N 0: 4
Введите цену принтера N 1: 2
Введите цену картриджа принтера N 1: 1
Введите количество печатаемых листов принтера N 1: 10
Введите критерий энергопотребления принтера N 1: 3
Введите цену принтера N 2: 4
Введите цену картриджа принтера N 2: 4
Введите количество печатаемых листов принтера N 2: 10
Введите критерий энергопотребления принтера N 2: 5
Введите цену принтера N 3: 2
Введите цену картриджа принтера N 3: 3
Введите количество печатаемых листов принтера N 3: 10
Введите критерий энергопотребления принтера N 3: 4
Введите цену принтера N 4: 1
Введите цену картриджа принтера N 4: 5
Введите количество печатаемых листов принтера N 4: 10
Введите критерий энергопотребления принтера N 4: 3
Введите бюджет на закупку: 8
'((2 9/10 14 4 3 1 0))
\end{verbatim}

Здесь оценка по первому критерию оказалась равной $2 \frac{9}{10}$, по второму критерию: $14$, а набор принтеров: $\{5, 4, 2, 1\}$ (при отсчёте от единицы).
