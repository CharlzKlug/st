\section{Философские взгляды на интеллект}

В обычном житейском представлении разумное существо --- это существо воспринимающее, мыслящее, обучающееся, обладающее желаниями и эмоциями, делающее свободный выбор и демонстрирующее целесообразное поведение. Философские и научные теории разума пытаются понять природу этой психической деятельности, природу Я или обладающего сознанием субъекта.

Люди всегда пытались понять то, что их окружает. Все явления, окружающие человека объяснялись духами, населяющими эти явления. Дождь шёл в ответ на ритуалы о дожде. Громы и молнии, штормы и землетрясения свидетельствовали о недовольстве духов. Их пытались задобрить, добиться их расположенияи снисхождения.

Разум, по Демокриту, --- не что иное, как особо тонкая организация специфических атомов, взаимодействующих друг с другом и с окружающей средой, производя тем самым внутреннюю и внешнюю деятельность сознательного существа. Разум образуется при рождении и распадается после смерти, чтобы никогда больше не возродиться.

Платон придерживался совершенно другого взгляда. Разум отличается от физических вещей и существует независимо от тела --- как до его рождения, так и после его смерти. Разум взаимодействует не только с вещами физического, чувственного мира, но также с абстрактными предметами во втором, полностью нефизическом мире, умопостигаемом мире. Разум может разобраться в том, что происходит в физическом мире, только в силу того, что ему доступен, через способность понимания, мир чистых понятий. Он использует понятия второго мира, чтобы постигать отдельные вещи, обнаруживаемые в первом мире.

По Аристотелю, разум --- не какая-то отдельная и отличная от физического тела вещь, но скорее «форма» тела, в самом общем смысле термина «форма», включающем все свойства и деятельности тела. Тело, обладающее разумом, --- это не соединение двух различных вещей, а единство, тело определенного рода, имеющее ментальные свойства.

В эпоху Возрождения Рене Декарт предложил новую, более радикальную форму двойственности тела и разума. С его точки зрения, разум не имеет пространственного протяжения, которое является самой сущностью материи. К этой новой идее он прибавил еще одну: разум способен к ясному и отчетливому постижению, совершенному пониманию собственной природы. Сущность разума, по мнению Декарта, заключается в деятельности мышления, способности к рассуждению. Но у Декарта возникли трудности с объяснением взаимодействия нематериального разума и материального тела. Идея о воздействии нефизического разума на механический мозг также нарушала законы сохранения импульса и кинетической энергии. Декарт знал о некоторых из этих проблем, но не смог их разрешить.

С точки зрения Жюльена Офре де Ламетри, французского врача и философа-материалиста, умственная деятельность, отличающая человека, также является чисто физическим явлением. Ламетри в 1748 году анонимно опубликовал сочинение ``Человек-машина'' где выдвигал теорию, по которой утверждалось что в основании всей жизненной и ментальной деятельности, в том числе декартовского рационального мышления, нет ничего, кроме сложной организации материи. Но эта теория не давала какого-либо конкретного объяснения познавательной деятельности.
