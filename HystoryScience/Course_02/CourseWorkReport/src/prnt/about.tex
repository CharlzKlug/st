\Csection{Введение}

Мы живём в век продолжающегося бурного развития компьютеров и компьютерной техники. Появление компьютеров и глобальной сети интернет упростило и ускорило поиск информации. Если раньше в поисках нужной информации приходилось обращаться в библиотеки, выписывать журналы, искать книги, изучать архивы, то теперь можно буквально за несколько минут отыскать нужную информацию в компьютерной сети. Причём информация в глобальной сети представлена практически на всех языках и со всех стран.

Помимо информации глобальная сеть дала возможность выполнять работу удалённо, находясь в любом месте, где есть доступ к интернету. Всё больше развивается торговля через интернет. Сейчас трудно найти какую-либо компанию или фирму, у которой нет своей страницы в интернете. Развивается сфера оказания услуг через интернет. Причём услуги не только частного рода, но и государственные услуги: записаться на приём к врачу, оформить замену паспорта, оплатить коммунальные услуги и многое другое. Интернет становится неотъемлемой частью нашей жизни.

Казалось бы, с приходом интернета человек мог бы отказаться от бумаги и полностью перейти к использованию электронных форм документооборота. Отчасти это так, но лишь отчасти. Полностью отказаться от бумажного документа не получается. И если для общения нам сейчас вполне достаточно сервисов интернета: электронной почты, служб мгновенной доставки сообщений, звуковых и видеозвонков, то в сфере бюрократической (в хорошем смысле этого слова) жизни без бумажных документов зачастую не обойтись. Примеры: довереннсти, накладные, реестры, домовые книги и многое другое. Как следствие, у нас остаётся потребность в печатной технике: принтерах, копирах и сканерах.

На рынке оргтехники представлены самые разнообразные марки, модели и модификации множества различных принтеров. И у каждой модели есть свои нюансы: во-первых, цена --- неотъемлемая часть товарно-денежных отношений, во-вторых, цена расходников, в-третьих, количество листов печатаемых на одной заправке тонера (порошок для печати). Тут же возникает вопрос: как выбрать несколько принтеров таким образом, что бы, с одной стороны, затратить как можно меньше средств на покупку, а с другой стороны уменьшить расходы на печать каждого листа бумаги?

С учётом того, что потребности в принтерах есть практически в каждой организации, то эта задача весьма актуальна. Осуществив оптимальную закупку принтеров можно добиться снижения расходов на печать, что в свою очередь положительно скажется на бюджете организации. Помимо покупки принтеров полученные результаты можно адаптировать и к другим областям. Пример: закупка станков для завода. У каждого станка есть своя цена, помимо цены есть и расходные материалы: шлифовальные ленты для шлифовальных станков, фрезы для фрезерных станков и другие.

\pagebreak

