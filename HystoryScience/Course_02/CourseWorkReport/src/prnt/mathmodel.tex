\Csection{Постановка задачи}

Задан некоторый набор из $n$ принтеров, условно обозначим этот набор через $A = \{a_1, a_2, ..., a_n\}$ у каждого принтера есть своя цена $C=\{c_1, c_2, ..., c_n\}$. Причём функция отображающая цену принтера $C(x_i)= c_i$ взаимно однозначна. То есть, каждый конкретный принтер $x_i$ однозначно и единственным образом привязан к $c_i$. У каждого принтера есть своя цена картриджа (устройства, содержащего краску для печати) $K=\{k_1, k_2, ...,k_n\}$, цена картриджа также взаимно и однозначно соотносится с множеством $X$. Ресурс листов печатаемых каждым принтером на новом картридже обозначим через множество $R=\{r_1, r_2, ..., r_n\}$, оно также взаимно однозначно связано с множеством $X$. Общую денежную сумму выделенную на приобретение принтеров обозначим через $W$. Осуществить покупку принтеров свыше этой суммы невозможно. Задача заключается в том, чтобы приобрести на сумму $W$ такой набор принтеров, чтобы минимизировать расходы краски на распечатку.

Введём удельную стоимость печати одного листа $U=\{u_1, u_2, ..., u_n\}$, где $$u_i = \frac {k_i} {r_i}$$

То есть, мы разделили стоимость одного картриджа на количество распечатаемых страниц.

Тогда можно ввести условную ``стоимость'' $P={p_1, p_2, ..., p_n}$ для каждого принтера, причём наименее ценным принтером будет принтер с максимальной удельной стоимостью, а наиболее ценным --- принтер с наименьшей удельной стоимостью. Стоимость принтера с максимальной удельной стоимостью обозначим как 1, следующий за ним принтер обозначим как 2 и так далее. И здесь мы приходим к задаче о ранце.

\subsection*{Простейшая задача о ранце}

Простейшая задача о ранце (англ. \emph{Knapsack problem}) --- дано $N$ предметов, $n_i$ предмет имеет массу $c_i>0$ и стоимость $p_i>0$. Необходимо выбрать из этих предметов такой набор, чтобы суммарная масса не превосходила заданной величины $W$ (вместимость рюкзака), а суммарная стоимость была максимальна.

Добавим ещё одно множество $X=\{x_1, x_2, ..., x_n\}$ определяющее является ли предмет $i$ выбранным или нет, причём

$$x_i = \left\{
\begin{aligned}
1, \text{если предмет $i$ включается в набор} \\
0, \text{если предмет $i$ не включается в набор}
\end{aligned}
\right.
$$

Тогда математически задачу о ранце можно записать так:

$$
\left\{
\begin{aligned}
c_1 x_1 + c_2 x_2 + ... + c_n x_n \to \max \\
p_1 x_1 + p_2 x_2 + ... + p_n x_n \le W \\
x_i \in \{0, 1\}, i = \overline{1, n}
\end{aligned}
\right.
$$

\subsection*{Метод динамического программирования}

Пусть $Q(k,s)$ --- максимальная стоимость предметов, которые можно уложить в ранец вместимости $s$, если можно использовать только первые $k$ предметов, то есть $\{a_1, a_2, ..., a_k\}$, назовём этот набор допустимыми предметами для $Q(k,s)$. Причём,

$$Q(k, 0) = 0$$
$$Q(0, s) = 0$$

Найдём $Q(k, s)$. Здесь возможны два варианта:

\begin{enumerate}
\item{Если предмет $k$ не попал в ранец. Тогда $Q(k,s)$ равно максимальной стоимости ранца с такой же вместимостью и набором допустимых предметов $\{a_1, a_2, ..., a_{k-1}\}$, то есть, $Q(k,s) = Q(k-1,s)$.}
\item{Если предмет $k$ попал в ранец. Тогда $Q(k, s)$ равно максимальной стоимости ранца, где вес $s$ уменьшаем на вес $k$-ого предмета и набор допустимых предметов $\{a_1, a_2, ..., a_{k-1}\}$ плюс стоимость $k$, то есть $Q(k-1, s-p_k) + c_k$.}
\end{enumerate}

То есть: $Q(k, s) = \max{\{Q(k-1, s), Q(k-1, s- p_k) + c_k\}}$

Стоимость искомого набора равна $Q(n, W)$, так как нужно найти максимальную стоимость рюкзака, где все предметы допустимы и вместимость рюкзака $W$.

\subsection*{Определение вхождения предмета в набор}

Будем определять, входит ли $a_i$ предмет в искомый набор. Начинаем с элемента $Q(i, w)$, где $i = n$, $w = W$. Для этого сравниваем $Q(i, w)$ со следующими значениями:

\begin{enumerate}
\item{Максимальная стоимость ранца с такой же вместимостью и набором допустимых предметов $\{a_1, a_2, ..., a_{i-1}\}$, то есть $Q(i-1, w)$.}
\item{Максимальная стоимость ранца с вместимостью на $w_i$ меньше и набором допустимых предметов $\{a_1, a_2, ..., a_{i-1}\}$ плюс стоимость $c_i$, то есть $Q(i-1, w - w_i) + c_i$.}
\end{enumerate}

Максимальное значение записывается в $Q(i, w)$. Если решение записать в виде таблицы, где по столбцам записана вместимость ранца, а по строкам записаны предметы, доступные для размещения в рюкзаке, то можно записать формулу заполнения $i+1$ строки по $i$-й строке:

$$
Q(i+1, p) = \left\{
\begin{aligned}
Q(i,p), \text{если $p_{i+1}>w$} \\
\max{\{Q(i, p); c_{i+1} + Q(i, w - p_{i+1})\}}
\end{aligned}
\right.
$$

\paragraph*{Пример:}
Решим следующую задачу:

$$
\left\{
\begin{aligned}
2 x_0 + x_1 + 3 x_2 + 4 x_3 \to max \\
5 x_0 + 3 x_1 + 4 x_2 + 2 x_3 \le 12
\end{aligned}
\right.
$$

Решение будет в виде таблицы:

\begin{tabular}{|c|c|c|c|c|c|c|c|c|c|c|c|c|}
\hline
& \textbf{1} & \textbf{2} & \textbf{3} & \textbf{4} & \textbf{5} & \textbf{6} & \textbf{7} & \textbf{8} & \textbf{9} & \textbf{10} & \textbf{11} & \textbf{12} \\ \hline
\textbf{0} & 0 & 0 & 0 & 0 & $2_{0}$ & $2_{0}$ & $2_{0}$ & $2_{0}$ & $2_{0}$ & $2_{0}$ & $2_{0}$ & $2_{0}$ \\ \hline
\textbf{1} & 0 & 0 & $1_{1}$ & $1_{1}$ & $2_{0}$ & $2_{0}$ & $2_{0}$ & $3_{1,0}$ & $3_{1,0}$ & $3_{1,0}$ & $3_{1,0}$ & $3_{1,0}$ \\ \hline
\textbf{2} & 0 & 0 & $1_{1}$ & $3_2$ & $3_2$ & $3_2$ & $4_{2,1}$ & $4_{2,1}$ & $5_{2,0}$ & $5_{2,0}$ & $5_{2,0}$ & $6_{2,1,0}$ \\ \hline
\textbf{3} & 0 & $4_3$ & $4_3$ & $4_3$ & $5_{3,1}$ & $7_{3,2}$ & $7_{3,2}$ & $7_{3,2}$ & $8_{3,2,1}$ & $8_{3,2,1}$ & $9_{3,2,0}$ & $9_{3,2,0}$ \\ \hline
\end{tabular}

Ответом для этой задачи будет $Q(3,12) = 9_{3,2,0}$, то есть, максимально возможная стоимость составляет 9 и достигается выбором третьего, второго и нулевого предметов ($4 + 3 + 2 = 9$). При этом общий вес предметов составляет $2 + 4 + 5 = 11$.

\subsection*{Бикритериальная задача о ранце}

Зачастую случается так, что помимо одного критерия приходится учитывать ещё и другие критерии, например: кроме удельной стоимости печати одного листа у принтеров есть ещё и энергопотребление или выделение озона при печати. Ясно, что чем меньше энергопотребление, тем меньше накладные расходы или чем меньше озона выделяется при печати, тем меньше вреда здоровью пользователей этого принтера.

Энергопотребление для нашей задачи в математическом виде можно описать аналогично удельной стоимости, то есть, принтеру с максимальным энергопотреблением мы присвоим ``стоимость'' в 1, следующему за этим принтером --- 2 и так далее, принтеру с минимальным энергопотреблением в результате будет присвоена максимальная стоимость --- $n$. Математически новую задачу можно описать так:

$$
\left\{
\begin{aligned}
c_1^1 x_1^1 + c_2^1 x_2 + ... + c_n^1 x_n \to \max \\
c_1^2 x_1 + c_2^2 x_2 + ... + c_n^2 x_n \to \max \\
w_1 x_1 + w_2 x_1 + ... + w_n x_n \le W
\end{aligned}
\right.
$$

где $\{c_1^1, c_2^1, ..., c_n^1\}$ --- показатели по первому критерию (в нашем случае это удельная стоимость печати одного листа), $\{c_1^2, c_2^2, ..., c_n^2\}$ --- показатели по второму критерию (энергопотребление) и $\{w_1, w_2, ..., w_n\}$ --- стоимость принтера.

Теперь для каждого решения вводится совокупность эффективных оценок $E(Q)$, где $Q$ --- какое-либо решение. Пример: $Q(k, p)$ --- это решение, в котором можно использовать предметы от 1 до $k$, $p$ --- максимальный допустимый вес ($p = 1, ..., W$). Отсюда система эффективных оценок для $E(Q) = E(k, p)$. Если же $k = n$, а $p = W$, то мы получим эффективную оценку для всей задачи.

Система эффективных оценок строится на основе рекуррентного соотношения и начинается с $E(1, p)$ (1-й предмет, $p$ --- максимальный вес).

$$
E(1, p) = \left\{
\begin{aligned}
(0, 0), \text{если $w_1 > p$} \\
(C_1^1, C_1^2), \text{если $w_1 \le p$}
\end{aligned}
\right.
$$

так заполняется первая строка. После заполнения $k$-й строки переходим к последней строке:

$$
E(k+1, p) = \left\{
\begin{aligned}
E(k, p), \text{если $w_{k+1} > p$} \\
\text{eff}\{E(k,p) \cup [(c_{k+1}^1, c_{k+1}^2) \oplus E(k, p - w_{k+1})]\}
\end{aligned}
\right.
$$

\paragraph*{Пример:}

Бикритериальная задача задана следующим образом:

$$
\left\{
\begin{aligned}
5x_1 + 6 x_2 + 2 x_3 + 3 x_4 + x_5 \to \max \\
4 x_1 + 3 x_2 + 5 x_3 + 4 x_4 + 3 x_5 \to \max \\
3 x_1 + 2 x_2 + 4 x_3 + 2 x_4 + x_5 \le 8
\end{aligned}
\right.
$$

Решим её:

\begin{tabular}{|c|c|c|c|c|c|c|c|c|}
\hline
 & \textbf{1} & \textbf{2} & \textbf{3} & \textbf{4} & \textbf{5} & \textbf{6} & \textbf{7} & \textbf{8} \\ \hline
\textbf{1} & 0 & 0 & $(5, 4)_{1}$ & $(5, 4)_{1}$ & $(5, 4)_{1}$ & $(5, 4)_{1}$ & $(5, 4)_{1}$ & $(5, 4)_{1}$ \\ \hline
\textbf{2} & 0 & $(6, 3)_{2}$ & \specialcell{$(5, 4)_1$ \\ $(6, 3)_2$} & \specialcell{$(5, 4)_1$ \\ $(6,3)_2$} & $(11, 7)_{1,2}$ & $(11, 7)_{1,2}$ & $(11, 7)_{1,2}$ & $(11, 7)_{1,2}$ \\ \hline
\textbf{3} & 0 & $(6,3)_2$ & \specialcell{$(5, 4)_1$ \\ $(6, 3)_2$} & \specialcell{$(5, 4)_1$ \\ $(6, 3)_2$ \\ $(2, 5)_3$} & $(11, 7)_{1,2}$ & \specialcell{$(11,7)_{1,2}$ \\ $(8, 8)_{3,2}$} & \specialcell{$(11,7)_{1,2}$ \\ $(7, 9)_{3,1}$ \\ $(8, 8)_{3,2}$} & \specialcell{$(11,7)_{1,2}$ \\ $(7, 9)_{3,1}$ \\ $(8, 8)_{3,2}$} \\ \hline
\textbf{4} & 0 & \specialcell{$(3,4)_4$ \\ $(6, 3)_2$} & \specialcell{$(5,4)_4$ \\ $(6, 3)_2$} & $(9, 7)_{4,2}$ & \specialcell{$(11, 7)_{1,2}$ \\ $(8, 8)_{1,4}$} & \specialcell{$(8, 8)_{1,4}$ \\ $(5, 9)_{3,4}$ \\ $(11,7)_{1,2}$} & $(14, 11)_{1,2,4}$ & \specialcell{$(14, 11)_{1,2,4}$ \\ $(11, 12)_{3,2,4}$} \\ \hline
\textbf{5} & $(1, 3)_5$ & \specialcell{$(3, 4)_4$ \\ $(6, 3)_2$} & \specialcell{$(4, 7)_{4,5}$ \\ $(7, 6)_{2,5}$} & $(9, 7)_{4, 2}$ & \specialcell{$(11, 7)_{1,2}$ \\ $(10, 10)_{4,2,5}$} & \specialcell{$(12, 10)_{1,2,5}$ \\ $(9, 11)_{5,4,1}$} & \specialcell{$(14, 11)_{1,2,4}$ \\ $(6, 12)_{3,4,5}$}& $(15, 14)_{1,2,4,5}$ \\ \hline
\end{tabular}

Здесь ответом является $(15, 14)$ и достигается этот результат набором из первого, второго, четвёртого и пятого предметов.

\subsection*{Методы компромисса}

Из последней задачи видно, что иногда возможны несколько ответов. В этом случае лицо принимающее решения должно выбрать из всей совокупности решений наиболее подходящее для него решение. Существует много методов поиска компромисса. Вот некоторые из них:

\subsection*{Лексикографическое упорядочение критериев}

Критерии, по которым производится отбор, упорядочиваются по убыванию их важности. Если в процессе решения задачи получился только один ответ, то алгоритм отбора на этом заканчивается. Если же в процессе решения возникло несколько возможных решений, то в этом случае решения упорядочивают по самому важному критерию. Если решение удовлетворяющее самому важному критерию оказалось единственным, то алгоритм заканчивается, если получилось так, что есть несколько решений одинаковых по самому важному критерию, то в этом случае обращают внимание на второй по важности критерий. Если опять осталось несколько решений, то переходят к третьему критерию и так далее.

\subsection*{Метод идеальной точки}

Пусть $D$ --- множество допустимых решений некоторой задачи. Решение $x$ принадлежащее $D$ обладает несколькими критериями: $K_1(x)$, $K_2(x)$, ..., $K_l(x)$. В этом случае можно ввести некоторое $l$ мерное пространство. Тогда вся совокупность решений даёт точки в пространстве формирующие область $T$. Далее в пространстве назначается идеальная точка $J$ не принадлежащая области $T$. В результате остаётся найти ближайшую к $J$ точку.
