\section{Техника и философия}
С давних времён человека сопровождает техника, но не сразу человеческая цивилизация стала технической. В ХХ веке в развитии техники произошёл качественный скачок, из-за этого техника стала предметом изучения философии. Немецкий философ Эрнст Капп и русский инженер Пётр Климентьевич Энгельмейер были первыми, кто начал рассуждать о проблемах техники и общества. С тех же пор сама техника, её развитие, место и роль в цивилизации стали предметами изучения. Не только философы, но и сами инженеры, начали уделять осмыслению техники все большее внимание. Часто такое осмысление сводится к исключительно оптимистической оценке достижений и перспектив ин\-фор\-ма\-цион\-но-тех\-ни\-чес\-ко\-го развития. 

В это же время в гуманитарной среде возросло критическое отношение к такому техническому прогрессу современного общества. В современной философии внимание привлекается, прежде всего, к отрицательным сторонам информационно-технического прогресса.  Технический прогресс породил в ХХ веке большое количество специальных технических дисциплин, которые исследуют различные аспекты техники. В то же самое время техника в целом не является предметом исследования технических дисциплин. Многие естественные науки вынуждены принимать во внимание технику, они делают её предметом специального исследования, конечно, со своей особой естественнонаучной, например, физической точки зрения. Кроме того, из-за проникновения техники практически во все сферы жизни человечества, многие общественные науки обращаются к специальному анализу технического развития. 

Философия техники исследует феномен техники в целом, её место в общественном развитии и уделяет особое внимание вопросам и проблемам, связанным с влиянием техники на человека.  Что же такое техника? Как можно интерпретировать это понятие. Для древних греков ``технэ'' (античное ``технэ'' – это не техника в нашем понимании, а все, что сделано руками: военная техника, игрушки, модели, изделия ремесленников и даже произведения художников) располагается ниже мудрости, и ключом для понимания ``технэ'' является знание общего. Сократ и Аристотель придерживались следующей точки зрения соотношения знаний философов и ремесленников: ``наставники более мудры не благодаря умению действовать, а потому что они обладают отвлеченным знанием и знают причины''.

В средние века техника сохраняет свое вторичное значение и считается результатом божественного творчества. В новое время желание человека господствовать над природой реализуется в технике, она представляется как продолжение науки, выступает как сила человеческого разума и результат его инженерных способностей. Карл Маркс приходит к выводу, что производственные силы, средства производства являются изначальным базисом общества. Средствами производства является техника. Особый интерес представляет понимание техники в ХХ веке, потому что именно в это время техника превратилась в силу, господствующую над человеком. Сейчас мы под техникой понимаем следующее:

\begin{itemize}
\item{совокупность технических устройств (от самых простейших орудий до сложнейших технических систем);}
\item{совокупность различных видов технической деятельности по созданию подобных устройств;}
\item{совокупность технических знаний.}
\end{itemize}

К сфере техники можно отнести не только использование, но и само производство научно-технических знаний, т.е. современная техника неразрывно связана с развитием науки. Техника включается в самостоятельную сферу жизнедеятельности, а именно в техносферу. Под техносферой понимается исторически обусловленная, сознательно формируемая, поддерживаемая и совершенствуемая система отношений между человеком и природой, человеком и техникой, человеком и человеком на основе определенного технического миропонимания. Кроме техносферы философы рассматривают ноосферу. По мнению В.И. Вернадского, ``ноосфера — это гармоническое соединение природы и общества, это торжество разума и гуманизма, это слитая воедино наука, общественное развитие и государственная политика на благо человека, это — мир без оружия, войн и экологических проблем, это — мечта, цель, стоящая перед людьми доброй воли, это — вера в великую миссию науки и человечества, вооруженного наукой''.

Русский философ Владимир Александрович Кутырев утверждает, что ноосфера и техносфера — синонимы, и можно продолжить этот ряд понятиями наукосферы, рациосферы, инфосферы, интеллектосферы. ``И все они, порождаясь природой, ``снимают'' ее, противостоят ей''. В.А. Кутырев видит в ноосфере утопию и рассматривает технику как средство уничтожения всего человеческого. Карл Ясперс при анализе техники сформулировал мысль о том, что человеку необходимо опасаться техники, он может ``потеряться в ней'' и забыть о себе, потому что ``Техника двойственна. Поскольку техника сама не ставит перед собой целей, она находится по ту сторону добра и зла или предшествует им. Она может служить во благо или во зло людям. Она сама по себе нейтральна и противостоит тому и другому. Именно поэтому ее следует направлять''. Исходя из такого понимания, возникает вопрос о том, какое именно содержание придает технике человек? Не готовит ли он себе катастрофу?  Представитель феноменологов Гуссерль считал, что человек придает технике негативное содержание путем перевода богатого жизненного мира человека в научные понятия, на основе которых затем создается техника. В итоге жизненный мир человека теряется, и развивается кризис человека, его науки и техники. Техника -- это обычно бедный знак нашей жизни, его следует наполнить этой жизнью. Поэтому выходом из такого затруднительного положения можно считать процесс, когда наука и техника будут создаваться как полноценные знаки жизненного мира человека.

Интересный подход к пониманию техники предложил русский философ ХХ века Г.П. Щедровицкий. Сущность предлагаемой им философии он назвал мыследеятельностью, которая состоит в следующем: ``сначала нужно выработать правильную мысль (что достигается в процессе проведения многодневных семинаров), а затем разработать, причем непременно, программу действий''. Такой подход позволит подойти более ответственно не только к пониманию природы техники, но и обеспечить более качественный результат во время ее создания.  За последние несколько десятилетий возникло множество технических теорий, которые базируются не только на естествознании.

Такие теории могут быть названы абстрактными техническими теориями (например, системотехника, информатика, теория автоматов и др.), для которых характерно включение в фундаментальные инженерные исследования общей методологии. Особое место среди таких теорий занимает информационная техника и технология, как средство ускорения технического прогресса. Поэтому информационно-технический общество и проблема существования в нем человека становятся объектами пристального внимания, как философов, так и инженеров. Из-за того, что информационная техника и технология в ХХ веке стали бурно развиваться, на рубеже 60-х и 70-х годов возникли футуристические настроения в отношении важности информации в жизни человека, тогда же были сформированы характеристики будущего информационного общества.

Во-первых, переход экономических и социальных функций в информационном обществе от капитала к информации на основании соединения науки, техники и экономики, увеличение информоёмкости производимых продуктов, сопровождающееся ростом доли инноваций, маркетинга и рекламы в их стоимости, высокого уровня автоматизации производства, освобождающего человека от рутинной работы и т.п. Производство информационного продукта, а не продукта материального станет движущей силой образования и развития.  Во-вторых, фактором социальной дифференциации в информационном обществе выступает не собственность, а уровень знаний. В основе этого процесса, по утверждению Д. Белла, лежит рост сферы услуг за счет сферы материального производства, вызывающий, в свою очередь, преобладание в высших социальных эшелонах людей, специализирующихся на выработке систематически организованного знания. Подобный тип профессионального труда неотделим от всевозможных инноваций, что предъявляет повышенные требования к уровню знаний. Общество живет за счет инноваций и социального контроля за изменениями, оно пытается предвидеть будущее и осуществить планирование.

Закономерным следствием этого становится, по мнению Д. Белла, формирование новых социальных элит, основанное на уровне полученного образования.  В-третьих, информационное общество характеризуется симбиозом социальных организаций и информационных технологий. Возможность внедрения новых информационных технологий не только в промышленное производство, но и в социальную сферу определяется, прежде всего, через создание тех ли иных алгоритмов действия – принятия управленческих решений в неопределенной ситуации или в ситуации риска и т.п. Результатом этого должна стать новая рациональность информационного века, — основанная не на классической идее ``общественного договора'' или ``социального согласия'', а рациональность интеллектуальных технологий, позволяющая наконец-то осуществиться весьма почтенной по своему возрасту мечте об упорядочении социальной жизни.  Вышеописанные характеристики информационного мира, предложенные Д. Беллом, несомненно, имеют отношения к современному информационно-техническому человечеству. Это еще раз подчеркивает, насколько сложные и важные процессы протекают в человеческом обществе под влиянием техники, а в последнее время, особенное влияние на человека получила именно информационная техника, непосредственное отношение к которой имеют все специалисты в области вычислительной техники и информатики.

Итак, все взгляды различных философов и философских направлений в отношении техники едины в следующем: во-первых, техника есть знак или образ самого человека, во-вторых, человек не должен в технике забывать самого себя и, в-третьих, ключ к решению проблемы человека в информационно-техническом мире видится только в гармонии техники и человека. Необходимо в первую очередь быть человекам, тогда никакая техника не будет страшна.

Кроме того, в условиях современного информационного общества особую роль играет информационная техника, которая способна существенно изменить не только окружающую человека природу, но и внутренний мир самого человека и его устоявшийся образ жизни, превратить информацию в основной критерий дифференциации людей в обществе и изменить общественные отношения. 
