\section{Проблема ответственности ученых и инженеров}

Значительную долю гуманитарных проблем развития информационно-технического общества составляют проблемы, по существу этические или тесно связанные с таковыми. К области этики относят такие понятия как ``благо'' и ``зло'', ``ответственность'', ``справедливость'', ``свобода''. Этика техники подразумевает рассмотрение аспектов техники сквозь призму этих понятий.  Основным фактором развития информационно-технического общества можно считать научную и инженерную деятельность, которая в современном обществе, как показано выше, играет важную и возрастающую роль. Проблемы повышения эффективности научных исследований и разработок, их практического использования выдвигают сегодня инженерную деятельность как важнейшее условие развития экономики государств и планеты в целом.

Огромное количество технических вузов готовит инженеров различного профиля для разных отраслей хозяйствования. Современное развитие профессионального сознания инженеров предполагает осознание возможностей, границ и сущности своей специальности не только в узком смысле этого слова, но и в смысле осознания инженерной деятельности вообще, её целей и задач, а также изменений её ориентаций и приоритетов в культуре и ценностях человечества. Инженерная деятельность предполагает регулярное применение знаний, полученных в научной деятельности для создания искусственных, технических систем --- сооружений, устройств, механизмов, машин и т.п. Однако современный инженер, как и любой современный ученый, должен прислушиваться к голосу собственной совести, общественному мнению, потому что результаты его работы могут повлиять на здоровье и образ жизни человека, нарушить равновесие в живой природе. Особенно остро стает эта проблема именно сейчас, когда ``рукотворный мир'' человека может сравниться своими масштабами с естественным миром природы. Когда результаты инженерной деятельности становятся глобальными, то проблема влияния этой деятельности на окружающую нас среду и мир в нас, внутри человека, перестает быть узко профессиональной и становится предметом всеобщего обсуждения. Информационные системы, радио, телевидение, компьютеры и Интернет также можно отнести к таким результатам инженерной деятельности.  Проблемы негативных социальных и других последствий техники, проблемы этического самоопределения инженера возникли с самого момента появления инженерной профессии. Леонардо да Винчи, например, был обеспокоен возможным нежелательным характером своего изобретения и не захотел предать гласности идею аппарата подводного плавания, указав, что человек имеет злую природу и может использовать этот аппарат для совершения убийств на дне морском путем потопления судов.

Конечно, подобные решения тормозили технический и экономический прогресс, приходили в противоречие с требованиями рыночной экономической системы. Однако сегодня человечество находится в принципиально новой ситуации, когда невнимание к проблемам последствий внедрения новой техники и технологии может привести к необратимым негативным результатам для всей цивилизации. Кроме того, человечество находится на той стадии научно-технического развития, когда такие последствия необходимо, хотя бы частично, предусмотреть и минимизировать уже на ранних стадиях разработки новой техники.  Ученый или инженер, обладая колоссальными знаниями, может сделать больше, чем он имеет на то право. Поэтому, возникает вопрос о том, каким должен быть человек, прежде всего ученый и инженер, в информационно-техническом мире? Ответ на такой вопрос один: необходимо быть человеком моральным, ответственным!

Ответственность --- это ключ к решению этических проблем техники, который, возможно, позволит ближе подойти к решению проблемы вполне реальной экологической катастрофы, которая может быть результатом технологической деятельности человечества.  Необходимо предусмотреть такие механизмы и структуру научной и инженерно-технической деятельности, чтобы позволить обществу контролировать последствия научно-технических проектов. Контроль и влияние со стороны общества на инженерную и научную деятельность осуществляется по средствам государства, общественных организаций, средств массовой информации, которые, тоже являются результатом технического прогресса.

Можно сделать вывод, что проблема социальной ответственности ученых и инженеров может быть решена путем высокой моральности и ответственности этих людей перед собой и обществом. Кроме того, инженеры и ученые должны понимать важность своей работы, они должны понимать, уважать и придерживаться общечеловеческих ценностей и должны стремиться к гармонии между человеком, техникой и природой. 
