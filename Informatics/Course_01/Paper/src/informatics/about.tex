
\Csection{Введение}

Главная проблема философии -- человек. С точки зрения философии, человек -- это совокупность большого количества различных характеристик. Философия изучает всё, что может касаться человека. Человек мыслит и создаёт разнообразные сущности, окружающие человека и создающие информационно-техническую среду. При этом становится необходимым изучать влияние человека на природу и на самого себя. В этот момент философия начинает изучать информационно-техническую среду, создаваемую человеком. Кроме изучения информационно-технической среды философия поднимает множество различных вопросов, одной из которых является вопрос о последствиях такой деятельности. В наши дни информационно-техническое воздействие человека на окружающую среду обращает на себя особенно большое внимание, поскольку под угрозой существования оказывается сам человек.





В связи с этим исследование философией человека в информационно-технологической среде приобретает особую важность, которая, с учётом бурного развития технологий будет только увеличиваться. Поэтому данной проблеме философия уделяет большое внимание, доказательством этому могут служить публикации затрагивающие эту тему.  Этот реферат посвящён изучению влияния как человека на информационно-техническую среду, так и влияние информационно-технической среды на человека, рассмотреть процессы, протекающие при этом воздействии и показать ответственность ложащуюся на человека. Поэтому задачи реферата можно условно разделить таким образом:

\begin{itemize}

\item{какой смысл вкладывается в понятие техники в философии;}

\item{рассмотреть современные проблемы информационно-технического мира и рассмотреть возможные способы их преодоления;}
  
\item{рассмотреть этические проблемы ответственности человека.}
\end{itemize}

При составлении реферата использовалась литература, посвящённая философским вопросам природы человека, информационных технологий, развитию техники и последствиям, к которым может привести это развитие.


\pagebreak

