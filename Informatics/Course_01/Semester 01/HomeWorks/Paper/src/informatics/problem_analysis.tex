\section{Анализ проблем информационно-технического мира}

Техника развивалась в истории человечества не равномерно. На разных континентах техническая власть человека над природой формировалась по-разному и состояла, как правило, в умении человека делать простые устройства, упрощающие ему жизнь. Различия в технике у разных народов обусловлены множеством факторов: условия существования, доступность природных ресурсов, культура народа и так далее.  Принято считать, что бурное развитие техника получила в Западной Европе с возникновением капиталистов, которые поставили главной целью получение прибыли. Именно в этот период формируется тесная взаимосвязь техники с наукой, с конца ХVI века и на протяжении четырех последующих столетий инженеры и технические специалисты постепенно привыкли к мысли, что технический прогресс невозможен без науки, без фундаментального роста способности понимать природу, без выхода за пределы простых навыков и умений, какими бы сложными они ни были.

В результате наука стала двигателем техники и капиталистическая Европа не только преуспевала в овладении природой, но и училась преобразовывать ее. Теперь техника тесно связана с наукой, поэтому понимание природы техники невозможно без осмысления специфики науки. Когда мы пытаемся понять, в чем же состоит воздействие науки и техники на жизнь людей, мы должны одновременно идти по двум линиям. Первая связана с рассуждениями о том, является ли техника благом или злом для человечества? Вторая приводит к выводу: наука и техника несет в себе и жизнь и смерть. Мы также должны признать историчность наших отношений к воздействию науки и техники на жизнь общества и человека, и их взаимосвязь с нашей собственной культурой: эти отношения будут различными в зависимости от того, в чьих руках находится данная техника, на какой стадии технического развития производится та или иная оценка, о какой части человечества идет речь или какую часть человечества представляют эти оценки, к какому классу, расе, нации, поколению, полу принадлежит тот, кто делает эти оценки, каков уровень его культуры.

И в еще большей степени отношения к технике зависят от того, о каком именно техническом развитии идет речь.  Сейчас с уверенностью можно утверждать, что, в конечном счете, само существование человеческого рода будет зависеть от решений, связанных с научной технологией. Так мы испытываем на самих себе диалектику особенного и всеобщего из-за того сложного положения, в котором мы очутились, из-за объединенного воздействия технического прогресса прошедших двух столетий и политико-экономического господства рыночных отношений в тот же период, из-за подчинения природы и общества промышленному техническому капитализму.

Человечество было и продолжает быть охваченным процессом возникновения массового общества, процессом, который был бы невозможен без развития техники: это и тесно связанная с техническим прогрессом массовая безработица, сопровождаемая разрушением ремесел и распадением традиционных общественных связей, это и массовая культура, распространяемая средствами массовой информации, как печатными, так и электронными. В последнем случае происходит утрата человеком своей индивидуальности.

Огромное воздействие на науку и технику оказали войны, происходившие в последние два столетия. Войны нас, по-видимому, ничему не учат, и почти с уверенностью, можно сказать, что человечество находится в постоянном состоянии войны. Такая война стала новым, как техническим, так и политическим явлением, новым, ибо в такой войне сражение охватывает уже не только тех, кто участвует в нем, следуя патриотическому долгу или воинской обязанности, но и безоружное гражданское население, которое, однако, рассматривается как фактор экономического и военного потенциала противоборствующих сторон. Война всегда приносила человечеству бедствие, а науке и технике давала новый импульс в развитии.

Разве мировые войны, разразившиеся в ХХ веке, в которых применялись триумфальные достижения науки и техники, принесли человечеству что-либо иное, кроме несоизмеримой ни с чем беды? Техника не только служит войне, она часто даже вызывает, провоцирует войну. Война с техникой нарастает и ширится. Все великие воинственные народы были мастерами и изобретателями в области орудий, инженерных работ, тактики и т.п. Война родит и питает технику, техника питает войну. Техника организует природу для человека, но разве не грозит она поработить самого человека?

Итак, техника является, в некоторой степени, толчком многих социальных явлений. Рассмотрим, например, массовую культуру. На первый взгляд, заметно ее проникновение в обыденное сознание повсюду, от деревень до столиц. Демократичность и доступность школьного обучения, всеобщая грамотность, колоссальные тиражи газет и журналов, потоком сходящих со скоростных печатных машин, дешевые и неплохо выполненные цветные репродукции произведений живописи и высококачественные записи музыкальных произведений – все это, несомненно, можно считать положительным результатом достижений информационно-технического мира. Но при более близком рассмотрении мы обращаем внимание на обратную сторону, на негативные последствия внедрения в эту сферу новой информационной техники, такой, например, как телевидение и Интернет, способной настолько глубоко изменять массовое сознание, что можно говорить о переходе всеобщей грамотности в свою противоположность и личностную невосприимчивость к написанному слову – это также результат существования человека в современном информационно-техническом мире. Кроме явных плюсов в виде доступности любой информации за кратчайшее время проявляются и минусы: мировая сеть Интернет ещё и превращается в театр ведения информационных войн. С возникновения человечества люди начали обмениваться информацией не только в целях передачи действительной, правдивой информации, но и с целью дезинформирования или искажения информации в угоду своим целям. Теперь же с развитием Интернета потенциальными целями информационных войн становятся умы практически всех людей, имеющих подключение к Интернету. Не надо далеко идти за примерами. Из недавних событий ярким примером может служить печально известное, так называемое, ``Исламское государство'' развернувшее в сети Интернет широкомасштабную кампанию собственной пропаганды. Причём пропаганда очень качественная и направленная на широкие круги населения: от Азии и до Запада. Причём террористическое ``Исламское государство'' не гнушается никакими методами: от периодических терактов, освещаемых различными средствами массовой информации, до видео роликов, насыщенных спецэффектами, по качеству сравнимыми с продукцией Голливуда, срежиссированными, снятыми и обработанными профессионалами своего дела. И всё это происходит через Интернет. Примером жертвы такой пропаганды может служить студентка Московского государственного университета имени М.В. Ломоносова Варвара Караулова попавшая под влияние пропаганды, выехавшая в Сирию, но благодаря бдительности родителей возвращённая в Россию. Это только наиболее известный пример, а сколько молодёжи всё же попали в лапы ``Исламского государства''? Нам это неизвестно. Отсюда можно сделать вывод что информацию и технику можно использовать как во благо, так и во вред.

Человек в современном информационно-техническом мире, можно сказать, радикально отличается от прошлого. Определенный количественный рост достиг критической точки, за которой, как принято говорить, количество переходит в качество, рост вступает в некоторую новую фазу. Разрушительная мощь ядерных бомб, в буквальном смысле сверхчеловеческие возможности современной информационной техники, креативные и преобразующие возможности биохимической генной инженерии, позволяющей человеку ``изобретать'' новые ``природные'' биологические виды, космическая инженерия, эффективные методы контроля над рождаемостью — все это свидетельствует о том, что человечество достигло нового уровня своего технического потенциала. Но эти технические достижения и новшества открывают собой и новую стадию социального воздействия по сравнению с предыдущим техническим состоянием человечества. Отсюда вытекает важная характеристика нашего времени -- это всемирный характер социальных и технических проблем, которые формируют недостатки и пороки современного информационно-технического мира. К таким недостаткам можно отнести:

\begin{itemize}

\item{политические и экономические препятствия к тому, чтобы техника использовалась для ликвидации нищеты;}
  
\item{неспособность социальных наук и исследований современных общественных изменений, равно как и методологии общественных дисциплин, решать свои главные практические и теоретические задачи;}
  
\item{недостатки образования и воспитания во всем мире, препятствующие решению указанных проблем, мешающие здоровому, творческому пониманию науки и техники как составной части гуманистического воспитания в эпоху ин\-фор\-ма\-цион\-но-тех\-ни\-чес\-ко\-го прогресса; это относится и к подготовке специалистов, и к общему образованию большинства людей, к тому же подготовка специалистов страдает культивируемым элитизмом;}
  
\item{неспособность научной и технической элиты преодолеть свою национальную ограниченность, элитаристское сознание, если не считать нескольких исключений, например, таких как Всемирная организация здравоохранения; в особенности это касается неспособности противодействовать идеологическим наслоениям в науке.}
\end{itemize}

В связи с перечисленными проблемами стоит упомянуть некоторые современные научные и информационно-технические достижения, которые оказывают наиболее серьезное воздействие на человека. Ещё десятилетия назад были очерчены многие потенциально опасные достижения, сегодня этот вопрос только обострился. Ядерные испытания в военных целях. Новейшие достижения ядерной техники остаются, прежде всего, военным фактором, возможным скачком к еще худшим бедствиям, но в сознании людей все это утрачивает новизну, перестает быть ужасающим. Но в действительности дело обстоит еще хуже, потому что ядерное вооружение не сокращается, а напротив, оно все более распространяется и становится все более грозным. Всему этому пока не видно конца.

Кибернетика (гениальное изобретение Норберта Винера) как наука о разумных машинах еще не исчерпала своих возможностей. Рабочие роботы, автоматизированный труд, контроль за качеством продукции, информационные системы управления, компьютеры, Интернет и искусственный интеллект сейчас бурно развиваются. Насколько понимается специалистами-кибернетиками природа и сущность создаваемой ими техники? Информационная техника уходит все дальше вперед, приобретая все новые способности, все большую емкость программирования, становясь все более быстродействующей и компактной, проникая во все сферы жизнедеятельности человека, подвергая своему воздействию науки об обществе и природе, преображая весь ход научного познания от космических исследований до расчета работы магазинов, обеспечивая своевременность решений во всех сложнейших видах планирования экономики от национальных до международных масштабов. И эта ``бесшумная'' программно-математическая революция далека от своего завершения. Но к чему может привести это развитие?

Трудно оценить последствия влияния на человека так называемой ``зеленой революции'', когда современные пустыни можно будет превратить в цветущие поля, потому что техника позволит использовать для орошения почвы опресненную морскую воду или откроется перспектива использования химических препаратов, которые, будучи введены в живую ткань растений, позволят им самим опреснять соленую воду.  Биоинженерия, использующая достижения теоретической и экспериментальной генетики, новые успехи медицины, достигаемые посредством генетического воздействия на микроорганизмы, способные преобразовать фармацевтику, возрождение впечатляющих проектов улучшения человеческого генофонда; угроза бактериологической войны, не менее человекоубийственной, но значительно более ``дешевой'', чем ядерная; контроль над рождаемостью; фантастические потенциалы для производства животных и растительности открывают возможности производства рабочей силы и пищи, прикладная биология и биотехника -- к чему это все может привести?

Великолепное диагностическое оборудование, использующее компьютерную вычислительную систему, или аппараты, заменяющие некоторые органы человека, например, искусственная почка, воплощают в себе осуществленные возможности техники, поставленной на службу охране здоровья человека. В то же время, профессиональные заболевания или болезни, вызванные низким уровнем жизни, все еще остаются жестокой и опасной угрозой для жизни и труда человека.  Средства массовой информации уже давно перешагнули рамки возможностей обычной журналистики, радио и кино.

Химико-физическая теория жизненных процессов, для которых большое значение имеет способность нервных волокон переносить огромную информацию, и исключительная динамическая эластичность мышечной ткани. Такие высокомолекулярные соединения обеспечили бы невообразимый практический прогресс, ибо мышечная ткань, как известно, способна непосредственно преобразовывать химическую энергию в механическую. ``Мышечные двигатели'', по выражению П.Л. Капицы, остаются наиболее эффективными и экономичными машинами, превосходящими в этом отношении паровые, турбинные и прочие тепловые двигатели. Напрашивается вывод, что искусственная мышечная ткань стала бы толчком к возникновению эффективных, компактных механических двигателей, соразмерных человеку.  В исторически сложившемся разделении труда техническая элита наконец приходит к выполнению своей собственной специфической роли, своей власти, вытекающей из специализированного знания; эта элита оказывается в особом положении в сравнении с другими элитарными общественными группами и демократическими формами управления, отделяясь от них барьером сложности научно-технического знания, позволяющим сохранять секретность (военного или промышленного плана) внутри своего узкого круга. В наше время уже никто не сомневается в преимуществах, которые дает интеллектуальное развитие. В отличие от специализированного разделения труда в промышленном производстве современная научная специализация направлена не на замену квалифицированного неквалифицированным трудом; скорее, напротив, более специализированный и квалифицированный научный труд вытесняет менее специализированный и менее квалифицированный. У технических специалистов исчезает внутренняя потребность в целостном взгляде на технические и социальные проблемы, в гуманистическом и разностороннем образовании. Отсюда вытекают опасности для традиционных культурных институтов, для политической и общественной демократии. Эти опасности становятся тем более зловещими, чем в большей степени становится возможным узкотехническое овладение всеми планетными ресурсами.  Таким образом, научные и информационно-технические нововведения, успешные или неудачные, реально достижимые или только воображаемые, выступают как фактор, подрывающий устоявшийся уровень культурной жизни и общественного сознания. Это происходит по следующим причинам:

\begin{itemize}
  
\item{научно-технический прогресс бросает вызов власти, силе, значимости и даже самому существованию традиционных религиозных и эстетических переживаний во всех их формах;}
  
\item{он укрепляет в сознании людей символический фетиш науки и техники, или, иначе говоря, превращает науку в антинауку, рациональное в иррациональное;}
  
\item{он преобразует житейские отношения между людьми, изменяя социальные отношения производства, потребления и коммуникации;}
  
\item{он преображает социальные представления о том, что является удовольствием в исполнении желаний, ослабляя при этом действие культурных традиций, лишая индивида опоры на них, отдавая его во власть иррациональных и бесцеремонных, цепких манипуляций;}
  
\item{техника элитарного социального планирования отчуждается от человека, воспринимается им как разрозненный хаос сиюминутных, односторонних решений, не имеющих связи с реальными жизненными устремлениями людей, превращающих их в безликую массу;}
  
\item{всеобщий характер глобальных проблем в сочетании с безудержным техническим оптимизмом вступает в конфликт с жизненным опытом конкретного человека.}
\end{itemize}

Итак, связанная с наукой, техникой и информацией модернизация человеческой жизни раскрывается перед нами со всеми своими тревогами. Мы обязаны исследовать проблемы, связанные с тем, измеряются ли успехи техники и науки по шкале гуманизма, отвечают ли они потребностям индивидуального развития людей, нужна ли какая-то сверхобычная техника для преодоления опасностей, грозящих человечеству, не следуют ли за сиюминутными успехами непредвиденные и долговременные неудачи, не становится ли чудо науки чем-то подобным религиозным чудесам в сознании масс, а научная аргументация не превращается ли в религиозную риторику, содействует ли научный и информационно-технический прогресс сплочению всего человечества. Мы еще далеки от удовлетворительного понимания радостей и печалей, достижений и провалов, которыми полна техническая деятельность человечества. Среди множества различных технических альтернатив мы должны осуществлять свой выбор с чувством реальной возможности следовать подлинно человеческим ценностям, и должны научиться предвидеть опасности, которые может принести наша научная или инженерная деятельность.
