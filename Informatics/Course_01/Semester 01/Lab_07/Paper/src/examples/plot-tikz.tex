%%%%%%%%%%%%%%%%%%%%%%%%%%%%%%%%%%%%%%%%%%%%%%%%%%%%%%%%%%%%%%%%%%%%%%%%%%%%%%%%
%%%
%%% TIKZ
%%%

\subsubsection{Графики}
\index{графики}

\paragraph{Простые}

\subparagraph{Начало координат уголком}

\begin{center}
	\newcommand{\upPoint}{1.3}
	
	\newcommand{\startX}{2}
	\newcommand{\maxY}{2}

	\begin{tikzpicture}[scale=1.5]
		\draw [<->,thick] (0,2) node (yaxis) [above] {$t$}
			|- (4,0) node (xaxis) [right] {$x$};
		\draw [red] (\startX + 1,0) .. controls (2.7,0.7) and (2.3,1) .. (2,\upPoint) node [left] {\textcolor{blue}{$C^{-}$}};
		\draw [red] (\startX,0) .. controls (\startX,\maxY) and (\startX,\maxY) .. (\startX,\maxY) node [right] 		{\textcolor{blue}{$C^{0}$}}; 
		\draw [red] (\startX - 1,0) .. controls (1.3,0.7) and (1.7,1) .. (2,1.3) node [right] {\textcolor{blue}{$C^{+}$}};
	
		\draw [blue] [->] (\startX - 1,0) -> (1.3,0.7); % % касательные
		\draw [blue] [->] (\startX + 1,0) -> (2.7,0.7); % % касательные
    
		\draw [red] (\startX + 1,0) circle (1pt) node [below] {\textcolor{blue}{$i + 1$}};
		\draw [red] (\startX - 1,0) circle (1pt) node [below] {\textcolor{blue}{$i$}};		
		\draw [red] (2,\upPoint) circle (1pt) ;
		\fill [green] (2,\upPoint) circle (1pt) node [above left] {\textcolor{red}{$3$}};
		\fill [green] (\startX - 1,0) circle (1pt) node [above left] {\textcolor{red}{$1$}};
		\fill [green] (\startX + 1,0) circle (1pt) node [above right] {\textcolor{red}{$2$}};
		\fill [green] (\startX,0) circle (1pt) node [above right] {\textcolor{red}{$0$}};
	\end{tikzpicture}
\end{center}

\subparagraph{Начало координат крестиком}

\index{разрывы}

\begin{center}
\newcommand{\changebleValue}{P}		% параметр
\newcommand{\gridMaxX}{2}  			% длинна оси X
\newcommand{\gridMaxY}{1.5} 		% высота оси Y
\newcommand{\breakPoint}{1.0} 		% точка разрва
\newcommand{\firstWaveheight}{0.5} 	% высота первой области
\newcommand{\secondWaveheight}{1.0} % высота второй области

\newcommand{\drawWaves}{
\begin{tikzpicture}[scale=1.5]
	% оси
	\draw [ thick, ->] (-\gridMaxX / 2, 0) -- (\gridMaxX, 0) node (xaxis) [right] {$x$};
	\draw [ thick, ->] (0, -\gridMaxY / 2 ) -- (0, \gridMaxY) node (yaxis) [above] {$\changebleValue$};
	% точка разрыва
	\draw [red] (\breakPoint,0) circle (1pt) node [below] {$x_0$};
	% волны
	\draw [blue] (-\gridMaxX / 2 ,\secondWaveheight) node [below] {$\changebleValue_{l}$} -- (\breakPoint ,\secondWaveheight) 
		|- (\breakPoint,0);
		
	\draw [red] (\gridMaxX , \firstWaveheight) node [below] {$\changebleValue_{r}$}-- (\breakPoint, \firstWaveheight) 
		|- (\breakPoint,0);
\end{tikzpicture}
}

\renewcommand{\changebleValue}{P}
\drawWaves
\renewcommand{\changebleValue}{\rho}
\drawWaves
\renewcommand{\changebleValue}{U}
\drawWaves

\end{center}

\pagebreak

\subparagraph{Сетка (ручная)}

\begin{center}
\newcommand{\gridStep}{1.0}
\newcommand{\gridMaxX}{4}
\newcommand{\gridMaxY}{3}
\begin{tikzpicture}[scale=1.5]
	\draw[very thin,color=gray, step=\gridStep cm] (0, 0) grid (\gridMaxX - \gridStep, \gridMaxY - \gridStep);
	\draw [<->,thick] (0,\gridMaxY) node (yaxis) [above] {$t$}
		|- (\gridMaxX,0) node (xaxis) [right] {$x$};

	\fill [green] (0.5, 0) circle (2pt);  
	\fill [green] (1.5, 0) circle (2pt);  
	\fill [green] (2.5, 0) circle (2pt);  
	
	\fill [green] (0.5, 1) circle (2pt);  
	\fill [green] (1.5, 1) circle (2pt);  
	\fill [green] (2.5, 1) circle (2pt);  

	\draw [blue] (0.5, 0) circle (2pt);  
	\draw [blue] (1.5, 0) circle (2pt);  
	\draw [blue] (2.5, 0) circle (2pt);  
		
	\draw [blue] (0.5, 1) circle (2pt);  
	\draw [blue] (1.5, 1) circle (2pt);  
	\draw [blue] (2.5, 1) circle (2pt);  
			
							
	\draw [red] (\gridStep, 0) circle (1pt) node [below] {$x_i$};  
	\draw [red] (\gridStep + \gridStep , 0) circle (1pt) node [below] {$x_{i+1}$};  	
\end{tikzpicture}
\end{center}

\paragraph{Преобразования координат}

\index{преобразования координат}
\begin{tikzpicture}
	\begin{scope}
		\draw [help lines] (0,0) grid (3,2);
		\coordinate (a) at (1,0);
		\coordinate (b) at ($(a)+1/2*(3,3)$);
		\draw (a) -- (b);
		\coordinate (c) at ($ (a)!.25!(b) $);
		\coordinate (d) at ($ (c)!1cm!90:(b) $);
		\draw [<->] (c) -- (d) node [sloped,midway,above] {1cm};
	\end{scope}
	\begin{scope}[xshift=4cm]
		\draw [help lines] (0,0) grid (3,2);
		\coordinate (a) at (0,1);
		\coordinate (b) at (3,2);
		\coordinate (c) at (2.5,0);
		\draw (a) -- (b) -- (c) -- cycle;
		\draw[red] (a) -- ($(b)!(a)!(c)$);
		\draw[orange] (b) -- ($(a)!(b)!(c)$);
		\draw[blue] (c) -- ($(a)!(c)!(b)$);
	\end{scope}
\end{tikzpicture}


\pagebreak

\subsubsection{Дигаммы}

\index{дигаммы}
\index{деформация}
\index{градиент}

\paragraph{C  градиентом и деформацией}

\begin{center}

\tikzstyle{format} = [rounded rectangle,thick,minimum size=1cm,draw=blue!50!black!50,top color=white,bottom color=blue!50!black!20,font=\itshape]

\tikzstyle{serverf} = [rectangle,thick,minimum size=1cm,draw=blue!50!black!50,top color=white,bottom color=blue!50!black!20,font=\itshape]

\tikzstyle{clientf} = [rounded rectangle,thick,minimum size=1cm,draw=red!50!black!50,top color=white,bottom color=red!50!black!20,font=\itshape]

\tikzstyle{netf} = [draw=yellow!50!black!70,thick,minimum height=1cm,minimum width=2cm,top color=yellow!20,bottom color=yellow!60!black!20,decorate,decoration={random steps,segment length=3pt,amplitude=1pt}]

\begin{tikzpicture}[thick,	node distance=4cm,	text height=1.5ex,	text depth=.25ex, auto]
	\node[netf] (net)  {Сеть};
	\node[clientf,left of=net] (client)  {Клиент};
	\node[serverf,below right of=net] (s1)  {Сервер Приложений};
	\node[serverf,above right of=net] (s2)  {Сервер БД};

	\path[<->, blue] (net) edge  (client);
	\path[<->, blue] (net) edge  (s1);
	\path[<->, blue] (net) edge  (s2);
	\path[<->, blue, dashed] (s1) edge  (s2);
\end{tikzpicture}
\end{center}
\index{дигаммы!тень}
\index{тень}

\paragraph{С тенью}

\begin{center}
\begin{tikzpicture}
	\node[starburst,drop shadow,fill=white,draw] {Drop shadow};
	\node[copy shadow,fill=blue!20,draw=blue,thick] at (3.5,0) {Copy shadow};
	\node[circle,circular drop shadow,fill=blue!20,draw] at (6,0) {Circular};
\end{tikzpicture}
\end{center}

\pagebreak
