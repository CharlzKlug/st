\documentclass[a4paper, 12pt]{article}
\usepackage{index}
\usepackage[T2A]{fontenc}
\usepackage[utf8]{inputenc}
\usepackage[english, russian]{babel}
\usepackage{upquote}
\usepackage{textcomp}
\usepackage[pdftex]{graphicx}
\usepackage{pdfpages}
\usepackage{color}
\usepackage[pdfborder={0 0 0}]{hyperref}
\begin{document}
Домашняя работа по информатике. Выполнил студент группы ТМБО-01-15 Аметов Имиль.

\emph{Задача:}

Дана карта Карно (смотрите ниже). Найдите минимизированную дизъюнктивную нормальную форму.

\begin{tabular}{ccccc}
  \multicolumn{1}{c|}{} & \multicolumn{4}{c|}{$x_1x_2$} \\ \hline
  \multicolumn{1}{c|}{$x_3x_4$} & 00 & 01 & 11 & \multicolumn{1}{c|}{10} \\ \hline
  \multicolumn{1}{c|}{00} & && \multicolumn{1}{|c}{1} & \multicolumn{1}{c|}{1} \\ \cline{5-5} \cline{2-2}
  \multicolumn{1}{c|}{01} & \multicolumn{1}{|c|}{1} && \multicolumn{1}{|c}{1} &\multicolumn{1}{|c|}{1} \\ \cline{2-5} 
  \multicolumn{1}{c|}{11} && \multicolumn{1}{|c|}{1} & & \multicolumn{1}{c|}{}\\ \cline{3-3}
  \multicolumn{1}{c|}{10} &&&& \multicolumn{1}{c|}{}\\ \hline
\end{tabular}

На карте уже выделены участки объединения. Из них мы получаем следующее: $y(x_1, x_2, x_3, x_4) = x_1 \cdot \overline{x_3} \vee \overline{x_2} \cdot \overline{x_3} \cdot x_4 \vee \overline{x_1} \cdot x_2 \cdot x_3 \cdot x_4$.

Минимизируем заданный набор данных методом Куайна. Из таблицы получаем следующую совершенную дизъюнктивную нормальную форму:

\begin{eqnarray}\label{start_equation}
  y = x_1 \cdot x_2 \cdot \overline{x_3} \cdot \overline{x_4} \vee x_1 \cdot \overline{x_2} \cdot \overline{x_3} \cdot \overline{x_4} \vee \nonumber \\
  \overline{x_1} \cdot \overline{x_2} \cdot  \overline{x_3} \cdot x_4 \vee x_1 \cdot x_2 \cdot \overline{x_3} \cdot x_4 \vee \nonumber \\
  x_1 \cdot \overline{x_2} \cdot \overline{x_3} \cdot x_4 \vee \overline{x_1} \cdot x_2 \cdot x_3 \cdot x_4
\end{eqnarray}

Из этого уравнения получаем следующие комбинации, где применим закон склеивания:

$$x_1 \cdot x_2 \cdot \overline{x_3} \cdot \overline{x_4} \vee x_1 \cdot \overline{x_2} \cdot \overline{x_3} \cdot \overline{x_4} = x_1 \cdot \overline{x_3} \cdot \overline{x_4}$$
$$x_1 \cdot x_2 \cdot \overline{x_3} \cdot \overline{x_4} \vee x_1 \cdot x_2 \cdot \overline{x_3} \cdot x_4 = x_1 \cdot x_2 \cdot \overline{x_3}$$
$$x_1 \cdot \overline{x_2} \cdot \overline{x_3} \cdot \overline{x_4} \vee x_1 \cdot \overline{x_2} \cdot \overline{x_3} \cdot x_4 = x_1 \cdot \overline{x_2} \cdot \overline{x_3}$$
$$\overline{x_1} \cdot \overline{x_2} \cdot \overline{x_3} \cdot x_4 \vee x_1 \cdot \overline{x_2} \cdot \overline{x_3} \cdot x_4 = \overline{x_2} \cdot \overline{x_3} \cdot x_4$$
$$x_1 \cdot x_2 \cdot \overline{x_3} \cdot x_4 \vee x_1 \cdot \overline{x_2} \cdot \overline{x_3} \cdot x_4 = x_1 \cdot \overline{x_3} \cdot x_4$$

Подставляем полученные значения в формулу \ref{start_equation}:

\begin{eqnarray}\label{second_eq}
  y = x_1 \cdot x_2 \cdot \overline{x_3} \cdot \overline{x_4} \vee x_1 \cdot \overline{x_2} \cdot \overline{x_3} \cdot \overline{x_4} \vee \overline{x_1} \cdot \overline{x_2} \cdot  \overline{x_3} \cdot x_4 \vee \nonumber \nonumber \\
  x_1 \cdot x_2 \cdot \overline{x_3} \cdot x_4 \vee x_1 \cdot \overline{x_2} \cdot \overline{x_3} \cdot x_4 \vee \overline{x_1} \cdot x_2 \cdot x_3 \cdot x_4 \vee \nonumber \\
  x_1 \cdot \overline{x_3} \cdot \overline{x_4} \vee x_1 \cdot x_2 \cdot \overline{x_3} \vee x_1 \cdot \overline{x_2} \cdot \overline{x_3} \vee \overline{x_2} \cdot \overline{x_3} \cdot x_4 \vee x_1 \cdot \overline{x_3} \cdot x_4 = \nonumber \\
  \overline{x_1} \cdot x_2 \cdot x_3 \cdot x_4 \vee x_1 \cdot \overline{x_3} \cdot \overline{x_4} \vee x_1 \cdot x_2 \cdot \overline{x_3} \vee x_1 \cdot \overline{x_2} \cdot \overline{x_3} \vee \overline{x_2} \cdot \overline{x_3} \cdot x_4 \vee x_1 \cdot \overline{x_3} \cdot x_4 
\end{eqnarray}

Снова применяем склеивание:

$$x_1 \cdot \overline{x_3} \cdot \overline{x_4} \vee x_1 \cdot \overline{x_3} \cdot x_4 = x_1 \cdot \overline{x_3}$$
$$x_1 \cdot x_2 \cdot \overline{x_3} \vee x_1 \cdot \overline{x_2} \cdot \overline{x_3} = x_1 \cdot \overline{x_3}$$

Подставляем полученные значения в формулу \ref{second_eq}:

\begin{eqnarray}\label{third_eq}
  y = \overline{x_1} \cdot x_2 \cdot x_3 \cdot x_4 \vee x_1 \cdot \overline{x_3} \cdot \overline{x_4} \vee x_1 \cdot x_2 \cdot \overline{x_3} \vee x_1 \cdot \overline{x_2} \cdot \overline{x_3} \vee \overline{x_2} \cdot \overline{x_3} \cdot x_4 \vee x_1 \cdot \overline{x_3} \cdot x_4 \vee \nonumber \\
  x_1 \cdot \overline{x_3} = x_1 \cdot \overline{x_3} \vee \overline{x_2} \cdot \overline{x_3} \cdot x_4 \vee \overline{x_1} \cdot x_2 \cdot x_3 \cdot x_4
\end{eqnarray}

Формула \ref{third_eq} та же, что и формула, полученная по методу Карно. Следовательно, решение найдено верно.
\end{document}
