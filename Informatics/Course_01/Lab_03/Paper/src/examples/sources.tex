\section[Исходный код]{Исходный код}

%%%%%%%%%%%%%%%%%%%%%%%%%%%%%%%%%%%%%%%%%%%%%%%%%%%%%%%%%%%%%%%%%%%%%%%%%%%%%%%%
%%%
\subsection{lstlisting}
\index{lstlisting}
\index{листинги}
\index{исходники}
\index{код}
Исходный код с помощью пакета \textbf{listings} (или \textbf{listingsutf8}).
Пакет хорошо работает с однобайтовыми кодировками, но при любых настроках отказался дружить с utf8.

\begin{lstlisting}
	\usepackage[utf8]{inputenc}							% кодировка, тут очень аккуратно
\end{lstlisting}

\colorbox{yellow}{Проблема глобальна}.
И я не нашел стандартного пути решения (в pdf\LaTeX и \XeTeX~---~в $\Lambda$ ее нет).

%% \begin{lstlisting}[language=Tex, escapeinside='']
\begin{lstlisting}[escapeinside='', firstnumber=100]
	%\usepackage{listingsutf8}	
	\usepackage{listings}
	\lstset{
		language=Tex,
		tabsize=2,
		breaklines,
		columns=fullflexible,
		flexiblecolumns,
		frame=tb ,
		numbers=left,
		numberstyle=\footnotesize\color{gray},
		escapechar = |, % 'можно вывалиться в \TeX' 
		extendedchars = false,
			% extendedchars = true, 
				%% да именно так но не  \true
				%% \true == false
		inputencoding = utf8, % кодировка, очень аккуратно тут
			% inputencoding = utf8/cp1251, % кодировка, очень аккуратно тут
		keepspaces = true,
		belowcaptionskip=5pt
	}
\end{lstlisting}

Пути решения:
\begin{itemize}
	\item Не использовать русских комментариев
	\item Использовать \textbf{verbatim}, 
\end{itemize}

\begin{lstlisting}[language=ConfigNetTopo]
[localhost]

[[7200]]
image = /usr//bin/Dynamips/images/c7200-is-mz.122-40.bin
	ram = 128
	npe = npe-300

[[3640]]
	image = /usr/bin/Dynamips/images/3640-is-mz.122-40.bin
	ram = 64
	model = 3640
	slot0 = NM-1E
	slot1 = NM-1FE-TX
	slot2 = NM-1FE-TX
		
[[ROUTER Alpha]]
model = 7200
	slot0 = C7200-IO-FE
	slot1 = PA-8E
	f0/0 = LAN 1
	e1/0 = Client09 e0/0
	e1/1 = Client10 e0/0	
	console = 2000

[[ROUTER Client09]]
	model = 3640
	f1/0 = LAN 2
	f2/0 = LAN 29		
	console = 2010		
		
[[ROUTER Client10]]
	model = 3640
	f1/0 = LAN 2
	f2/0 = LAN 30	
	console = 2011
\end{lstlisting}


\pagebreak

%%%%%%%%%%%%%%%%%%%%%%%%%%%%%%%%%%%%%%%%%%%%%%%%%%%%%%%%%%%%%%%%%%%%%%%%%%%%%%%%
%%%
\subsection{verbatim}

\index{verbatim}
Его проблемы:
\begin{itemize}
	\item Нет подсветки синтаксиса
	\item Нет номеров строк
	\item Надо использовать пробелы вместо табуляции
\end{itemize}

\begin{verbatim}
    %\usepackage{listingsutf8}	%%  ---> %% utf8/cp1251
    \usepackage{listings}
    \lstset{
        language=Tex,
        tabsize=2,
        breaklines,
        columns=fullflexible,
        flexiblecolumns,
        frame=tb ,
        numbers=left,
        numberstyle={\footnotesize},
        extendedchars = false,
                % extendedchars = true, 
                        %% да именно так но не  \true
                        %% \true == false
        inputencoding = utf8, % кодировка, очень аккуратно тут
                % inputencoding = utf8/cp1251, 
        belowcaptionskip=5pt 
    }
\end{verbatim}

\pagebreak %% Разрыв страницы :-)
