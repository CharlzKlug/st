\documentclass[11pt]{article}
\usepackage[T2A]{fontenc}
\usepackage[russian,english]{babel}
\usepackage[utf8]{inputenc}

\usepackage{amssymb,amsmath,amsthm,amsfonts}
\usepackage{graphicx}
\usepackage{hyperref}
\title{Про силу притяжения}
\author{Аметов И.И.}
\begin{document}
\maketitle
У меня получилась следующая система

\begin{equation*}
\begin{cases}
\ddot {\boldsymbol r}_1 = \dfrac {G m_2(\boldsymbol r_2 - \boldsymbol r_1)} {|\boldsymbol r_2 - \boldsymbol r_1|^3} \\
\ddot {\boldsymbol r}_2 = \dfrac {G m_1(\boldsymbol r_1 - \boldsymbol r_2)} {|\boldsymbol r_1 - \boldsymbol r_2|^3} \\
\end{cases}
\end{equation*}

Преобразовал в такую систему

\begin{equation*}
\begin{cases}
\dot {\boldsymbol r}_1 = \boldsymbol a \\
\dot {\boldsymbol r}_2 = \boldsymbol b \\
\dot {\boldsymbol a} = \dfrac {G m_2(\boldsymbol r_2 - \boldsymbol r_1)} {|\boldsymbol r_2 - \boldsymbol r_1|^3} \\
\dot {\boldsymbol b} = \dfrac {G m_1(\boldsymbol r_1 - \boldsymbol r_2)} {|\boldsymbol r_1 - \boldsymbol r_2|^3}
\end{cases}
\end{equation*}

Ввёл такие обозначения:

$$\boldsymbol c \approx \boldsymbol r_1$$
$$\boldsymbol d \approx \boldsymbol r_2$$
$$\boldsymbol e \approx \boldsymbol a$$
$$\boldsymbol f \approx \boldsymbol b$$

По методу Рунге-Кутта:

$$\boldsymbol c_{n+1} = \boldsymbol c_n + \dfrac {1}{6} h ( \boldsymbol g_1 + 2 \boldsymbol g_2 + 2 \boldsymbol g_3 + \boldsymbol g_4)$$
$$\boldsymbol d_{n+1} = \boldsymbol d_n + \dfrac {1}{6} h ( \boldsymbol h_1 + 2 \boldsymbol h_2 + 2 \boldsymbol h_3 + \boldsymbol h_4)$$
$$\boldsymbol e_{n+1} = \boldsymbol e_n + \dfrac {1}{6} h ( \boldsymbol e_1 + 2 \boldsymbol e_2 + 2 \boldsymbol e_3 + \boldsymbol e_4)$$
$$\boldsymbol f_{n+1} = \boldsymbol f_n + \dfrac {1}{6} h ( \boldsymbol f_1 + 2 \boldsymbol f_2 + 2 \boldsymbol f_3 + \boldsymbol f_4)$$

Для коэффициентов $\boldsymbol g_1, .., \boldsymbol g_4$:

$$\boldsymbol g_1 = \boldsymbol a_n$$
$$\boldsymbol g_2 = \boldsymbol a_n + \dfrac {1} {2} h \boldsymbol g_1$$
$$\boldsymbol g_3 = \boldsymbol a_n + \dfrac {1} {2} h \boldsymbol g_2$$
$$\boldsymbol g_4 = \boldsymbol a_n + h \boldsymbol g_3$$

Для коэффициентов $\boldsymbol h_1, .., \boldsymbol h_4$:

$$\boldsymbol h_1 = \boldsymbol b_n$$
$$\boldsymbol h_2 = \boldsymbol b_n + \dfrac {1} {2} h \boldsymbol h_1$$
$$\boldsymbol h_3 = \boldsymbol b_n + \dfrac {1} {2} h \boldsymbol h_2$$
$$\boldsymbol h_4 = \boldsymbol b_n + h \boldsymbol h_3$$

Для коэффициентов $\boldsymbol k_1, .., \boldsymbol k_4$:

$$\boldsymbol k_1 = \dfrac {G m_2(\boldsymbol d_n - \boldsymbol c_n)} {|\boldsymbol d_n - \boldsymbol c_n|^3}$$
$$\boldsymbol k_2 = \dfrac {G m_2 \left ( \left ( \boldsymbol d_n + \dfrac {1} {2} h \boldsymbol h_1 \right ) - \left ( \boldsymbol c_n + \dfrac {1}{2} h \boldsymbol g_1 \right ) \right )} {\left | \left ( \boldsymbol d_n + \dfrac {1} {2} h \boldsymbol h_1 \right ) - \left ( \boldsymbol c_n + \dfrac {1}{2} h \boldsymbol g_1 \right ) \right |^3}$$
$$\boldsymbol k_3 = \dfrac {G m_2 \left ( \left ( \boldsymbol d_n + \dfrac {1} {2} h \boldsymbol h_2 \right ) - \left ( \boldsymbol c_n + \dfrac {1}{2} h \boldsymbol g_2 \right ) \right )} {\left | \left ( \boldsymbol d_n + \dfrac {1} {2} h \boldsymbol h_2 \right ) - \left ( \boldsymbol c_n + \dfrac {1}{2} h \boldsymbol g_2 \right ) \right |^3}$$
$$\boldsymbol k_4 = \dfrac {G m_2 \left ( \left ( \boldsymbol d_n + h \boldsymbol h_3 \right ) - \left ( \boldsymbol c_n + h \boldsymbol g_3 \right ) \right )} {\left | \left ( \boldsymbol d_n + h \boldsymbol h_3 \right ) - \left ( \boldsymbol c_n + h \boldsymbol g_3 \right ) \right |^3}$$

Для коэффициентов $\boldsymbol l_1, ..., \boldsymbol l_4$:

$$\boldsymbol l_1 = \dfrac {G m_1(\boldsymbol c_n - \boldsymbol d_n)} {|\boldsymbol c_n - \boldsymbol d_n|^3}$$
$$\boldsymbol l_2 = \dfrac {G m_1 \left ( \left ( \boldsymbol c_n + \dfrac {1} {2} h \boldsymbol g_1 \right ) - \left ( \boldsymbol d_n + \dfrac {1}{2} h \boldsymbol h_1 \right ) \right )} {\left | \left ( \boldsymbol c_n + \dfrac {1} {2} h \boldsymbol g_1 \right ) - \left ( \boldsymbol d_n + \dfrac {1}{2} h \boldsymbol h_1 \right ) \right |^3}$$
$$\boldsymbol l_3 = \dfrac {G m_1 \left ( \left ( \boldsymbol c_n + \dfrac {1} {2} h \boldsymbol g_2 \right ) - \left ( \boldsymbol d_n + \dfrac {1}{2} h \boldsymbol h_2 \right ) \right )} {\left | \left ( \boldsymbol c_n + \dfrac {1} {2} h \boldsymbol g_2 \right ) - \left ( \boldsymbol d_n + \dfrac {1}{2} h \boldsymbol h_2 \right ) \right |^3}$$
$$\boldsymbol l_4 = \dfrac {G m_1 \left ( \left ( \boldsymbol c_n + h \boldsymbol g_3 \right ) - \left ( \boldsymbol d_n + h \boldsymbol h_3 \right ) \right )} {\left | \left ( \boldsymbol c_n + h \boldsymbol g_3 \right ) - \left ( \boldsymbol d_n + h \boldsymbol h_3 \right ) \right |^3}$$
\end{document}
