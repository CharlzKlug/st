\section{Перевод}

\begin{russian}

  \subsection{Введение}


Массачусетский Институт Технологий (МИТ) --- это частный исследовательский университет в Кэмбридже, Массачусетс. Основан в 1861 году в ответ на увеличивающуюся индустриализацию Соединённых Штатов, МИТ перенял Европейскую модель политехнических университетов с акцентом на лабораторные исследования в прикладной науке и инженерии. Со Второй Мировой Войны и Холодной Войны исследователи работали над компьютерами, радиолокацией и инерциальной навигацией. В послевоенное время при Джеймсе Киллиане исследования в области обороны способствовали быстрому разрастанию факультета и кампуса. Существующий в данное время кампус площадью в 168 акров (68 гектар) открыт в 1916 году и тянется в одну милю (1.6 километров) вдоль северного берега бассейна реки Чарльз.

МИТ, своими пятью школами и одним колледжем, составляющими в общей сложности 32 отделения, традиционно известен исследованиями и образованием в физических науках и инженерии, а в последнее время также в биологии, экономике, лингвистике и менеджменте. МИТ часто упоминается в числе лучших университетов мира.

На 2015 год с МИТ-ом связаны 84 Нобелевских лауреата, 52 обладателя Национальной научной медали США, 45 стипендиантов Родса, 38 стипендиантов МакАртура, 34 космонавта и 2 Филдсовских медалиста. МИТ обладает сильной предпринимательской культурой, и общие доходы компаний, созданных выпускниками МИТ можно рассматривать как одиннадцатую, по величине, экономику в мире.

\subsection{История}
История Массачусетского Института Технологий может быть отслежена до 1861 года регистрацией ``Массачусетского Института Технологий и Бостонского Сообщества Естественной Истории'' прежде всего во главе с Виллиамом Бартоном Роджерсом.

\subsubsection{Видение и миссия}

10 Апреля 1861 года губернатор штата Массачусетс подписал хартию о регистрации ``Массачусетского Института Технологий и Бостонского Сообщества Естественной Истории'', которая была предложена Виллиамом Бартоном Роджерсом, учёным естествоиспытателем. Роджерс стремился создать новую форму высшего образования в ответ на трудности, связанные с быстрым развитием в науке и технологии в середине 19-го века, поскольку он верил что классические институты были плохо подготовлены для них.
\end{russian}
