\section{Перевод}

\begin{russian}

  \subsection{Введение}


Массачусетский Институт Технологий (МИТ) --- это частный исследовательский университет в Кэмбридже, Массачусетс. Основан в 1861 году в ответ на увеличивающуюся индустриализацию Соединённых Штатов, МИТ перенял Европейскую модель политехнических университетов с акцентом на лабораторные исследования в прикладной науке и инженерии. Со Второй Мировой Войны и Холодной Войны исследователи работали над компьютерами, радиолокацией и инерциальной навигацией. В послевоенное время при Джеймсе Киллиане исследования в области обороны способствовали быстрому разрастанию факультета и кампуса. Существующий в данное время кампус площадью в 168 акров (68 гектар) открыт в 1916 году и тянется в одну милю (1.6 километров) вдоль северного берега бассейна реки Чарльз.

МИТ, своими пятью школами и одним колледжем, составляющими в общей сложности 32 отделения, традиционно известен исследованиями и образованием в физических науках и инженерии, а в последнее время также в биологии, экономике, лингвистике и менеджменте. МИТ часто упоминается в числе лучших университетов мира.

На 2015 год с МИТ-ом связаны 84 Нобелевских лауреата, 52 обладателя Национальной научной медали США, 45 стипендиантов Родса, 38 стипендиантов МакАртура, 34 космонавта и 2 Филдсовских медалиста. МИТ обладает сильной предпринимательской культурой, и общие доходы компаний, созданных выпускниками МИТ можно рассматривать как одиннадцатую, по величине, экономику в мире.

\subsection {Исследования}

Открытия и разработки в электронике, памяти на магнитных сердечниках, радары, одноэлектронные транзисторы и управление на основе инерциальной навигации были сделаны или существенно разработаны исследователями МИТ-а. Гарольд Юджин Эджертон стал пионером в высокоскоростной фотографии и сонарах. Клод Е. Шеннон разработал большую часть информационной теории и открыл приложение Булевой логики к теории проектирования цифровых схем. Сотрудники и исследователи МИТ-а сделали фундаментальный вклад в области информатики, в частности, кибернетики, искусственного интеллекта, компьютерных языков, машинного обучения, робототехники и криптографии. По-крайней мере девять лауреатов Премии Тьюринга и семь обладателей Премии Драпера в области инженерии связаны или были связанными с МИТ-ом.

За прошлое и настоящее время факультет физики выигрывал восемь Премий Нобеля, четыре Медали Дирака и три премии Вольфа за их вклад в субатомную и квантовую теории. Члены кафедры химии были награждены тремя Премиями Нобеля и одной Премией Вольфа за открытие новых способов и методов синтеза. Биологи МИТ были награждены шестью Премиями Нобеля за их вклад в генетику, иммунологию, онкологию и молекулярную биологию. Профессор Эрик Ландер был одним из ведущих лидеров в Проекте Генома Человека. Позитроний, синтетический пенициллин, синтетические самовоспроизводящиеся молекулы и генетические основы бокового (латерального) амиотрофического склероза (также известного как БАС, болезнь Лу Герига или болезнь Шарко) и болезни Хантингтона были впервые открыты в МИТ. 
\subsection{История}
История Массачусетского Института Технологий может быть отслежена до 1861 года регистрацией ``Массачусетского Института Технологий и Бостонского Сообщества Естественной Истории'' прежде всего во главе с Виллиамом Бартоном Роджерсом.

\subsubsection{Видение и миссия}

10 Апреля 1861 года губернатор штата Массачусетс подписал хартию о регистрации ``Массачусетского Института Технологий и Бостонского Сообщества Естественной Истории'', которая была предложена Виллиамом Бартоном Роджерсом, учёным естествоиспытателем. Роджерс стремился создать новую форму высшего образования в ответ на трудности, связанные с быстрым развитием в науке и технологии в середине 19-го века, поскольку он верил что классические институты были плохо подготовлены для них.

Когда хартия была подписана, Роджерс начал собирать средства, разрабатывать учебный план и искать подходящее место для института. План Роджерса, как это стало известно, состоял из трёх принципов: образовательная ценность полезных знаний, необходимость ``обучения через выполнение'' и интеграция профессионального и гуманитарного составляющих в начальный уровень образования. МИТ стал пионером в области лабораторного обучения. Его базовая философия: ``обучение -- не манипуляция и мелкие детали изучаемого, с которыми можно работать только в мастерской, но внедрение всех научных принципов, формирующих основу и объясняющую их;''

Поскольку открытый конфликт в Гражданской Войне США разразился только два дня спустя 12-го апреля, 1861 года, то Роджерс столкнулся с огромными трудностями при сборе средств, требуемых для того, чтобы соответствовать финансовым условиям со стороны штата. Таким образом, набор факультета и студентов был отложен, но, всё же первые занятия МИТ-а были проведены в арендованном помещении Меркантиль Билдинг в деловом центре Бостона в 1865 году.

\subsection{Бостон Тек (1865-1916)}

Постройка первого здания МИТ была завершена в Бостонском Бэк Бэе в 1866 году и стала известна как ``Бостон Тек'' пока кампус не был перемещён через реку Чарльз в Кембридж в 1916 году.

На Всемирной Выставке в Филадельфии в 1875 году, Рункль был впечатлён работой Виктора Делла-Вос из Российской империи, представившего педагогический подход, комбинирующий практическое и теоретическое обучение, применённое в Императорском Московском техническом училище (ныне Московский государственный технический университет им. Н. Э. Баумана). Рункль стал сторонником этого подхода и ввёл его в МИТ.

Фрэнсис Амаса Уокер был выбран на должность Президента попечительского совета МИТ 25 мая 1881 года. Уокер ввёл новый курс обучения с упором на экономику, историю, юриспруденцию, английский и другие современные языки. Уокер также предложил реформу и расширение организации института создав Исполнительный Коммитет, состоящий из пятидесяти членов попечительского совета, предназначением которого было решение возникающих административных проблем и подчеркнул важность посещения собраний коммитета и их советы в основных решениях в управлении факультетом.

При президентстве Уокера были введены новые программы: Электрическая Инженерия в 1882, Химическая Инженерия в 1888, Коммунальные Службы в 1889, Геология в 1890, Кораблестроение в 1893.

МИТ стал первым университетом нации, в учебном плане которого были введены: архитектура (1865), электрическая инженерия (1882), коммунальные службы (1889), кораблестроение и морская инженерия (1895), инженерия в области аэронавтики (1914), метеорология (1928), ядерная физика (1935) и искусственный интеллект (1960).

\subsection{Первая Мировая Война (1917---1939)}

В Военно-Морских Силах Соединённых Штатов во время Первой Мировой Войны возникла необходимость в программе подготовки пилотов для развивающихся технологий военной авиации. После объявления Соединёнными штатами войны 6 апреля 1917 года, Военно-Морские Силы развернули программу подготовки пилотов, состоящую из трёх частей: начиналось с двух месяцев базовой школы, следом шла предварительная подготовка курсанта к самостояльным полётам и расширенная лётная подготовка для присвоения квалификации военно-морского лётчика с комиссованием в Военно-Морской Резерв Военно-Воздушных Сил.

Коммандир Джером Кларк Хансейкер, ранее обучавшийся, а позже преподававший в МИТ авиационную технику, предложил Министру Военно-Морских Сил создать первую базовую школу Военно-Морских Сил для подготовки пилотов при МИТ. Первый набор из пятидесяти студентов прибыл для прохождения обучения 23 июля 1917 года, в программу обучения входили: электричество, сигналы, фотография, мореплавание, судоходство, стрельба, аэролетательные двигатели, теория полёта и авиационные приборы.

\subsection{Вторая Мировая Война и Холодная Война (1940-1966)}

МИТ радикально изменил своё участие в военных исследованиях во время Второй Мировой Войны. Буш, который стал вице-президентом МИТ-а (фактически проректором) был назначен главой Управления Научных Исследований и Разработок, ответственных за Манхэттенский Проект. Исследования, спонсируемые правительством, привели к фантастическому росту числа сотрудников Института, занятых в исследованиях, увеличению физического оснащения лабораторий и сдвигу образовательного фокуса с обучения студентов, на обучение аспирантов.

Усиливалась Холодная Война и Космическая Гонка, также росла озабоченность в технологическом разрыве между Соединёнными Штатами и Советским Союзом в период с 1950-х и до 1960-х годов, участие МИТ-а в военно-индустриальном комплексе становится источником гордости университетского городка.

\subsection {Социальные движения и активность (1966-1980)}

\subsubsection {Совместное обучение}

В МИТ-е номинально обучались студенты мужского и женского пола, поскольку первый студент женского пола Эллен Суоллоу Ричардс поступила в 1870 году. Однако, девушки-студентки составляли ничтожное меньшинство (десяток-другой человек) до завершения постройки первого женского общежития, МакКормик Холл, в 1964 году. В выпуске 2013 года женщины составляли 45\% студентов и 31\% аспирантов. Ричардс также стала первой девушкой-участником факультета МИТ-а, специализирующемся на здоровье окружающей среды.

\subsubsection {Антивоенные протесты}

Однако, в поздних 1960-х и ранних 1970-х, активный протест студентов и активистов из преподавательско-профессорского состава против военных исследований привели к тому, что администрация МИТ была вынуждена свернуть эти лаборатории, позже из этих лабораторий возникли Лаборатория Чарльза Старка Драпера и Лаборатория Линкольна. Уровень этих протестов отражает тот факт, что из МИТ-а было больше всего имён в ``списке врагов президента Никсона'' чем из других отдельных организаций, среди них президент МИТ Джером Визнер и профессор Ноам Хомски. Записи, выявленные в ходе Уотергейтского скандала показали, что Никсон приказал сократить федеральные субсидии МИТ-у ``из-за анти-оборонных предубеждений Визнера''.

\subsubsection {Социальные движения}

Частично МИТ-овский штамм анти-авторитаризма проявляет себя и в других формах. В 1977 году, две девушки, студенты младших курсов, Сюзан Гилберт и Роксанна Ритчи получили дисциплинарное взыскание за публикацию статьи от 28 апреля в ``альтернативном'' еженедельном кампусе МИТ-а, проводимого по четвергам. Названная как ``Руководство пользователя по Мужчинам МИТ-а'' статья была сексуальным обзором 36 мужчин, с которыми эти две девушки переспали, после чего была составлена таблица, в которой мужчины были расставлены в зависимости от их производительности. Гилберт и Ритчи планировали переработать таблицу в рейтинговую систему и книгу с фотографиями мужчин, которые могли бы использовать женщины, но эта статья привела не только к дисциплинарному взысканию против авторов, но и протестной петиции, подписанной двумястами студентами и неодобренной президентом Джеромом Б. Визнером, опубликовавшему жёсткую критику этой статьи.
\end{russian}
