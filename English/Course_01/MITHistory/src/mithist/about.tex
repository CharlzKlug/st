\Csection{Introduction}

The Massachusetts Institute of Technology (MIT) is a private research university in Cambridge, Massachusetts. Founded in 1861 in response to the increasing industrialization of the United States, MIT adopted a European polytechnic university model and stressed laboratory instruction in applied science and engineering. Researchers worked on computers, radar, and inertial guidance during World War II and the Cold War. Post-war defense research contributed to the rapid expansion of the faculty and campus under James Killian. The current 168-acre (68.0 ha) campus opened in 1916 and extends over 1 mile (1.6 km) along the northern bank of the Charles River basin.

MIT, with five schools and one college which contain a total of 32 departments, is traditionally known for its research and education in the physical sciences and engineering, and more recently in biology, economics, linguistics, and management as well. MIT is often cited as among the world's top universities.

As of 2015, 84 Nobel laureates, 52 National Medal of Science recipients, 45 Rhodes Scholars, 38 MacArthur Fellows, 34 astronauts, and 2 Fields Medalists have been affiliated with MIT. The school has a strong entrepreneurial culture, and the aggregated revenues of companies founded by MIT alumni would rank as the eleventh-largest economy in the world.

\subsection {Research}

In electronics, magnetic core memory, radar, single electron transistors, and inertial guidance controls were invented or substantially developed by MIT researchers. Harold Eugene Edgerton was a pioneer in high speed photography and sonar. Claude E. Shannon developed much of modern information theory and discovered the application of Boolean logic to digital circuit design theory. In the domain of computer science, MIT faculty and researchers made fundamental contributions to cybernetics, artificial intelligence, computer languages, machine learning, robotics, and cryptography. At least nine Turing Award laureates and seven recipients of the Draper Prize in engineering have been or are currently associated with MIT.

Current and previous physics faculty have won eight Nobel Prizes, four Dirac Medals, and three Wolf Prizes predominantly for their contributions to subatomic and quantum theory. Members of the chemistry department have been awarded three Nobel Prizes and one Wolf Prize for the discovery of novel syntheses and methods. MIT biologists have been awarded six Nobel Prizes for their contributions to genetics, immunology, oncology, and molecular biology. Professor Eric Lander was one of the principal leaders of the Human Genome Project. Positronium atoms, synthetic penicillin, synthetic self-replicating molecules, and the genetic bases for Amyotrophic lateral sclerosis (also known as ALS or Lou Gehrig's disease) and Huntington's disease were first discovered at MIT.

\pagebreak
