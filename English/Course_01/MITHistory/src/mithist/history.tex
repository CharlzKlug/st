\section{The History}

The history of the Massachusetts Institute of Technology can be traced back to the 1861 incorporation of the "Massachusetts Institute of Technology and Boston Society of Natural History" led primarily by William Barton Rogers.

\subsection{Vision and mission}

As early as 1859, the Massachusetts State Legislature was given a proposal for use of newly opened lands in Back Bay in Boston for a museum and Conservatory of Art and Science.

On April 10, 1861, the governor of the Commonwealth of Massachusetts signed a charter for the incorporation of the "Massachusetts Institute of Technology and Boston Society of Natural History" which had been submitted by William Barton Rogers, a natural scientist. Rogers sought to establish a new form of higher education to address the challenges posed by rapid advances in science and technology in the mid-19th century, that he believed classic institutions were ill-prepared to deal with.

With the charter approved, Rogers began raising funds, developing a curriculum and looking for a suitable location. The Rogers Plan, as it came to be known, was rooted in three principles: the educational value of useful knowledge, the necessity of “learning by doing,” and integrating a professional and liberal arts education at the undergraduate level. MIT was a pioneer in the use of laboratory instruction. Its founding philosophy is "the teaching, not of the manipulations and minute details of the arts, which can be done only in the workshop, but the inculcation of all the scientific principles which form the basis and explanation of them;"

Because open conflict in the Civil War broke out only two days later on April 12, 1861, Rogers faced enormous difficulties raising funds to match conditional financial commitments from the state. Thus, his recruitment of faculty and students was delayed, but eventually MIT's first classes were held in rented space at the Mercantile Building in downtown Boston in 1865.

\subsection{Boston Tech (1865–1916)}

Construction of the first MIT building was completed in Boston's Back Bay in 1866 and would be known as "Boston Tech" until the campus moved across the Charles River to Cambridge in 1916.

At the Philadelphia Centennial Exhibition of 1875, Runkle was impressed by the work of the Russian Victor Della-Vos, who had introduced a pedagogical approach combining manual and theoretical instruction at the Moscow Imperial Technical Academy. Runkle became an advocate of this approach, introducing it at MIT.

MIT's financial position was severely undermined following the Panic of 1873 and subsequent Long Depression.[8] Enrollments decreased sharply after 1875 and by 1878, the university had abolished three professorships, reduced faculty salaries, and there was talk among members of the Corporation of closing the Institute.[9][10] In 1879, Runkle retired from a nine-year tenure as the MIT's second president trying to weather these challenges, but the board of trustees (the "MIT Corporation") was unable to secure a new successor and elected the seventy-five-year-old founder William Barton Rogers back to the post in the interim under his stipulated conditions that he be allowed to resign upon the discovery of a successor and 100,000 (2,303,000 in 2009) be raised to fund the Institute's obligations.[9]
An 1889 photogravure of the 1865 "Rogers" Building in the foreground with the 1883 "Walker" Building in the background.

Rogers wrote Francis Amasa Walker in June 1880 to offer him the Presidency. On May 30, 1882, during Walker's first Commencement exercises, Rogers died mid-speech where his last words were famously "bituminous coal".

Francis Amasa Walker as President of MIT

Walker established a new general course of study (Course IX) emphasizing economics, history, law, English, and modern languages.[15] Walker also set out to reform and expand the Institute's organization by creating an Executive Committee, apart from the fifty-member Corporation, to handle regular administrative issues and emphasized the importance of faculty governance by regularly attending their meetings and seeking their advice on major decisions.[16][17]

MIT's inability to secure a more stable financial footing during this era can largely be attributed to the existence of the Lawrence Scientific School at Harvard. Given the choice between funding technological research at the oldest university in the nation or an independent and adolescent institution, potential benefactors were indifferent or even hostile to funding MIT's competing mission.[18] Earlier overtures from Harvard President Charles William Eliot towards consolidation of the two schools were rejected or disrupted by Rogers in 1870 and 1878 and despite his tenure at the analogous Sheffield School, Walker also remained committed to MIT's independence from the larger institution.[19]

In light of the difficulties in raising capital for these expansions and despite its privately endowed status, Walker and other members of the Corporation lobbied the Massachusetts legislature for a 200,000 dollars grant to aid in the industrial development of the Commonwealth. After intensive negotiations that called upon his extensive connections and civic experience, in 1887 the legislature made a grant of 300,000 dollars over two years to the Institute, and would lead to a total of 1.6 million in grants from the Commonwealth to the Institute before the practice was abolished in 1921.[20]

Walker sought to erect a new building on to address the increasingly cramped conditions of the original Boylston Street campus located near Copley Square.[21] Because the stipulations of the original land grant prevented MIT from covering more than two-ninths of its current lot, Walker announced his intention to build the industrial expansion on a lot directly across from the Trinity Church fully intending that their opposition would lead to favorable terms for selling the proposed land and funding construction elsewhere.[22] With the financial health of the Institute only beginning to recover, Walker began construction on the partially funded expansion fully expecting the immediacy of the project to be a persuasive tool for raising its funds. The strategy was only partially successful as the 1883 building had laboratory facilities that were second-to-none but also lacked the outward architectural grandeur of its sister building and was generally considered an eyesore on its surroundings.[23] Mechanical shops were moved out of the Rogers Building in the mid-1880s to accommodate other programs and in 1892 the Institute began construction on another Copley Square building. New programs were also launched under Walker's tenure: Electrical Engineering in 1882, Chemical Engineering in 1888, Sanitary Engineering in 1889, Geology in 1890, Naval Architecture in 1893.[24]
A 1905 map of MIT's Boston campus.

Walker also sought to improve the state of student life and alumni relations by supporting the creation of a gymnasium, dormitories, and the Technology Club which served to foster a stronger identity and loyalty among the largely commuter student body.[25] Walker also won considerable praise from the student body by reducing the required time spent in recitation and preparation, limited the faculty to examinations lasting no longer than three hours, expanded entrance examinations to other cities, started a summer curriculum, and launched masters and doctoral graduate degree programs. These reforms were largely a response to Walker's on-going defense of the Institute and its curriculum from outside accusations of overwork, poor writing, unapplicable skills, and status as a "mere" trade school.[26] Between 1881 and 1897, enrollments quadrupled from 302 to 1,198 students, annual degrees granted increased from 28 to 179, faculty appointments quadrupled from 38 to 156, and the endowment grew thirteen-fold from $137,000 to $1,798,000.[10][27]

In the following years, the science and engineering curriculum drifted away from Rogers' ideal of combining general and professional studies and became focused on more vocational or practical and less theoretical concerns. To the extent that MIT had overspecialized to the detriment of other programs, "the school up the river" courted MIT’s administration with hopes of merging the schools. An initial proposal in 1900 was cancelled after protests from MIT's alumni.[28] In 1914, a merger of MIT and Harvard's Applied Science departments was formally announced[29] and was to begin "when the Institute will occupy its splendid new buildings in Cambridge."[30] However, in 1917, the arrangement with Harvard was cancelled due to a decision by the State Judicial Court.[31]

MIT was the first university in the nation to have a curriculum in: architecture (1865), electrical engineering (1882), sanitary engineering (1889), naval architecture and marine engineering (1895), aeronautical engineering (1914), meteorology (1928), nuclear physics (1935), and artificial intelligence (1960s).[32]
