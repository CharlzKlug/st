\Csection{Текст программы}

\begin{lstlisting}[style=customoz, caption=widget.h]
#ifndef WIDGET_H
#define WIDGET_H

#include <QWidget>
#include <QPainter>
#include <cmath>
#include "Sleeper.h"
#include <QTimer>
#include <QLabel>
#include <QVBoxLayout>
#include <QLineEdit>
#include <QPushButton>

class Widget : public QWidget
{
    Q_OBJECT
private:
    int R, r, n, x, y;
    QPolygon* polygon;
    int angle;
    int angleStep; // Шаг приращения угла шатуна
    int circleAngle; // Угол поворота колеса
    QLineEdit* REdit;
    QLineEdit* NEdit;
    QLineEdit* rEdit;
    double DegreeToRadian(int);
    double RadianToDegree(double);
    bool isRotateArea(int);
    void around();
    void straightLine();
    int getRotateArea(int);
    int getTreshold(); // Пороговое значение поворота вокруг угла
    int circleRotate(int);
public:
    Widget(QWidget *parent = 0);
    ~Widget();
    void calcPolygon();
    void drawScene(int x, int y);
protected:
    void paintEvent(QPaintEvent *);
public slots:
    void onTimer();
    void newValues(); // Смена радиусов и углов
};

#endif // WIDGET_H
\end{lstlisting}

\begin{lstlisting}[style=customoz, caption=widget.cpp]
#include "widget.h"

void Widget::calcPolygon()
{
    polygon->clear();
    for (int i = 0; i < n; i++)
    {
        polygon->append(QPoint(cos(6.28*i/n)*R+width()/2, sin(6.28*i/n)*R+height()/2));
    }
}

Widget::Widget(QWidget *parent)
    : QWidget(parent)
{
    R = 50;
    r = 50;
    n = 3;
    polygon = new QPolygon();
    x = 0; y=0;
    angle = 0;
    circleAngle = 180;
    angleStep = 1;

    QTimer * myTimer = new QTimer(this);
    connect(myTimer, SIGNAL(timeout()), this, SLOT(onTimer()));
    myTimer->start(62);
    QWidget* controlWindow;
    controlWindow = new QWidget;
    QVBoxLayout* verticalLayout = new QVBoxLayout;

    // Радиус многоугольника
    QHBoxLayout* RHorLayout = new QHBoxLayout;
    QLabel* RLabel = new QLabel("Polygon radius:");
    RHorLayout->addWidget(RLabel);

    REdit = new QLineEdit(QString::number(R));
    RHorLayout->addWidget(REdit);
    verticalLayout->addLayout(RHorLayout);

    // Количество углов многоугольника
    QHBoxLayout* NHorLayout = new QHBoxLayout;
    QLabel* NLabel = new QLabel("Polygon angle count:");
    NHorLayout->addWidget(NLabel);

    NEdit = new QLineEdit(QString::number(n));
    NHorLayout->addWidget(NEdit);
    verticalLayout->addLayout(NHorLayout);

    // Радиус катящегося круга
    QHBoxLayout* rHorLayout = new QHBoxLayout;
    QLabel* rLabel = new QLabel("Polygon angle count:");
    rHorLayout->addWidget(rLabel);

    rEdit = new QLineEdit(QString::number(r));
    rHorLayout->addWidget(rEdit);
    verticalLayout->addLayout(rHorLayout);

    // Кнопка отправки значений
    QPushButton* ChangeValueButton = new QPushButton("Change values");
    verticalLayout->addWidget(ChangeValueButton);
    connect(ChangeValueButton, SIGNAL(clicked()), this, SLOT(newValues()));
    controlWindow->setLayout(verticalLayout);
    //RLabel.show();
    controlWindow->show();
}

Widget::~Widget()
{

}

void Widget::drawScene(int x, int y)
{
    QPainter painter(this);
    calcPolygon();
    painter.drawPolygon(*polygon);
    painter.setPen(Qt::blue);
    painter.drawText(width()/2, 50, "Degree: "+QString::number(angle));
    if (isRotateArea(angle))
    {
        painter.drawText(width()/2, 60, "Rotate area");
        painter.drawLine(x, y,
                         r*cos(DegreeToRadian(circleAngle - circleRotate(angle))) + x,
                         -r*sin(DegreeToRadian(circleAngle - circleRotate(angle)))+ y);

    }
    painter.drawEllipse(QPoint(x, y), r, r);
    painter.drawLine(width()/2, height()/2, x, y);
}

void Widget::paintEvent(QPaintEvent *)
{
    drawScene(x, y);
}

void Widget::onTimer()
{
    if (isRotateArea(angle))
        around();
    else straightLine();
    this->repaint();
    angle += angleStep;
    angle %= 360;
}

void Widget::around()
{
    int k = getRotateArea(angle);
    int rotateAngle = k*360/n;
    int sA = angle - rotateAngle;
    double drA = -DegreeToRadian(rotateAngle);
    double mA = DegreeToRadian(abs(sA));
    double asn = asin(R*sin(mA)/r);
    double oX = R + r*cos(mA + asn);
    double oY = r*sin(mA + asn);
    if (sA < 0) oY *= -1;
    x = oX*cos(drA) + oY*sin(drA) + width()/2;
    y = oY*cos(drA) - oX*sin(drA) + height()/2;
}

void Widget::straightLine()
{
    double partAngle = DegreeToRadian(180.0/n);
    double Rm = R + r/cos(partAngle);
    double remainderAngle = DegreeToRadian(angle%(360/n));
    int fullAngle = angle*n/360;
    double circlePart = 2*partAngle;
    double oldX = Rm*cos(partAngle)*cos(remainderAngle)/cos(partAngle - remainderAngle);
    double oldY = Rm*cos(partAngle)*sin(remainderAngle)/cos(partAngle - remainderAngle);
    double rotateConst = -fullAngle*circlePart;
    x = oldX*cos(rotateConst) + oldY*sin(rotateConst) + width()/2;
    y = oldY*cos(rotateConst) - oldX*sin(rotateConst) + height()/2;
}

double Widget::DegreeToRadian(int degree)
{
    return (3.141592653589793*degree)/180;
}

double Widget::RadianToDegree(double radian)
{
    return 180*radian/3.141592653589793;
}

bool Widget::isRotateArea(int angle)
{
    if (getRotateArea(angle) <= n) return true;
    return false;
}

int Widget::getRotateArea(int angle)
{
    int threshold = getTreshold();
    bool inRotateArea = false;
    int areaNumber = n+1;
    for (int k = 0; k <= n && !inRotateArea; k++)
        if (angle < (k*360/n + threshold) && angle > (k*360/n - threshold))
        {
            inRotateArea = true;
            areaNumber = k;
        }
    return areaNumber;
}

int Widget::getTreshold()
{
    double halfCirclePart = DegreeToRadian(180/n);
    return RadianToDegree(atan(r*sin(halfCirclePart)/(R + r*cos(halfCirclePart))));
}

int Widget::circleRotate(int phi)
{
    return phi + RadianToDegree(asin(R*sin(DegreeToRadian(phi))/r));
}

void Widget::newValues()
{
    R = REdit->text().toInt();
    r = rEdit->text().toInt();
    n = NEdit->text().toInt();
}
\end{lstlisting}

\begin{lstlisting}[style=customoz, caption=main.cpp]
#include "widget.h"
#include <QApplication>

int main(int argc, char *argv[])
{
    QApplication a(argc, argv);
    Widget w;
    w.show();

    return a.exec();
}
\end{lstlisting}
