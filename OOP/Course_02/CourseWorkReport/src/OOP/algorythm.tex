\Csection{Алгоритм программы}

Программа реализована в виде двух виджетов. Один виджет представляет основную канву на которой отображается анимация, другой виджет служит в качестве панели управления.

В основном виджете введены следующие функции и процедуры:

\begin{itemize}
\item{\verb|calcPolygon()| --- формирует $n$-угольник. В цикле вычисляются координаты вершин многоугольника, полученные точки заносятся в указатель \verb|polygon| для последующей отрисовки.}

\item{\verb|Widget(QWidget *parent)| --- конструктор класса. Задаёт следующие переменные: радиус окружности описанной вокруг многоугольника окружности (\verb|R = 50|), радиус окружности, вращающейся вокруг многоугольника (\verb|r = 50|), количество углов многоугольника (\verb|n = 3|), начальный угол (\verb|angle = 0|) и шаг приращения угла (\verb|angleStep = 1|). Здесь же создаётся виджет управления. Виджет управления позволяет менять размеры радиусов и количество углов. Для анимации вращения используется таймер класса \verb|QTimer|.}

\item{\verb|drawScene(int x, int y)| --- процедура отрисовки многоугольника и круга, вращающегося по этому многоугольнику.}

\item{\verb|paintEvent(QPaintEvent *)| --- слот для отрисовки.}

\item{\verb|onTimer()| --- слот реагирующий на событие от таймера. Предназначен для приращения угла поворота и для отрисовки движения окружности либо по прямой, либо вокруг угла многоугольника}

\item{\verb|around()| --- процедура вычисления координат $x$ и $y$ центра окружности, осуществляющей вращение вокруг угла многоугольника.}

\item{\verb|straightLine()| --- процедура вычисления координат $x$ и $y$ центра окружности, осуществляющей движение вдоль прямой.}

\item{\verb|DegreeToRadian(int degree)| --- функция перевода градусов в радианы.}

\item{\verb|RadianToDegree(double radian)| --- функция перевода радиан в градусы.}

\item{\verb|isRotateArea(int angle)| --- функция, определяющая, находится ли угол в участке поворота.}
\end{itemize}

