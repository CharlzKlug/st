\Csection{Введение}

Большую часть информации человек получает посредством зрения. По этой причине развитие разнообразных форм визуальной информации никогда не будет останавливаться. Примерами могут служить даже наскальные рисунки из времён первобытно-общинного строя. С развитием науки и техники также улучшались инструменты визуализации информации: изобретение печати, развитие живописи, возникновение театра, а в последствии и кино. С появлением электронно-вычислительных машин возникла возможность их применение в сфере графики. Очень быстро начала развиваться компьютерная графика. В наши дни компьютерная графика служит мощным инструментом как в сфере развлечений (компьютерные игры, спецэффекты и прочее), так и в научной сфере (визуализация поведения разнообразных реальных моделей, по разработанным математическим моделям).

В качестве задачи курсовой работы была выбрана проблема моделирования поворота трёхмерных объектов вокруг осей координат с последующей проекцией полученного преобразования на плоскость $OXY$ и отображением этой проекции на экран компьютера.

Такая постановка задачи позволяет в достаточно полной мере отразить полученные навыки работы с виджетами фреймворка Qt в объектно-ориентированном программировании. Помимо работы с виджетами в данной курсовой работе используется библиотека QPainter. Эта библиотека используется для рисования на некоторой области виджета. Дополнительно была рассмотрена возможность многопоточного программирования.


\pagebreak

