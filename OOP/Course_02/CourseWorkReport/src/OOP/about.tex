\Csection{Введение}

Целью данной курсовой работы является отработка навыков самостоятельного проведения исследований некоторой задачи, выявления закономерностей, формирование на базе этих закономерностей математической модели и реализация этой модели в виде программы для ЭВМ.

В качестве задачи курсовой работы была выбрана проблема моделирования вращения круга вокруг правильного $n$ угольника. Ставилась цель написать программу, способную визуально отобразить этот процесс на экране компьютера.

Такая постановка задачи позволяет в достаточно полной мере отразить полученные навыки работы с виджетами фреймворка Qt в объектно-ориентированном программировании. Помимо работы с виджетами в данной курсовой работе используется библиотека QPainter. Эта библиотека используется для рисования на некоторой области виджета.


\pagebreak

