\section{Заключение}

При выяснении смысла техники в философии оказалось, что технику воспринимали по-разному в различные времена истории. Рассмотрены наиболее существенные и интересные интерпретации техники и подходы к ее пониманию. Большинство ученых и мыслителей, которые вели исследование этой проблемы, считают, что

\begin{itemize}

\item{техника отражает в себе человека, поэтому она такая же противоречивая, как и человек;}
  
\item{человек не должен делать из техники самоцель;}
  
\item{ключ к решению проблемы человека в информационно-техническом обществе нужно искать только в гармонии техники и человека.}

\end{itemize}

Рассмотренный материал подводит нас к следующим выводам:

\begin{itemize}
  
\item{многие информационно-технические достижения не могут трактоваться исключительно с положительной стороны, они при детальном рассмотрении имеют отрицательные стороны, которые в совокупности способны разрушить человека;}
  
\item{часто положительные достижения информационно-технического развития носят локальный характер, и при этом их отрицательные стороны – глобальный или вообще скрытый, что способно привести к катастрофическим для человека последствиям в будущем.}
  
\item{в мире происходит определенный количественный рост технического развития, который, достигая критической точки, переходит в качество, и тогда техника вступает в некоторую новую фазу, важно, чтобы эта фаза не была губительной для человека, для этого необходимо уже сейчас обращать особое внимание и подвергать критике любые научные и информационно-технические предложения;}
  
\item{ученые и инженеры в любой отрасли науки обязаны следить за своей деятельностью, чтобы она не принесла человечеству беды ни в ближайшее время, ни в отдаленном будущем, а, следовательно, они несут ответственность за свою деятельность;}
  
\item{в информационно-техническом обществе техническая элита (ученые и инженеры) получает своеобразную власть, потому что обладает специализированным знанием, а значит, ответственность этой элиты за судьбу человечества возрастает еще больше;}
  
\item{выявилось существование определенного барьера между естествоиспытателями, представителями технических наук и учеными-гуманитариями, который является результатом узкой специализации технических наук и помехой инженерам в понимании и осмыслении роли техники в жизни человечества;}
  
\end{itemize}

Для преодоления негативных тенденций информационно-технического мира необходимо всесторонне исследовать феномен техники и специфику информационных достижений, увеличить гуманитарную составляющую в узкотехническом образовании при подготовке инженеров. Но прежде всего – независимо от профессиональной деятельности быть человеком моральным и дорожить общечеловеческими ценностями.
