\Csection{Постановка задачи}
\begin{center}
\begin{figure}
\begin{tikzpicture}
\draw (2, 0) -- (0, -2) -- (-2, 0) -- (0, 2) -- (2, 0);
\draw [->] (-3, 0) -- (5, 0);
\draw [->] (0, 3) -- (0, -3);
\draw (0, 0) node [below left] {$O$};
\draw (5, 0) node [below right] {$X$};
\draw (0, -3) node [below left] {$Y$};
\draw (3, 0) circle (1cm);
\draw [dotted] ({2+sqrt(2)/2}, {-sqrt(2)/2}) circle (1cm);
%\fill (0, 0) circle [radius=2pt];
\fill ({2+sqrt(2)/2}, {-sqrt(2)/2}) circle [radius=2pt];
\fill (3, 0) circle [radius=2pt];
\draw (0, 0) -- ({2+sqrt(2)/2}, {-sqrt(2)/2});
\draw (2, 0) node [above] {$A$};
\draw (3, 0) node [above] {$B$};
\draw ({2+sqrt(2)/2}, {-sqrt(2)/2}) node [right] {$C$};
\draw [dotted] (2, 0) -- ({2+sqrt(2)/2}, {-sqrt(2)/2});
\draw (0, -2) node [right] {$D$};
\fill (0, -2) circle [radius=2pt];
\end{tikzpicture}
\caption{Чертёж задачи}
\end{figure}
\end{center}

Требуется реализовать вращение круга с окружностью $r$ вокруг правильного $n$ угольника с радиусом описанной окружности $R$. Круг должен вращаться постоянно касаясь одной точкой правильного $n$ угольника.

На входе программа запрашивает число сторон правильного многоугольника, значение радиуса описанной вокруг многоугольника окружности и радиус круга.

В основном окне отображается процесс работы модели.

Основная задача заключается в определении координат центра окружности, производящей движение. В качестве аргумента функции вычисления координат центра окружности был взят угол $\alpha$ между направлением оси $OX$ и отрезка соединяющего точку начала координат $O$ с центром вращающейся окружности.

В этом случае движение окружности можно разбить на две составные части: движение вокруг угла (на рисунке точка $A$) и движение по прямой линии вдоль ребра многоугольника. Поскольку многоугольник правильный, то его возможно представить в виде множества отрезков, повёрнутых относительно начала координат на угол $\frac {360k} {n}$, где $k = 0, 1, 2, ...$ --- целое число, $n$ --- количество сторон или углов многоугольника. Тогда становится возможным определять положение центра окружности на отрезке $AD$ и в дальнейшем осуществлять поворот этого центра вокруг начала координат. Так можно описать движение центра окружности по всей фигуре.

Движение вокруг угла многоугольника возникает при угле

$$\alpha \in \left (\frac {360k} {n} - \arctg \left (\frac {r \sin(\frac {180} {n})} {R+r \cos(\frac {180} {n})} \right ),
\frac {360k} {n} +\arctg \left (\frac {r \sin(\frac {180} {n})} {R+r \cos(\frac {180} {n})} \right )\right )$$

В этом случае координаты $x'$ и $y'$ возле угла многоугольника можно вычислить по формулам:

\begin{gather*}
x' = R + r \cos\left(\alpha - \frac {360 k} {n} + \arcsin\left(\frac {R} {r} \sin \left(\alpha - \frac {360k} {n}\right)\right) \right) \\
y' = wr \sin \left(\alpha - \frac {360k} {n} + \arcsin \left(\frac {R} {r} \sin \left(\alpha - \frac {360k} {n}\right) \right) \right)
\end{gather*}

где $k$ --- порядковый номер угла, возле которого осуществляется движение, а $w$ --- принимает значение $-1$, если угол $\alpha$ меньше $\frac {360k} {n}$ или равен $1$, если угол $\alpha$ больше $\frac {360k} {n}$.

После этого осуществляется поворот на угол $\frac {360k} {n}$:

\begin{gather*}
x = x' \cos (\frac {360k} {n}) - y' \sin (\frac {360k} {n}) + P_x \\
y = y' \cos (\frac {360k} {n}) + x' \sin (\frac {360k} {n}) + P_y
\end{gather*}

здесь $P_x = \frac {\text{ширина экрана}} {2}$, а $P_y = \frac {\text{длина экрана}} {2}$.

При движении по прямой вычисления центра движущейся окружности будут такими:

\begin{gather*}
x' = \left(R + r \cos^{-1}\left(\frac {180} {n}\right) \right) \cos \left(\alpha \mod \frac {360} {n}\right) / \cos \left(\frac {180} {n} - \alpha \mod \frac {360} {n}\right) \\
y' = \left(R + r \cos^{-1}\left(\frac {180} {n}\right) \right) \sin \left(\alpha \mod \frac {360} {n}\right) / \cos \left(\frac {180} {n} - \alpha \mod \frac {360} {n}\right)
\end{gather*}

После чего нужно осуществить поворот:

\begin{gather*}
x = x' \cos(\beta) + y' \sin (\beta) + P_x \\
y = y' \cos (\beta) - x' \sin (\beta) + P_y
\end{gather*}

здесь $\beta = - ((\alpha n) \div 360) (\frac {180} {n})_{\text{radian}}$, где ``$\div$'' --- операция получения неполного частного, $()_{\text{radian}}$ --- перевод в радианы.
