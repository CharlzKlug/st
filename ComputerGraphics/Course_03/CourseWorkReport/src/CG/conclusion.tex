\section {Заключение}

Графический ускоритель за время своего существования прошёл большой путь от инструмента, единственное назначение которого заключается в простом отображении информации до мощного, быстродейственного и во многом универсального прибора, способного быть подспорьем в широком круге решаемых задач.

Такое быстрое развитие вычислительной способности техники основывается на уменьшении применяемого технологического процесса производства процессоров. На сегодняшний день самым ближайшим технологическим процессом является техпроцесс 7 нм (нанометров)\cite{samsung7nm}. Для сравнения: радиус самого большого ядра --- ядро атома цезия --- около 267 пикометров\cite{chemege}, то есть, диаметр ядра цезия около 534 пикометров. Разделив 7 нанометров на 534 пикометров получаем: $\frac {7 \times 10^{-9}} {534 \times 10^{-12}} \approx 0.013 \times 10^3 = 13$ раз. Естественно, что безгранично уменьшать технологический процесс не представляется возможным, поскольку нет способа сделать производство меньше размеров ядер. Отсюда можно заключить, что неизбежна остановка наращивания вычислительных мощностей.

Один из способов наращивания вычислительных мощностей --- это выполнение расчётов на кластерах графических процессоров. Но в этом случае придётся жертвовать размерами помещений, энергозатратами на питание и охлаждение вычислительной техники.

Но при уменьшении технологического процесса в дело вступают явления квантовой механики. Оказывается, что теоретически и, отчасти, практически можно задействовать квантовую физику.

Впервые идею построения квантового компьютера высказал Ричард Филлипс Фейнман\cite{ran}. В качестве наименьшего носителя информации в квантовом компьютере используется кубит. Кубит, так же как и бит, может находиться в двух состояниях $|0\rangle$ и $|1\rangle$, но в отличие от бита может находиться в промежуточном состоянии (в суперпозиции). Кроме того, один кубит может быть связан с другим кубитом эффектом квантовой запутанности (один кубит ``знает'' что происходит с другим кубитом). Тогда при выполнении операции над одним кубитом также выполняется операция над другим кубитом. То есть, получаются параллельные вычисления не на процессорном, а на физическом уровне. Теоретически это может дать резкий рост производительности.

На данный момент, на практике, не всё работает гладко. Квантовые алгоритмы реализованы не для широкого круга задач. Если быть точнее, то с помощью квантовых алгоритмов, пока, получается решать достаточно узкий круг задач. Например, квантовый алгоритм Шора даёт возможность за достаточно короткое время вычислять простые множители больших чисел. Для обычных компьютеров такая задача трудновыполнима. За счёт сложности вычисления простых множителей функционирует криптографический алгоритм RSA. Если же развитие квантовых компьютеров получит достаточное развитие, то алгоритм RSA будет скомпрометирован\cite{ran}.

Кроме узости области применение квантовых алгоритмов есть и физические проблемы --- на кубиты могут оказывать влияние внешние помехи (например: скачки напряжения). Они вносят свою лепту в сложность производства квантовых компьютеров.

На момент написания работы, лидирующим производителем квантовых компьютеров является канадская фирма D-Wave\cite{dwave}. В её последнем продукте: квантовом суперкомпьютере D-Wave 2000Q используется 2000 кубитов. По заявлению D-Wave квантовый суперкомпьютер D-Wave 2000Q потребляет 25 киловатт электроэнергии в то время как традиционные суперкомпьютеры на той же задаче потребляют 2500 киловатт\cite{dwave2000q}.

Суммируя вышесказанное, можно предположить что в будущем применение видеокарт для параллельных вычислений уступит место квантовым вычислениям. И, возможно, даже видеокарты будут основаны на квантовых вычислениях.
