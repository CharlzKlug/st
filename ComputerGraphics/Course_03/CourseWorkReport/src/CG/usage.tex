\section{Применение параллельных вычислений}

Сегодня параллельные вычисления с применением графических процессоров используются во многих сферах начиная с медицинской науки и заканичивая, но не ограничиваясь прогнозированием погоды. Ниже представлены некоторые области применения графических ускорителей.

\begin{itemize}
\item{Медицина}
  \begin{itemize}
  \item{В Австралийском национальном университете проводятся исследования развития болезни Паркинсона с помощью методов машинного обучения\cite{almanac1}.}
  \item{В стартапе Zebra Medical Vision накопленные данные в клинических исследованиях используются для оценки рисков развития болезней, их предупреждения и помощи в организации и проведении профилактических лечений\cite{almanac1}.}
  \item{В нью-йоркской Школе медицины Икана при больнице Маунт-Синай глубокое обучение используется для анализа медицинских карт и определения пациентов, с высоким риском заболевания опасными болезнями в течении года\cite{almanac2}.}
 \end{itemize}

\item{Энергетика}

\begin{itemize}
  \item{Компания-стартап PowerScout из Калифорнии применяет GPU для прогнозирования, какие домохозяйства могут с большой долей вероятности приобрести солнечные панели. Разработки \selectlanguage{english}{Po\-wer\-Scout} также позволяют определить объём энергии, который можно получить с крыши одного дома. При этом необязательно самостоятельно проводить подсчёты. Необходимые данные извлекаются из коммерческих баз, спутниковых снимков. Также учитываются возможные затенения, например деревья рядом с домами, отбрасывающие тень на крыши\cite{almanac1}.}  
\end{itemize}

\item{Астрономия}

\begin{itemize}
  \item{В Университетском колледже Лондона вычисления на GPU используются для определения планет, на которых возможно поддерживать жизнь. Название программы --- RobERt (Ro\-bo\-tic Exo\-pla\-net Re\-co\-gni\-tion, ``роботизированное распознавание эк\-зо\-пла\-нет'')\cite{almanac2}.}
\end{itemize}

\item{Агрономия}

\begin{itemize}
\item{Немецкая компания PEAT применяет глубокое обучение для создания инструмента для диагностики и лечения болезней растений. Пользователь фотографирует больные растения и загружает полученные изображения в программу PEAT ``Plantix'', после чего получает рекомендации по лечению\cite{almanac3}.}
\end{itemize}

\item{Химия}

\begin{itemize}
\item{Abalone --- программа молекулярного моделирования, предназначенная для изучения молекулярной динамики био\-по\-ли\-ме\-ров\cite{chem}\footnote{\url{http://www.biomolecular-modeling.com/Abalone/}}.}

\item{CP2K --- программа для атомарной и молекулярной симуляции твёрдых тел, жидкостей, молекулярных и биологических сис\-тем\cite{chem}\footnote{\url{https://www.cp2k.org/}}.}


\end{itemize}
\end{itemize}
