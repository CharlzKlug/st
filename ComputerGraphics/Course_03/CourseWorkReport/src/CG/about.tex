\Csection{Введение}

Целью данной курсовой работы является получение знаний в области компьютерной графики, изучение основных принципов и приёмов построения современной графики.

В качестве темы курсовой работы была выбрана архитектура параллельных вычислений на GPU (англ. \textit{graphics processing unit, GPU}). В качестве примера в данной работе рассматриваются две наиболее распространённые технологии: NVIDIA CUDA и AMD FireStream.

Компьютерная техника в целом и компьютерная графика в частности продолжают быстро и бурно развиваться. Практически каждый год возникают новые и отмирают старые технологии. Если раньше компьютеры в основном могли выполнять вычисления лишь в однопоточном режиме, то теперь с появлением архитектур и машин с поддержкой многопоточных вычислений возникают новые горизонты возможностей в плане быстродействия и решения новых задач. Одной из таких возможностей стала реализация эффективных многопоточных вычислений на видеокартах. Изначально параллельные вычисления в видеокартах предназначались для обсчёта высококачественной графики. Но со временем исследователи обнаружили что с помощью видеокарт можно не только обрабатывать графику, но и решать задачи в таких областях как флуоресцентная микроскопия, молекулярная динамика, финансовая аналитика, ультразвуковые исследования и диагностика рака и многие другие.


\pagebreak

